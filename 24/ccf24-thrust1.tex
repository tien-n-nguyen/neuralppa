\section{Thrust 1. LLM-based Multi Agents for Predictive Static Program Analysis}
\label{sec:thrust1}

%\section{Approximation and Refinement}

\subsection{Code Approximation with LLM}

\begin{figure}[t]
	\centering
	\lstset{
		numbers=left,
		numberstyle= \tiny,
		keywordstyle= \color{blue!70},
		commentstyle= \color{red!50!green!50!blue!50},
		frame=shadowbox,
		rulesepcolor= \color{red!20!green!20!blue!20} ,
		xleftmargin=2.7em,xrightmargin=2.7em, aboveskip=1em,
		framexleftmargin=1.5em,
                numbersep= 5pt,
		language=Python,
    basicstyle=\tiny\ttfamily,
    numberstyle=\tiny\ttfamily,
    emphstyle=\bfseries,
                moredelim=**[is][\color{red}]{@}{@},
		escapeinside= {(*@}{@*)}
	}
\begin{minipage}{.5\textwidth}        
\begin{lstlisting}[caption = ]
(*@{\color{red}{First One-shot Prompt Setting@*)
(*@{\color{red}{System Prompt@*)
Task: Analyze the given partial Java code snippet and compiler errors. Generate the necessary code to complete the snippet by adding to the header only. Ensure the snippet becomes a compilable Java unit without modifying the original body. Provide type information for any new elements introduced in the code.
Input:
Partial Code Snippet: (*@{\color{blue}<code> @*)
Compiler Errors: (*@{\color{blue}<error> @*)
Output:
1. Code Approximation: Provide the enhanced code snippet.
2. Type Information: Detail the type information for each element introduced or used in the snippet.(*@\color{black}@*).

(*@{\color{red}{First One-Shot Prompt Setting@*)
(*@{\color{red}{User Prompt}@*)
Given a partial Java code snippet, compiler output/errors, and the expected outcome, your task is to generate the necessary code to complete the snippet. The additional code should address the issues indicated by the compiler output and achieve the desired outcome. Focus on enhancing the code header to make the snippet a compilable Java unit, without modifying the original code body.
Partial Code Snippet (Do Not Modify):
```
(*@{\color{blue}<code> @*)
```
\end{lstlisting}
\end{minipage}
\begin{minipage}{.5\textwidth}        
  \begin{lstlisting}[caption = ]
Compiler Output/Errors:
```
(*@{\color{blue}<error> @*)
```
Expected Outcome:
Using the information provided by the compiler, the model should fill in the missing information of the partial code snippet, focusing solely on enhancing the code header. The goal is to make the partial code snippet a compilable Java unit without modifying any part of the existing code body. ...
Your first task is to analyze the provided code, identify what are missing and complete the Java snippet. Do not modify the original code. Additionally, you do not need to write a method to solve missing API calls.
After completing the code, your second task is to fill in type informations ... For example ...
Here is an exampular procedure of how you should construct your answer.
Suppose You have given a partial code snippet
// begin of the partial code snippet
    <sample code snippet>
//end of the partial code snippet

Here is what your response is supposed to be:
Code Approximation:
```java
... // LLM1 approximated code here.
```
Type Information:
... // LLM1 proposed type information here.
Disclaimer: Do not include any explanation or comments.
\end{lstlisting}
\end{minipage}
\vspace{-12pt}
\caption{Initial Design of the Prompts for Approximation}
\label{fig:approx-prompt}
\end{figure}


% \input{technical-details/first-usr-prompt}

The initial stage of our framework involves Code Approximation, aimed at roughly filling in missing details and type details within incomplete code snippets with the help of a compiler. One could use GPT~\cite{ChatGPT}, referred to as LLMs. While GPT is utilized here, any LLM could be suitable. GPT aids in approximating the code completion since the precise edits made by developers are unknown. Our objective is to adequately fill in missing information at the class level to facilitate program analysis and derive results for the original incomplete code.
%
To precisely navigate the LLM through code approximation, we utilize a one-shot prompting approach that includes a definitive example of the desired fix and type information alongside the expected output format (Figure~\ref{fig:approx-prompt}). This prompt meticulously instructs the LLM to concentrate on class-level completions and specifies the task of reconstructing initialization variables and re-establishing missing setup method calls essential for API interactions. By presenting a representative example within the prompt, we crystallize the requirements, guiding the LLM to replicate this pattern in its code generation and ensuring that the output adheres closely to the structural and functional expectations set forth for the task.

On the left side of Figure~\ref{fig:approx-prompt} is our system prompt template. The System Prompt directs the LLM to analyze a partial Java code snippet along with its compiler errors and to augment the code by exclusively adding to the header. The completion must render the snippet a compilable Java method without altering the body. It also requires specifying type information for any new code elements. The prompt refines the task into two clear outputs: the first is the Code Approximation, where the LLM provides the amended header, and the second is Type Information, detailing the data types for each element involved in the fix.

On the right side of Figure~\ref{fig:approx-prompt} is an example of the user prompt; the LLM is tasked with analyzing a provided partial Java code snippet and associated compiler errors. The objective is to enhance the code by modifying the header section exclusively, ensuring the snippet evolves into a compilable Java unit while keeping the original code body intact. This focused directive requires the LLM to interpret the compiler output and errors, using these insights to determine the precise additions needed in the code's header for successful compilation.

During the first approximation, we will send all compiler's output to the LLM, and the LLM's response must address the identified compiler issues and align with the expected outcome detailed in the example from the user prompt. After enhancing the code, the LLM's secondary task involves detailing the type information for any introduced or utilized elements within the snippet, ensuring clarity and consistency in the code's type usage. This process underscores the LLM's role in not just code completion but also in fostering an understanding of the type system within the Java environment.

%The first step of our framework is Code Approximation, which aims to
%approximately filling in the missing information for the incomplete
%code snippet. We leverage GPT~\cite{ChatGPT} in our implementation,
%designated as LLM1 in Fig.~\ref{fig:overview}. In practice, any
%Large Language Model (LLM) could be applicable. We use GPT to
%approximately complete the code as we do not know the exact code that
%will be edited by developers. Our goal is to sufficiently fill in the
%missing information at the method level in order to facilitate the
%program analysis and derive the results for the original incomplete
%code.

%To guide LLM1 in predicting the missing code, we construct a prompt
%that articulates the specific requirements in our approximation
%task. The prompt explicitly instructs LLM1 to focus on method-level
%completion, emphasizing repairing initialization variables and
%restoring missing setup method calls for the API calls in the method's
%function. The following points detail the aspects covered in our
%prompts.

%The System Prompt instructs LLM1 to prioritize the method's structure
%and confine enhancements strictly to method level. Since the missing
%information may include global variable definitions, LLM1 is asked to
%adapt and re-frame these within the local scope of the method,
%ensuring that variables potentially relevant on a global scale are
%instead treated as local entities, preserving the method's
%self-contained and facilitating integration with the existing code
%base.

%In our User Prompt, in addition to method-level completion, {\tool}
%directs LLM1 to analyze within the method's scope: it must identify
%and repair any missing initialization sequences and infer the types
%and uses of variables based on the context within the given
%partial code. So, as LLM1 completes local initialization, it performs
%type reasoning simultaneously to maintain the logical flow and ensure
%syntactical integrity with the existing partial code.

\subsection{Refinement for Validation}

% \input{technical-details/self-correction-sys-prompt}
\begin{figure}[t]
	\centering
	\lstset{
		numbers=left,
		numberstyle= \tiny,
		keywordstyle= \color{blue!70},
		commentstyle= \color{red!50!green!50!blue!50},
		frame=shadowbox,
		rulesepcolor= \color{red!20!green!20!blue!20} ,
		xleftmargin=2.7em,xrightmargin=2.7em, aboveskip=1em,
		framexleftmargin=1.5em,
                numbersep= 5pt,
		language=Python,
    basicstyle=\tiny\ttfamily,
    numberstyle=\tiny\ttfamily,
    emphstyle=\bfseries,
                moredelim=**[is][\color{red}]{@}{@},
		escapeinside= {(*@}{@*)}
	}
\begin{minipage}{.5\textwidth}        
\begin{lstlisting}[caption = ]
(*@{\color{red}{Iterative Prompt Setting@*)
(*@{\color{red}{System Prompt@*)
You are provided with the original Java code snippet, a modified version of that code, and the latest compiler output/errors. Your task is to analyze these elements to identify and correct the errors in the modified code. Ensure your corrections align with the intended functionality and logic of the original snippet while addressing the specific issues highlighted by the new compiler errors.
Instructions:
Original Code Snippet: (*@{\color{blue}<original\_code> @*)
Modified Code with Errors: (*@{\color{blue}<modified\_code>@*)
New Compiler Output/Errors: (*@{\color{blue}<new\_error>@*)
\end{lstlisting}
\end{minipage}
\begin{minipage}{.5\textwidth}        
\begin{lstlisting}[caption = ]
(*@{\color{red}{Iterative Prompt Setting@*)
(*@{\color{red}{User Prompt@*)
After incorporating the suggested code enhancements to the original code snippet, the code was recompiled and resulted in new compiler output/errors. Your task is to analyze these new errors, understand the context of both the original and modified code, and apply further modifications to correct the errors without altering the core functionality or logic of the original code.
Input:
Partial Code Snippet (Do Not Modify): 
```
(*@{\color{blue}<original\_code> @*)
```
Modified Code with Errors: 
```
(*@{\color{blue}<modified\_code>@*)
```
New Compiler Output/Errors: 
```
(*@{\color{blue}<new\_error> (with error specific prompt)@*)
```
Expected Outcome:
(*@{\color{blue}{**Same as First One-Shot Prompt Setting**@*)
Here is an exampular procedure of how you should construct your answer.
(*@{\color{blue}{**Same as First One-Shot Prompt Setting**@*)
Disclaimer: Do not include any explanation or comments.
\end{lstlisting}
\end{minipage}

\vspace{-12pt}
\caption{Prompt for Refinement Loop in {\tool}}
\label{fig:refine-prompt}
\end{figure}


Following LLM1's initial attempt, we proceed to the refinement phase, aimed at iteratively enhancing the generated code until it meets the criteria for program analysis. Validation of this refinement can be achieved through various mechanisms, including a compiler, program analysis tools, verification tools, specification checking tools, model checkers, etc.

%In this study, to demonstrate our framework, we utilize a compiler to verify that the augmented code remains syntactically correct and preserves the logic and structure of the original snippet. The structure of our prompt for refinement via a compiler is displayed in Figure~\ref{fig:refine-prompt}. 

\subsubsection{Validation and Error-Specific Feedback Loop}

The verifier checks the LLM1-generated code for errors. If the code passes, it is accepted for further analysis. If not, we will use the specific errors to guide the next steps.
%\subsubsection{Error-Specific Feedback Loop}
%\begin{figure}[t]
	\centering
	\lstset{
		numbers=left,
		numberstyle= \tiny,
		keywordstyle= \color{blue!70},
		commentstyle= \color{red!50!green!50!blue!50},
		frame=shadowbox,
		rulesepcolor= \color{red!20!green!20!blue!20} ,
		xleftmargin=1.5em,xrightmargin=1em, aboveskip=1em,
		framexleftmargin=1.5em,
                numbersep= 5pt,
		language=Python,
    basicstyle=\tiny\ttfamily,
    numberstyle=\tiny\ttfamily,
    emphstyle=\bfseries,
		escapeinside= {(*@}{@*)}
	}
\begin{minipage}{0.45\textwidth}        
\begin{lstlisting}[caption = ]
(*@{\color{red}{Error Specific Prompt@*)
(*@{\color{red}{<modified\_code>@*)
LLM Generated Code Snippet : 
import oopsla.util.MyPackage;
public class MyClass {
    private static final int GLOBAL_NUMBER = 0;
    @Override
    public void main(String[] args, SessionFactory sessionFactory) {
        if(MY_VALUE) {
            return null;
        }
        else {
            ;
        }
    }
}

\end{lstlisting}
\end{minipage}
\begin{minipage}{0.45\textwidth}        
\begin{lstlisting}[caption = ]
(*@{\color{red}{Compiler's Output@*)
(*@{\color{green}{test.java:3:  error: class MyClass is public, should be declared in a file named MyClass.java@*)
public class MyClass {
       ^ 
(*@{\color{green}{test.java:2: error: package oopsla.util does not exist@*)
import oopsla.util.MyPackage;
                  ^
(*@{\color{green}{test.java:6: error: cannot find symbol@*)
    public void main(String[] args, SessionFactory sessionFactory) {
                                    ^
  symbol:   class SessionFactory
  location: class MyClass
test.java:5: error: method does not override or implement a method from a supertype
    @Override
    ^
(*@{\color{green}{test.java:7: error: cannot find symbol@*)
        if(MY_VALUE) {
           ^
  symbol:   variable MY_VALUE
  location: class MyClass
test.java:7: error: illegal parenthesized expression
        if(MY_VALUE) {
          ^
test.java:8: error: incompatible types: unexpected return value
            return null;
7 errors

\end{lstlisting}
\end{minipage}
\begin{minipage}{0.45\textwidth}        
\begin{lstlisting}[caption = ]
(*@{\color{red}{<new\_error> (with error specific prompt)@*)
error: method does not override or implement a method from a supertype at:
@Override
error: illegal parenthesized expression at:
if(MY_VALUE) {
error: incompatible types : unexpected return value at:
return null;
3 errors
To fix these errors:
for 'method does not override or implement a method from a supertype' error, make sure the method signature correctly matches an existing method in the supertype.
for 'illegal parenthesized expression' error, make sure the parentheses are correctly placed.
for 'unexpected return value' error, make sure the method's return type matches the actual value returned and adheres to Java's return value conventions
\end{lstlisting}
\end{minipage}
\vspace{-12pt}
\caption{Normal Error Message From Compiler{\tool}}
\label{fig:err-speciic-prompt}
\end{figure}

Each unique error message is examined to guide the next iteration of code generation for every identified error type. The revised prompt is crafted to direct LLMs, encapsulating the essence of the necessary correction and ensuring that subsequent code generation addresses the identified issues. LLMs is provided with the original partial code, its previous output, and the new error-specific prompt as inputs for the next iteration. This iterative process continues, with each cycle targeting the elimination of a specific category of errors flagged by the compiler.

%As the compiler identifies errors, all distinct error messages are analyzed to inform the next iteration of code generation for each identified error type. The newly crafted prompt is designed to guide LLM1, encapsulating the required correction's essence and ensuring that LLM1's subsequent code generation focuses on resolving the identified issues. LLM1 receives the original partial code, its previous output, and the new, error-specific prompt as inputs for the next iteration. This process is repeated, with each cycle aimed at eliminating a specific category of errors, as highlighted by the compiler.

%\subsubsection{Convergence on 'Symbol Not Found' Errors}:

The iterative refinement process is designed to address and resolve all detectable errors until only specific errors remain. Especially the \code{'Symbol Not Found'}, \code{'Package Does Not Exist'}, and \code{'Class name and File name do not match'} errors.
This specific focus is because, given that LLMs' task is to fill in missing information and complete the Java code snippet, it might introduce variables or methods that reference types from external libraries or packages not present in the environment.
%In a real-world development setting, it's impractical to expect that every possible package or library referenced by GPT's suggestions would be available or installed in the developer's environment.
Ignore the 'Class name and File name do not match' errors simply because, with LLMs' current potential, it is not possible for LLMs to create a Java file specifically and align it with the class name inside.

%Thus, by allowing the refinement process to narrow down to specific errors, we acknowledge and accommodate the limitations imposed by the environment, focusing our correction efforts on syntactical and structural accuracy within the method-level scope while acknowledging the external dependencies that might be unresolved within the current context.

%\subsubsection{Sample Showcase of error-specific-prompt crafting}

%Here is an example of how we will craft our error-specific-feedback prompt (Fig. ~\ref{fig:err-speciic-prompt}), represented as \code{<new\_error>} in Fig.~\ref{fig:refine-prompt}. In this example, the error-specific prompt is a strategic tool designed to instruct the LLM on addressing and correcting specific errors from the compiler's output while intentionally disregarding others.

%The LLM-generated code snippet, displayed in the top box, has been compiled, yielding the errors listed in the middle part. Four of these errors, highlighted in green, must be addressed in refinement, as per the error-specific prompt in the third part. These ignored errors are:

%1. The class name and file name do not match
%2. The package \code{oopsla.util} does not exist.
%3. The \code{SessionFactory} symbol cannot be found.
%4. \code{MY\_VALUE} symbol cannot be found.

%The bottom box then presents the new error-specific prompt, which filters out the green-highlighted errors and explicitly guides the LLM to focus only on the remaining issues identified by the compiler. It specifically instructs the LLM to correct the syntax and logic according to the code's original intent, such as ensuring that the method signature matches any existing superclass methods and that the return type is consistent with Java's conventions. This tailored approach allows the LLM to hone in on the most critical errors that need to be addressed, optimizing the refinement loop for efficiency and accuracy in code correction.

%=============================== END OF REMOVAL =====================

%\begin{figure}[t]
%	\centering
%	\lstset{
%		numbers=left,
%		numberstyle= \tiny,
%		keywordstyle= \color{blue!70},
%		commentstyle= \color{red!50!green!50!blue!50},
%		frame=shadowbox,
%		rulesepcolor= \color{red!20!green!20!blue!20} ,
%		xleftmargin=1.5em,xrightmargin=0em, aboveskip=1em,
%		framexleftmargin=1.5em,
%               numbersep= 5pt,
%		language=Java,
%    basicstyle=\scriptsize\ttfamily,
%    numberstyle=\scriptsize\ttfamily,
%    emphstyle=\bfseries,
%                moredelim=**[is][\color{red}]{@}{@},
%		escapeinside= {(*@}{@*)}
%	}
%\begin{lstlisting}[]
%if (codec != null) {
%  in = new LineReader(codec.createInputStream(fileIn),job);
%  end = Long.MAX_VALUE;
%} else {
%    if ( start != 0 ) {
%      skipFirstLine = true;
%      --start;
%      fileIn.seek ( start );
%    }
%    in = new LineReader(fileIn,job);
%}
%if (skipFirstLine) {
%  start += in.readLine (new Text(),0,(int) Math.min((long) Integer.MAX_VALUE, end-start));
%}
%\end{lstlisting}
%\vspace{-12pt}
%\caption{Incomplete code from S/O post \#16180130}
%\label{fig:motiv}
%\end{figure}

\begin{figure}[t]
	\centering
	\lstset{
		numbers=left,
		numberstyle= \tiny,
		keywordstyle= \color{blue!70},
		commentstyle= \color{red!50!green!50!blue!50},
		frame=shadowbox,
		rulesepcolor= \color{red!20!green!20!blue!20} ,
		xleftmargin=1.5em,xrightmargin=0em, aboveskip=1em,
		framexleftmargin=1.5em,
                numbersep= 5pt,
		language=Java,
    basicstyle=\tiny\ttfamily,
    numberstyle=\tiny\ttfamily,
    emphstyle=\bfseries,
                moredelim=**[is][\color{red}]{@}{@},
		escapeinside= {(*@}{@*)}
	}
\begin{lstlisting}[]
import java.io.IOException;
import java.io.InputStream;
import org.apache.hadoop.conf.Configuration;
import org.apache.hadoop.fs.FSDataInputStream;
import org.apache.hadoop.fs.Path;
import org.apache.hadoop.io.Text;
import org.apache.hadoop.io.compress.CompressionCodec;
import org.apache.hadoop...CompressionCodecFactory;...
public class FileProcessing implements ... {
  public processFile (Configuration job, String fileName, long start, long end, boolean skipFirstLine, CompressionCodec codec) throws IOException {
    FSDataInputStream fileIn = new FSDataInputStream(fileName);
    LineReader in;
    // incomplete code from S/O post 16180130
    if (codec != null) {
      in = new LineReader(codec.createInputStream(fileIn),job);
      end = Long.MAX_VALUE;
    } else {
      if (start != 0) {
         skipFirstLine = true;
          --start;
         fileIn.seek(start);
      }
      in = new LineReader(fileIn,job);
    }
    if (skipFirstLine) { 
      start += in.readLine(new Text(),0,(int) Math.min((long) Integer.MAX_VALUE,end-start));
        }
        this.pos = start;
    }
\end{lstlisting}
\vspace{-12pt}
\caption{The complete code predicted by {\tool}. Lines 14--29 correspond to incomplete code from S/O post 16180130.}
\label{fig:pca}
\end{figure}
