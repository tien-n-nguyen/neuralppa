\documentclass[11pt]{article}

\usepackage{graphicx,times}
\usepackage{wrapfig}
\usepackage{amsmath,epsfig}
\usepackage{setspace,array}
\usepackage{cite}

\usepackage{sectsty}
\sectionfont{\large}
\subsectionfont{\normalsize}
\subsubsectionfont{\normalsize}

\usepackage[compact]{titlesec}

\renewcommand{\theequation}{\thesection.\arabic{equation}}
\renewcommand{\baselinestretch}{1.0}

\newtheorem{Definition}{Definition}
\newtheorem{Claim}{Claim}
\newtheorem{Lemma}{Lemma}
\newtheorem{Theorem}{Theorem}
\newtheorem{Property}{Property}

\newcommand{\revise} {\bf}

\newcommand{\code}[1]{{\small\texttt{#1}}}
\newcommand{\op}{\tau}
\newcommand{\refac}{\rho}
\newcommand{\edit}{\sigma}
\newcommand{\T}{\theta}
\newcommand{\comp}{;}
\newcommand{\pre}{\prec_P}
\newcommand{\meth}{KSISA}

\newcommand{\MyParagraph}[1]{\textbf{#1}{ }}

\usepackage{vmargin}
\setpapersize{USletter}
\setmarginsrb{1.0in}{1in}{1.0in}{1in}%
           {0pt}{0mm}{0pt}{10mm}
\newcommand{\remove}[1]{}

\usepackage{tweaklist}
\renewcommand{\enumhook}{\setlength{\topsep}{0pt}%
  \setlength{\itemsep}{0pt}}
\renewcommand{\itemhook}{\setlength{\topsep}{0pt}%
  \setlength{\itemsep}{0pt}}
\renewcommand{\descripthook}{\setlength{\topsep}{0pt}%
  \setlength{\itemsep}{0pt}}

\begin{document}

% Collaborative Research: SHF: Small: 

\begin{center}
  {\bf Project Summary: SHF: Small: Neural Program Analysis Infrastructure and Its Applications}
\end{center}
\vspace{-.1in}


{\bf Overview.}
Detecting and fixing bugs are important in developing reliable and
high-quality software. Recognizing the importance of bug detecting and fixing processes,
several approaches have been proposed to support developers. With the
advances of {\em machine learning (ML)} (incl. {\em deep learning}
(DL)), the bottom-up, data-driven software engineering (SE) research
in automated bug detection and fixing that relies on ML/DL has drawn
extensive attention from academic and industry. For the entire process
of bug detecting/fixing to be fully effective, it is crucial to have
an effective {\bf code representation learning (CRL)}, which is aimed
to {\em produce effective, high-quality vector representations that
  capture the code features} that are important the bug detect-fix
ML/DL models. Code as data has rich structures and semantics with high
dimensional features, making large-scale learning and reasoning over
massive high dimensional features challenging. Thus, it is crucial to
{\em learn to convert high-dimensional, structural, semantics-rich
  features into low-dimensional vectors} to be used as the input for
the application-specific ML/DL models in the downstream bug
detecting/fixing tasks.  The state-of-the-art {\em learning to
  represent code features produced by any program analysis algorithms
  in term of vector representations, is still in its infancy
  stage}. Most of the existing approaches in SE and PL for bug
detection and fixing lack adequate research in CRL and comprehensive
knowledge on advantages of CRL in different bug detect-fix
tasks. They simply use {\em off-the-shelf, general-purpose vector
  learning techniques in ML or SE}, leading to low performance of the
entire bug detecting/fixing process.

In our preliminary work, we have studied and evaluated several
existing CRL approaches to bug detect-fix tasks. Our empirical results
showed that {\em there is no silver bullet CRL for all bug detect-fix
  models}, i.e., general-purpose CRL techniques are not effective for
all those models. Our philosophy is that the modeling and learning of
representations of program elements and their features must be
accurate, effective, and especially customized to specific
algorithms/models for bug detection/fixing.

\noindent {\bf Intellectual Merits.}  Aiming to have an effective
customization for CRL in bug detect-fix tasks, we propose the
following thrusts of research: (1) {\bf Design Framework and
  Environment for Code Representation Learning (CRL)}. We aim to
abstract our CRL designing experience and define the conceptual CRL
design framework with design principles and building blocks: entities,
relations among entities, representation learning models generating
code vectors. We will build a visual environment in which users can
graphically customize/add their own blocks at any design points, and
our editor recommends the best choices learned from practice and they
use them to define their workflows.
(2) {\bf Quality Evaluation Framework for Code Representation
  Learning.} If the vectors do not represent well the features that
they are supposed to capture (e.g, code structures or dependencies),
the models for the downstream tasks cannot perform well. Thus, we
propose a quality evaluation framework with various metrics that rely
on vector locality and vector offsetting to measure the quality of
vector representations. (3) {\bf Applications of CRL Framework in Bug
  Detection, Testing, Fault Localization, and Auto Program Repair.}
We will leverage our CRL framework to advance the state-of-the-art bug
detection/fixing research in those areas. The CRL framework
will be used to make innovation on code RL, test coverage RL, program
dependence RL, code transformation RL, etc.

\noindent {\bf Broader Impacts.}  (1) Our results will be {\em
  transformative and directly benefit to our society}, leading to
increasing developers’ productivity and software quality. Our
validation involves students and professionals, promoting teaching and
learning of software qualities that have wide impacts in industry and
academic communities. (2) Our results will {\em foster research
  activities} in related fields such as deep learning and software
quality and reliability. This project will produce theoretical
concepts and techniques that are novel even in deep learning, e.g.,
novel neural networks for modeling and learning code. The collected
large-scale bug/fix corpus will be useful for software quality and
reliability research.
(3) The research will enhance the infrastructure for teaching and
research by providing tools and data sets for use by students and
practitioners, and for enhancement by other researchers. We will
provide related learning modules for students.

\noindent {\bf Keywords:} Code Representation Learning, Deep Learning, Code Vector Representation, Bug Detection/Fix.


%Finding and fixing bugs are vital to produce reliable and high-quality
%software. Failing to fix a bug could result in severe consequences.
%In 1996, the Ariane 5 rocket, the European Space Agency's \$1 billion,
%was destroyed less than a minute after launch, due to a bug in the
%on-board guidance computer program. A study commissioned by the US
%Department of Commerce' National Institute of Standards and Technology
%(NIST) concluded that software bugs, or errors, are so prevalent and
%so detrimental that they cost the US economy an estimated \$59 billion
%annually, or about 0.6 percent of the gross domestic product.

%\textbf{- We could learn to fix a bug from similar ones}.

%There is no known, established straightforward methodology for bug
%fixing. Thus, like other problem-solving tasks, developers generally
%base on their own knowledge and experience with the systems, or learn
%from the others to do this task. In other words, they could
%effectively find and fix a bug by consulting the similar bugs and
%fixes to the one they are dealing with. However, currently, such
%learning is still ad-hoc, manually, and un-systematically. For
%example, when facing a bug, people could go to a forum to post a
%question and wait for the help from the others. They could also
%compile the reference manuals, as well as reports, tutorials,
%discussions to find a suitable fix.

%If we have automatic tool support that captures knowledge about bugs
%and fixes and leverages it in fixing the recurring/similar bugs, we
%could reduce the cost for software development.

%\textbf{- Current approaches supporting fixing recurring bugs are still limited}.

%However, current tool support that captures knowledge of known fixes
%and leverage it in fixing/patching in similar bugs are still
%limited. Some automatic tools are limited in recognize and synchronize
%the fixes, etc. For example, there is no any extensive research that
%discover the cause, the nature, and the characteristics of recurring
%bugs and fixes? That is, why are they recurring? How popular they are?
%How they are alike? What features help us recognize their
%recurrence/similarity? 1. How could we identify the API-related code
%peers within and across software projects and assess their popularity?
%. Does the limited support to their co-evolution affect the quality
%of the software system and the effectiveness of the development
%process? 3. How could we synchronize their changes, i.e. given the
%fix/patch of a peer, recommend developers the fixes/patches for its
%other peers?

\end{document} 
