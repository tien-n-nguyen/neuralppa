\documentclass[11pt]{article}

\usepackage{graphicx,times}
\usepackage{wrapfig}
\usepackage{amsmath,epsfig}
\usepackage{setspace,array}
\usepackage{cite}
\usepackage{xspace}
\usepackage{enumitem}

\usepackage{sectsty}
\sectionfont{\large}
\subsectionfont{\normalsize}
\subsubsectionfont{\normalsize}

\usepackage[compact]{titlesec}

\renewcommand{\theequation}{\thesection.\arabic{equation}}
\renewcommand{\baselinestretch}{1.0}

\newtheorem{Definition}{Definition}
\newtheorem{Claim}{Claim}
\newtheorem{Lemma}{Lemma}
\newtheorem{Theorem}{Theorem}
\newtheorem{Property}{Property}

\newcommand{\revise} {\bf}

\newcommand{\code}[1]{{\small\texttt{#1}}}
\newcommand{\op}{\tau}
\newcommand{\refac}{\rho}
\newcommand{\edit}{\sigma}
\newcommand{\T}{\theta}
\newcommand{\comp}{;}
\newcommand{\pre}{\prec_P}
\newcommand{\meth}{KSISA}


\newcommand{\MyParagraph}[1]{\textbf{#1}{ }}

\usepackage{vmargin}
\setpapersize{USletter}
\setmarginsrb{1.0in}{1in}{1.0in}{1in}%
           {0pt}{0mm}{0pt}{10mm}
\newcommand{\remove}[1]{}

\usepackage{tweaklist}
\renewcommand{\enumhook}{\setlength{\topsep}{0pt}%
  \setlength{\itemsep}{0pt}}
\renewcommand{\itemhook}{\setlength{\topsep}{0pt}%
  \setlength{\itemsep}{0pt}}
\renewcommand{\descripthook}{\setlength{\topsep}{0pt}%
  \setlength{\itemsep}{0pt}}

\newcommand{\tool}{\textsc{NeuralPPA}\xspace}

\begin{document}

% Collaborative Research: SHF: Small: 



\begin{center}
  {\bf Project Summary: SHF: Small: Neural Program Analysis Infrastructure and Its Applications}
\end{center}
\vspace{-.1in}


%\noindent {\bf Overview.}

\section{Overview}

Analyzing code snippets is not straightforward as they are often
incomplete, un-parseable, contain declaration/reference ambiguity, and
may be interspersed between user comments. Such an infrastructure must
include fundamental supports/services at the structural and semantic
levels, thus enabling the program analysis techniques to be built
upon. The basic infrastructure for partial program analysis, i.e., for
analyzing incomplete code is not yet available. No such infrastructure
has lead to several limitations. For example, it is impossible to
utilize vulnerability detection tools to find vulnerabilities in code
snippets because they rely on program representations that cannot be
built for the incomplete code. In addition to vulnerability detection,
such an infrastructure is also beneficial to other software
engineering (SE) tasks that can tolerate a low level of errors and
imprecision in building program representations.

To this effect, we set out to investigate {\tool}, a {\em \underline{Neural} Network-Based \underline{P}rogram \underline{A}nalysis} infrastructure. We aim to establish {\em a scientific foundation, novel methodologies, frameworks, models, and algorithmic solutions for neural program analysis}. We address two major issues in Program Analysis:

(1) {\bf Enabling analysis of incomplete code fragments}, i.e.,
partial program analysis;

(2) {\bf Empowering existing PA tools by making them more sound and
  complete}. {\tool} will allow the construction of efficient program
analysis techniques for (partial) code on which downstream software
engineering applications can be built.

%\noindent {\bf Intellectual Merit.}

\section{Intellectual Merit}

%In this work,
Our key philosophy is that {\em the analysis of partial code can be
  learned from the analysis of entire programs in the wealth of
  information from large-scale, open-source software
  repositories}. We propose the following thrusts of
research. First, the basic infrastructure in {\tool} is the neural
structural analysis component. It learns from the syntactic structures
of the complete code in the training dataset collected from
large-scale code repositories, to derive the abstract syntax tree
(AST) that best represents the syntactic structure.
%Next, this component is to tag the code tokens with the types of the
%syntactic units.
Second, the basis components for several analysis techniques on the
semantics of the program include 1) the identification of the APIs of
the external libraries in the external references in the partial code,
2) the inference of the type information for the entities in the
partial code, and 3) the inference of the program dependencies among
the statements in the partial code. Third, symbolic execution performs
executing a program abstractly, so that one abstract execution covers
multiple possible inputs, which are assumed to have symbolic
values. We explore an AI area named neuro-symbolic
learning, which seeks to combine rule-based AI approaches
with modern deep learning techniques to facilitate symbolic execution.
%Symbolic execution is a means of analyzing a
%program to determine what inputs cause each part of a program to
%execute.
Our last thrust aims to evaluate our basic partial
program analysis infrastructure in a few software engineering applications.
%We choose the
%following software engineering applications:
%1) software vulnerability detection for code snippets, 2) fault
%localization, and 3) code completion.

%\noindent {\bf Broader Impacts.}

\section{Broader Impacts}


(1) {\tool} will be {\em transformative and directly benefit to our
  society}, leading to increasing developers’ productivity and
software quality.
%It enables efficient \textit{impact analysis} during the integration
%of code snippets by identifying the potential consequences of the
%change.
It enables \textit{partial program slicing} and \emph{empowers}
dynamic analysis by identifying additional path conditions (neural
constraints) for the SAT/SMT solvers which helps in exploring the
right subset of the symbolic state space.
%
Our validation involves students and professionals, promoting teaching
and learning of software qualities that have wide impacts in industry
and academic communities. (2) Our results will {\em foster research
  activities} in related fields such as deep learning and software
reliability. This project will also produce novelties in deep learning,
e.g., novel neural networks for code. (3) The research will enhance
the tools for teaching and research by providing tools and
data sets for use by students and practitioners, and for enhancement
by other researchers. We will provide related learning modules.

\noindent {\bf Keywords:} Deep Learning, Partial Program Analysis, Neural Networks.


%Finding and fixing bugs are vital to produce reliable and high-quality
%software. Failing to fix a bug could result in severe consequences.
%In 1996, the Ariane 5 rocket, the European Space Agency's \$1 billion,
%was destroyed less than a minute after launch, due to a bug in the
%on-board guidance computer program. A study commissioned by the US
%Department of Commerce' National Institute of Standards and Technology
%(NIST) concluded that software bugs, or errors, are so prevalent and
%so detrimental that they cost the US economy an estimated \$59 billion
%annually, or about 0.6 percent of the gross domestic product.

%\textbf{- We could learn to fix a bug from similar ones}.

%There is no known, established straightforward methodology for bug
%fixing. Thus, like other problem-solving tasks, developers generally
%base on their own knowledge and experience with the systems, or learn
%from the others to do this task. In other words, they could
%effectively find and fix a bug by consulting the similar bugs and
%fixes to the one they are dealing with. However, currently, such
%learning is still ad-hoc, manually, and un-systematically. For
%example, when facing a bug, people could go to a forum to post a
%question and wait for the help from the others. They could also
%compile the reference manuals, as well as reports, tutorials,
%discussions to find a suitable fix.

%If we have automatic tool support that captures knowledge about bugs
%and fixes and leverages it in fixing the recurring/similar bugs, we
%could reduce the cost for software development.

%\textbf{- Current approaches supporting fixing recurring bugs are still limited}.

%However, current tool support that captures knowledge of known fixes
%and leverage it in fixing/patching in similar bugs are still
%limited. Some automatic tools are limited in recognize and synchronize
%the fixes, etc. For example, there is no any extensive research that
%discover the cause, the nature, and the characteristics of recurring
%bugs and fixes? That is, why are they recurring? How popular they are?
%How they are alike? What features help us recognize their
%recurrence/similarity? 1. How could we identify the API-related code
%peers within and across software projects and assess their popularity?
%. Does the limited support to their co-evolution affect the quality
%of the software system and the effectiveness of the development
%process? 3. How could we synchronize their changes, i.e. given the
%fix/patch of a peer, recommend developers the fixes/patches for its
%other peers?

\end{document} 
