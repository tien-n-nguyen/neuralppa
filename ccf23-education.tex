%\section{Education and Dissemination Plan - Curriculum Development Activities}


\noindent {\bf Undergraduate and Graduate Software Engineering (SE) Education.} 
%PI Wang teaches fundamental and core SE courses NJIT: Building Web Applications (IS218) and System Design (IS390).
PI Nguyen is one of the key SE faculty members in the BS, MS, and Ph.D.
degree programs in SE at UT Dallas. 
%
He is one of the faculty who has initiated and contributed to
Undergraduate SE Program when he was at Iowa State University.
%He has
%successfully introduced several courses including Software
%Architecture and Design (CprE339) and Software Project Management
%(CprE329). 
Taking advantage of this project, PI Nguyen will introduce a deep
learning component and a new course in NLP+SE.
The key teaching philosophy in this course is the combination of theory and practice.
%in which students will be introduced different principles and theories in the application of NLP in SE artifacts. 
%The tentative modules include 1) basic principles, processes, and paradigms in NLP, 2) programming languages versus natural languages, 3) API usages and reuse, 4) Statisical models used in SE applications, 5) machine translation and code migration, 6) language models for source code, 7) applications of machine translation in SE, 8) word embeddings and SE applications, 9) deep learning and SE applications, etc.
%\noindent {\em Graduate SE Education} 
PI Nguyen will teach a newly developed course, 
Selected Topics on AI in SE and PL,
including relevant topics to this proposal, e.g, 
AI in bug detection, fault localization, automated repair. 
%The PI Nguyen works
%with other UTD faculty to develop a series of six to seven SE graduate
%courses that will be offered at least every semester. This year, the
PI Nguyen introduced a new graduate level course on ``AI/ML for
Code''. The PI will introduce a new graduate course on the topic of
NLP+SE.
%Tools developed by this research will be used in class projects. 
The course will focus on several SE advanced methods that
aim to help advance software engineering with AI/ML techniques. 
The tentative topics include 1) program analysis, 2) source code
analysis with AI/ML, 3) cross-language analysis between texts and code,
4) AI/ML for code, etc.
%4) code and text retrieval in SE applications, 
%3) NLP techniques and
%type inference, 4) NLP and de-ofuscation, 5) bug-fixing and
%machine learning.



