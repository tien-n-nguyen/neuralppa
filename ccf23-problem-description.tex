%\section{Introduction}

Visually impaired individuals encounter numerous challenges in their
daily lives. For those who are passionate about learning programming
and aspiring to become software developers, the hurdles they must
overcome are even more pronounced. They grapple with their visual
disabilities while navigating the intricacies of demanding programming
tasks, necessitating heightened mental focus and determination to
succeed. Despite those challenges, according to a recent survey by
StackOverflow~\cite{blind-code} on 64,000 software developers, among 4.4
million programmers in the US workforce, there are about 1/200 of them
($\approx$ 20,000) visually impaired programmers. The challenges for
those programmers and for those who have passion to become software
engineers can come from different angles, especially from tools
and technologies.

First, accessibility of learning materials: most programming
resources, including textbooks, online tutorials, and coding
platforms, heavily rely on visual content such as code examples,
diagrams, and graphics. Visually impaired learners may struggle to
access these materials effectively, making it challenging to grasp
programming concepts. Second, screen readers compatibility: the key
current technology to support visuall impaired programmers is {\em
screen reading}. Visually impaired programmers often rely on screen
readers to access digital content. However, not all programming
environments and tools are optimized for compatibility with screen
readers, which can hinder their ability to navigate code, debug
errors, or interact with the development environment efficiently.
Third, Integrated Development Environment (IDE) and editor
accessibility: IDEs and code editors usually have complex user
interfaces, and their accessibility for visually impaired individuals
can be limited. Some IDEs may not support screen readers, or their
layout and features may not be easily navigable with assistive
technology. Fourth, visual debugging: debugging to find errors in
source code isa important process. Debugging code often involves
visually inspecting variables, data structures, and program
flow. Visually impaired programmers may face difficulties in debugging
due to their reliance on screen readers, which might not adequately
convey complex visual information. Fifth, learning visual concepts:
some programming concepts are inherently visual, such as GUI design
and web development. For visually impaired individuals, understanding
and implementing such concepts can be challenging, and alternative
approaches may be required. Finally, access to assistive technology:
not all visually impaired individuals have access to or are familiar
with the latest assistive technologies, which could limit their
ability to participate fully in the programming world.

In this proposal, we advocate for a promotion of inclusivity of
visually-impaired programmers and aspiring ones. Creating accessible
learning resources, and raising awareness about the challenges faced
by visually impaired programmers can help create a more diverse and
inclusive programming community. We propose a paradigm shift in
creating technologies in supporting the visually-impaired people in
learning programming and the visually-impaired programmers. Instead of
heavily relying screen reading, we leverage the advances in generative
Artificial Intelligence (AI) to build assistive technologies in {\em
Voice-to-Code}. This will be feasible with the machine learning (ML)
advances in Voice-to-Text and Text-to-Code.




