\section{Related Work}
\label{sec:related}

{\bf Research on the programming experiences of VIPLs} is notably
lacking, both in terms of quantity and quality. Mealin and Murphy-Hill
conducted qualitative interviews~\cite{mealin-vlhcc12} with eight experienced blind
developers and discovered several noteworthy insights. Although some
of these developers had utilized popular IDEs, e.g., Eclipse and
Visual Studio, they primarily relied on text editors for their coding
tasks. Interestingly, the interviewees indicated a tendency to switch
between different editors or IDEs based on the specific demands of
their projects. This behavior suggests that certain aspects of IDEs
are indeed useful and practical for blind programmers. However, the
overall inaccessibility of these environments often discouraged them
from using them over alternative tools.
%This discovery strongly underscores the need for concerted efforts to
%make IDEs more accessible and inclusive for this community.
Furthermore, the study revealed that these developers either refrained
from using or rarely engaged with the tools and features provided
within the IDEs.
%The precise reasons for this behavior were not
%conclusively determined, and
The study proposed several potential hypotheses. These included the
challenges associated with navigating the user interfaces of IDEs or
the complexities involved in setting up these tools. This gap in
research can be attributed to the study's methodology, which relied on
qualitative interviews with a predetermined set of questions. While
this approach helped surface the most pressing issues as reported by
the developers themselves, it fell short in uncovering the underlying
causes of these problems, as the interviewees did not possess these
answers.
%Consequently, when conducting research with specialized user groups such as visually impaired individuals, it is imperative to meticulously design the research methodology to not only gather valuable insights but also address the "why" questions.
Mealin and Murphy-Hill's study~\cite{mealin-vlhcc12} also yielded valuable insights into
the primary challenges faced by blind software developers, including
(1) difficulties in gaining an overview of the code, whether their own
or someone else's; (2) challenges in searching for information within
the code; and (3) struggles with accessing the tool library and
discovering tools within the IDEs. These points serve as valuable
inspiration for the development of tools and products aimed at
enhancing the productivity of VIPLs.

In a parallel study, Albusays and Ludi conducted a quantitative survey~\cite{khaled16} involving 69 experienced blind programmers who self-reported their experiences working on programming projects. Their findings reinforced the challenges outlined by Mealin and Murphy-Hill~\cite{mealin-vlhcc12}. These developers encountered considerable difficulties in navigating through complex code structures and layouts. Additionally, some respondents expressed discomfort and hesitancy when seeking assistance from or collaborating with sighted fellow developers. In addition to these new findings, Albusays and Ludi's research echoed the shared observation of the lack of accessibility in IDEs and their tools for visually impaired users.

%Both studies, however, encountered a common limitation in their inability to uncover comprehensive explanations for this inaccessibility, a limitation stemming from their reliance on respondent-dependent research methodologies.

%============================

{\bf Accessible and assistive tools for visually impaired developers}.
In 2011, Stefik et al. introduced Sodbeans~\cite{stefik11}, an auditory
integrated development environment (IDE) that employs audio cues to
convey information about code to developers. A notable feature of
Sodbeans is its utilization of scoping cues, which signify
relationships between program constructs.
%For instance, numerical values like "1 nested loop" denote different
%levels of loops. Testing with users revealed that scoping cues enhance
%code comprehension uniformly across various experience levels of
%sighted programmers, with no adverse effects.
However, Sodbeans primarily focuses on improving code comprehension,
with its potential to address the primary issue identified by Mealin
and Murphy-Hill ~\cite{mealin-vlhcc12} – gaining an overview of code – considered a
secondary goal necessitating further trials with non-sighted
developers for verification. Sodbeans was later integrated into
Oracle's Netbeans as an IDE extension.

Before their work on Sodbeans IDE, Stefik {\em et al.} created the
Wicked Audio Debugger (WAD)~\cite{wad07}, an audio-based add-in debugger for
Microsoft Visual Studio in 2007. WAD aimed to assist blind developers
in code navigation and debugging, potentially saving time and effort
in bug identification and resolution. However, feasibility testing
involved ten sighted senior-level computer science students simulating
"blind" behavior without visual code access or prior structural
knowledge. Similar to the study~\cite{stefik11} on Sodbeans, the
positive impact on blind developers' productivity remained
inconclusive.

Similarly addressing this challenge, Baker {\em et al.} developed a
plug-in called StructJumper~\cite{BakerML15} for Eclipse to aid
visually impaired developers in comprehending and navigating code. The
plug-in comprises two main features, TreeView and Text Editor,
creating a hierarchical tree based on Java's nesting
structure. Effectiveness tests conducted with seven experienced blind
programmers showed a slight positive impact on task completion time.
%As of May 2023, no updated versions of StructJumper are publicly available on the internet.
However, being an Eclipse plug-in limits its accessibility for many blind programmers, as they must switch between different IDEs with exclusive assistive tools and plug-ins.

Although each of these three initiatives can provide valuable support
to blind software developers, they currently exist as separate
entities integrated into different IDEs (Netbeans, Eclipse, and Visual
Studio).

Armaly and McMillan~\cite{armaly-tse18} performed a study on program
comprehension strategies by blind and sighted
programmers. AudioHighlight~\cite{armaly-icsme18} renders the code
inside a web view and places HTML heading tags on the structural
elements of a source file such as classes, functions and control flow
statements. A rich body of literature describes accessibility tools
designed for blind programmers, and many of these tools have been
tested in diverse
environments~\cite{10.1145/2897586.2897616,10.1145/1056808.1057018,10.1145/800071.802260,10.1006/jnca.1998.0060,10.1145/2501988.2502047,10.1145/568600.568611,10.1145/3308561.3354616,10.1145/1168987.1169035,10.1145/1408760.1408761,10.1145/238386.238405,10.5555/1103141.1103147,10.1145/1953163.1953323,5306335,10.1145/1121341.1121427,10.1145/2889160.2889188,10.1145/2889160.2891041,10.1145/3132525.3132550,10.1145/3170427.3188696,10.1145/2661334.2661385,6344485,10.1145/292834.292839,10.1145/563517.563372,10.1145/2556288.2557073,5967168,10.1145/1029014.1028654,10.1145/354324.354356,4268242,10.1016/j.ijhcs.2011.07.002,10.1145/1953163.1953323,10.1145/2982142.2982168,casten2005knowledge,CHIANG2005394,crossland2014smartphone,10.1145/1357054.1357250,gill2013digital,horowitz2003influence,10.1145/274497.274512,10.1145/354324.354327,janiszewski2006low,10.1145/1639642.1639663,margrain2000helping,massof2002model,massof2002model,scott2002impact,theofanos2005helping,watson1997national}. However,
none of them addresses the aforementioned issues with generative AI as
in our prposal.

%This necessitates developers to install and use all three
%IDEs simultaneously and correctly incorporate the respective plug-ins,
%creating a time-consuming and demotivating barrier for visually
%impaired programmers seeking to optimize their work environment.
%Furthermore, these tools were designed primarily for experienced blind
%programmers rather than beginners, as evident from the selection of
%testers, who were either professionally trained in computer science or
%had years of experience in the field.
%Insights gathered from these initiatives may differ significantly from
%those of beginners just embarking on their programming.

%Despite prior efforts to understand the challenges faced by blind software developers and create supportive tools, numerous gaps remain that can be addressed to better serve this unique group of learners and workers. This study introduces Product P4H with the aim of addressing the four major gaps identified through an analysis of existing work:

%Our ongoing effort in an empirical study include (1) uncovering
%obstacles hindering the pursuit of programming education by the
%general blind population, (2) addressing the needs of novice
%programmers, (3) conducting user research to comprehensively
%understand the challenges faced by blind programmers, and (4)
%eliminating the fragmentation of existing solutions. The fifth gap,
%feasibility testing with a suitable user group (inexperienced blind
%programmers), is partially addressed through the integration of a data
%tracking system within the product. However, further steps to fully
%address this fifth gap will be discussed in the limitations and future
%work section.




