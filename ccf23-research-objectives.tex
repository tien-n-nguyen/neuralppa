\subsection{Research Objectives}

%\begin{figure}[t]
%    \centering
%    \includegraphics[width=0.83\textwidth]{graphs/neuralppa}
%    \includegraphics[width=0.92\textwidth]{figures/infra-design-3.png}
%    \vspace{-10pt}
%    \caption{{\tool}: Machine Learning-based Program Analysis Infrastructure}
%    \label{fig:arch}
%\end{figure}

We seek to create a paradigm shift in assisting the visually-impaired
programmers and learners with a Large Language Model
(LLM)-based Voice-to-Code ({\bf LLM-V2C}) framework. We aim to establish {\em
  a scientific foundation, novel methodologies, frameworks, models,
  and algorithmic solutions for Voice-to-Code assistive technology}.
%
To address the above issues with the current assistive technology for
visually-impaired programmers and learners (VIPLs), we embark
on the development of a groundbreaking framework that leverages voice
interaction and advanced AI technologies. We aim to focus our effort
in the following~thrusts:

{\bf Enabling Voice-Driven Interaction}: Our primary objective is to
empower visually-impaired programmers and students to interact
seamlessly with an IDE through the power of voice. This means that
instead of relying solely on audio cues, they will be able to
communicate their coding instructions and commands using spoken
language. By doing so, we aim to bridge the accessibility gap and
provide an intuitive and efficient way for visually-impaired
individuals to engage with programming tasks.

{\bf Voice-to-Text Conversion}: In our framework, the spoken word
becomes a bridge to the digital realm. We utilize state-of-the-art
voice recognition technology to accurately convert spoken language
into text. This initial step is crucial in translating the user's
voice commands and instructions into a format that can be further
processed by the system. This ensures that the users' intent is
accurately captured and preserved throughout the programming process.

{\bf Text-to-Code Generation with Large Language Models (LLMs)}: Upon converting
voice inputs into textual form, we introduce the power of generative
AI. A large language model (LLM) with
generative capabilities takes the converted text and transforms it
into actual code. This step involves understanding the context,
syntax, and semantics of the instructions, and generating
corresponding code segments. The AI's ability to comprehend natural
language ensures that the generated code aligns with the user's
intentions.

{\bf Code Playback for Verification}: An integral aspect of our
framework is ensuring that the generated code is faithful to the
user's input. After the AI generates the code, it is read back to the
user through synthesized speech. This auditory feedback mechanism
enables visually-impaired programmers and students to verify the
accuracy of the generated code. By engaging their sense of hearing,
users can catch any discrepancies or errors and provide timely
corrections.

  
{\bf Empowering Modifications and Iterations}: Flexibility and control
are central to our framework's design. The visually-impaired users
have the ability to modify and refine the generated code according to
their preferences and needs. Through voice commands or text-based
interactions, they can make alterations, experiment with different
approaches, and fine-tune the code to align with their coding
goals. This iterative process ensures that the users are active
participants in the programming tasks.

In summary, our innovative framework for VIPLs offers a transformative
way to engage with programming tasks. By harnessing the capabilities
of voice recognition, generative AI from LLMs, and auditory feedback,
we are working to create an environment that fosters inclusivity,
creativity, and empowerment. Through this initiative, we aspire to
break down barriers and make the world of coding accessible to all,
regardless of visual impairments, contributing to a more diverse and
vibrant programming community.

\subsection{Thrusts of Research}

We propose the following thrusts of research in {\tool} (see Table~\ref{tab:milestones}):

%\vspace{3pt}
%\noindent \textbf{Thrust 1. Neural Structural Analysis Infrastructure
%  \code{NeuralStruct}.} ({\em Section~\ref{}}) Source code has
%well-defined structures and semantics. Thus, the basic infrastructure
%in {\tool} is the neural structural analysis component.  This
%component has two main tasks. First, it learns from the syntactic
%structures of the complete code in the training dataset collected from
%large-scale code repositories, to derive the abstract syntax tree
%(AST) that best represents the syntactic structure of the given
%partial code, i.e., with the highest likelihood/probability.  The
%traditional lexical analyzer still works for partial code due to the
%independence nature of lexical tokens. The second task of this
%component is to tag the code tokens with the types of the syntactic
%units including the statement types (\code{if}, \code{for}, etc.),
%variables, fields, methods, classes, etc. Both of the tasks can be
%performed with our learning-based approaches in a dual-learning
%manner.

\vspace{3pt}
\noindent \textbf{Thrust 1. Empirical Study on Current Assistive
  Technology for VIPLs (on-going)} ({\em Section~\ref{sec:thrust1}})
In order to gain deeper insights into the challenges faced by
visually-impaired programmers and students in utilizing current
assistive technologies for programming or learning to program, we have
been conducting an empirical study. This study involves engaging a
diverse group of visually-impaired individuals with varying levels of
programming experience including the beginners. Through in-depth
interviews, surveys, and usability testing, we aimed to uncover the
specific roadblocks, limitations, and usability issues encountered
when utilizing existing assistive tools in programming
contexts. Participants were encouraged to share their personal
experiences, frustrations, and suggestions for improvement. By
systematically analyzing the data collected from this study, we sought
to identify common themes, technical obstacles, and areas where
assistive technologies fell short in meeting the unique needs of
visually-impaired individuals in the programming domain. Ultimately,
this empirical study serves as a foundational step toward informing
the development of more effective and inclusive assistive technologies
that cater to the diverse requirements of visually-impaired
programmers and students. We will expand our current preliminary study
for large groups of VIPLs.

%\vspace{3pt}
\noindent \textbf{Thrust 2. Generative Large Language Models (LLMs)
  for Voice-Driven Programming Framework.}  ({\em
  Section~\ref{sec:thrust2}}) Our initiative focuses on empowering
visually-impaired programmers and students by enabling them to
seamlessly interact with an IDE through voice commands. Rather than
relying solely on auditory cues, this approach allows them to
articulate coding instructions and commands using spoken language,
effectively bridging the accessibility gap and providing an intuitive
means for engaging with programming tasks. Through cutting-edge voice
recognition technology, spoken language is converted into text,
serving as a crucial initial step in translating users' voice inputs
into a processable format.

Subsequently, our framework integrates generative artificial
intelligence (AI) to facilitate the transformation of text back into
code. A sophisticated language model with generative capabilities
processes the textual input, considering the nuances of context,
syntax, and semantics to generate corresponding code segments that
accurately align with the users' intentions. To ensure the accuracy of
the generated code, our framework employs auditory feedback, where the
AI-generated code is read aloud to the user. This feedback mechanism
allows visually-impaired programmers and students to verify the code's
precision through their sense of hearing, enabling them to promptly
identify errors and discrepancies.

Central to our design is the empowerment of users through flexibility
and control. Visually-impaired individuals can actively modify and
refine the generated code to suit their preferences and coding
goals. This adaptability is facilitated through voice commands or
text-based interactions, allowing for alterations, experimentation
with diverse approaches, and fine-tuning of the code. This iterative
process ensures that users remain engaged and actively involved in the
programming tasks, ultimately fostering a sense of ownership and
inclusivity in their coding journey.

\vspace{3pt}
\noindent \textbf{Thrust 3. Voice-Driven Features for an IDE to
  support VIPLs.} ({\em Section~\ref{sec:thrust3}}) Shifting away from
          {\em Screen Reading} and introducing a revolutionary
          programming environment tailored to the needs of
          visually-impaired individuals, our cutting-edge platform
          offers a multifaceted approach to programming through
          seamless voice interaction. At the core of this environment
          lies direct communication with GPT, a sophisticated language
          model, which serves as a personalized programming
          assistant. Users can articulate their programming tasks
          through natural language voice commands, receiving real-time
          feedback, suggestions, and guidance to navigate coding
          challenges effectively. This feature not only democratizes
          the programming experience but also empowers VIPLs to engage
          in a more intuitive and productive coding process.

Our programming environment goes beyond voice interaction by enabling
the transformation of voice commands into tangible code. Leveraging
advanced voice recognition and generative AI technologies, users'
spoken instructions are converted into actual code snippets. This
bridges the gap between verbal expression and code creation, ensuring
that the users' intent is accurately translated into executable code
segments. The generated code is then read back to the users through
synthesized speech, allowing them to verify its accuracy and
completeness, while promoting a understanding of the
programming logic.

This innovative environment extends its functionality to seamlessly
integrate generated code with existing projects, simplifying the
collaborative process and enabling a cohesive workflow. In addition,
users can harness the power of voice to initiate debugging sessions,
identifying and rectifying errors through intuitive voice
commands. The environment also facilitates the generation of
comprehensive documentation, providing a holistic overview of the
codebase, which can be enriched and customized through voice
interactions. Furthermore, users can effortlessly create test cases
using voice commands, enhancing the testing and validation of their
code. By encompassing these features, our environment
empowers VIPLs with a versatile and inclusive
toolset, fostering an effective programming journey while mitigating
accessibility barriers.


%1) Voice interaction with ChatGPT on programming tasks

%2) Code Generation from Voice

%3) Reading back the generated code for verification

%4) Code integration with existing code

%5) Debugging code via Voice Commands

%6) Documentation generation

%7) Test case generation


\begin{table*}[t]
	\vspace{-15pt}
\begin{center}
{\footnotesize{
\begin{tabular}{cc}
\begin{tabular}[t]{|p{0.2in}|p{2.95in}|} 
\hline
\multicolumn{2}{|>{\columncolor[gray]{0}}c|}{\textcolor{white}
{\bf Year 1 Project Milestones \& Deliverables}}\\
\hline 
\hline
\multicolumn{2}{|c|}{\bf T1. Empirical Study on Assistive Technology for VIPLs}\\
\hline
    {\bf 1.1} & Ongoing survey and interviews\\
    {\bf 1.2} & Expanding the study\\
{\bf 1.3} & Evaluation of the survey results\\
\hline
\hline
\multicolumn{2}{|c|}{\bf T2. AI/ML for Voice-to-Code}\\ 
\hline
    {\bf 2.1} & AI/ML models and LLMs\\
\hline
%\hline
%\multicolumn{2}{|c|}{\bf Integrate Code Synthesis into Tools}\\
%\hline
%{\bf 1.5} & \goalOneFour.\\
%\hline
\multicolumn{2}{c}{}
\end{tabular}
&
\begin{tabular}[t]{|p{0.2in}|p{2.95in}|} \hline
\multicolumn{2}{|>{\columncolor[gray]{0}}c|}{\textcolor{white}
{\bf Year 2 Project Milestones \& Deliverables}}\\
\hline 
\hline
\multicolumn{2}{|c|}{\bf T2. AI/ML for Voice-to-Code}\\
\hline
{\bf 2.2} & Voice Recognition and Interaction with ChatGPT for VIPLs\\
{\bf 2.3} & LLMs for Code Generation and Reading Back\\
{\bf 2.4} & LLMs and Test Generation\\
%{\bf 2.5} & Evaluate CRL Framework with Existing Models\\
%{\bf 2.3} & \goalTwoThree.\\

\hline
\hline
\multicolumn{2}{|c|}{\bf T3. Voice-driven IDE for VIPLs}\\ 
\hline
%{\bf 3.1} & Design New Code Representations and Learning Models.\\
{\bf 3.1} & LLMs for Documentation Generation\\
%{\bf 2.4} & Advance FL and RT-CI Approaches.\\
%{\bf 2.5} & Advance Regression Testing in CI Approaches.\\
%{\bf 2.5} & Advance APR Approaches with Framework.\\
\hline
%\hline
%\multicolumn{2}{|c|}{\bf Community Involvement: Capacity Building}\\
%\hline
%{\bf 2.4} & \goalTwoFour.\\
%{\bf 2.5} & \goalTwoFive.\\
%{\bf 2.6} & \goalTwoSix.\\
%\hline
\multicolumn{2}{c}{}
\end{tabular}
\end{tabular}\\
\vspace*{-.3cm}
\begin{tabular}{c}\hline
\multicolumn{1}{|>{\centering\columncolor[gray]{0}}p{6.44in}|}{\textcolor{white}
{\bf Year 3 Project Milestones \& Deliverables}}\\
\hline
\end{tabular}\\
\vspace*{-.2cm}
\begin{tabular}{cc}
\begin{tabular}[t]{|p{0.2in}|p{2.95in}|}
\hline
\multicolumn{2}{|c|}{\bf T2. AI/ML for Voice-to-Code}\\
\hline
{\bf 3.2} & Code Integration with LLMs\\
{\bf 3.3} & Debugging and Fixing\\

%{\bf 3.3} & Testing on Models in IDE tools.\\
\hline
%\hline
%\multicolumn{2}{|c|}{\bf \goalTwo}\\ 
%\hline
%{\bf 3.3} & \goalThreeThree.\\
%\hline
\multicolumn{2}{c}{}
\end{tabular}
&
\begin{tabular}[t]{|p{0.2in}|p{2.95in}|}
\hline
\multicolumn{2}{|c|}{\bf T3. Voice-driven IDE for VIPLs}\\
\hline
%{\bf 3.1} & Design New Code Representations\\

{\bf 4.1} & Feature Integration in the IDE\\
{\bf 4.2} & IDE Integration and Testing\\

\hline
\multicolumn{2}{c}{}
\end{tabular}
\end{tabular}
\vspace{-15pt}
}}
\end{center}
\vspace*{-.3in}
%\caption{Tasks and Milestones. (Rep. = Representation)}
\caption{The 3-year schedule of Thrusts, Tasks, and Milestones of this proposal.}
%the schedule of Thrusts, Tasks, and Milestones of this proposal.
%\vspace{-10pt}
\label{tab:milestones}
\vspace{-10pt}
\end{table*}
%

