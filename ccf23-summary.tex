\documentclass[11pt]{article}

\usepackage{graphicx,times}
\usepackage{wrapfig}
\usepackage{amsmath,epsfig}
\usepackage{setspace,array}
\usepackage{cite}
\usepackage{xspace}
\usepackage{enumitem}

\usepackage{sectsty}
\sectionfont{\large}
\subsectionfont{\normalsize}
\subsubsectionfont{\normalsize}

\usepackage[compact]{titlesec}

\renewcommand{\theequation}{\thesection.\arabic{equation}}
\renewcommand{\baselinestretch}{1.0}

\newtheorem{Definition}{Definition}
\newtheorem{Claim}{Claim}
\newtheorem{Lemma}{Lemma}
\newtheorem{Theorem}{Theorem}
\newtheorem{Property}{Property}

\newcommand{\revise} {\bf}

\newcommand{\code}[1]{{\small\texttt{#1}}}
\newcommand{\op}{\tau}
\newcommand{\refac}{\rho}
\newcommand{\edit}{\sigma}
\newcommand{\T}{\theta}
\newcommand{\comp}{;}
\newcommand{\pre}{\prec_P}
\newcommand{\meth}{KSISA}


\newcommand{\MyParagraph}[1]{\textbf{#1}{ }}

\usepackage{vmargin}
\setpapersize{USletter}
\setmarginsrb{1.0in}{1in}{1.0in}{1in}%
           {0pt}{0mm}{0pt}{10mm}
\newcommand{\remove}[1]{}

\usepackage{tweaklist}
\renewcommand{\enumhook}{\setlength{\topsep}{0pt}%
  \setlength{\itemsep}{0pt}}
\renewcommand{\itemhook}{\setlength{\topsep}{0pt}%
  \setlength{\itemsep}{0pt}}
\renewcommand{\descripthook}{\setlength{\topsep}{0pt}%
  \setlength{\itemsep}{0pt}}

\newcommand{\tool}{\textsc{LLM-V2C}\xspace}

\thispagestyle{empty}

\begin{document}

% Collaborative Research: SHF: Small: 



\begin{center}
%  {\bf Project Summary: SHF: Small: Empowering Visually-Impaired
%    Programmers and Learners with Voice-Driven Coding Framework}
%\end{center}
{\large \bf SHF: Small: Empowering Visually-Impaired Programmers and Learners\\ with Voice-Driven Generative AI for Coding and Large Language Models}
\end{center}
\vspace{-.1in}


%\noindent {\bf Overview.}

\section{Overview}

Visually impaired individuals encounter numerous challenges in their
daily lives. For those who are passionate about learning programming
and aspiring to become software developers, the hurdles they must
overcome are even more pronounced. The challenges for them can come
from different angles, especially from tools and
technologies. Visually impaired programmers often rely on screen
readers to access digital content. However, not all integrated
development environments (IDEs) are optimized for compatibility with
screen readers, which can hinder their ability to navigate code, debug
errors, or interact with the IDE efficiently. Code editors usually
have complex user interfaces, and their accessibility for visually
impaired individuals can be limited. Debugging code often involves
visually inspecting variables, data structures, and program
flow. Visually impaired programmers may face difficulties in debugging
due to their reliance on screen readers. For learners, some
concepts are inherently visual, such as GUI design and web
development. For visually impaired individuals, understanding and
implementing such concepts can be challenging.

In this proposal, we advocate for a promotion of inclusivity of
visually-impaired programmers and aspiring ones. We propose a paradigm
shift in creating technologies in supporting the visually-impaired
people in learning programming and the visually-impaired
programmers. Instead of heavily relying screen reading, we leverage
the advances in generative Artificial Intelligence to build assistive
technologies in {\em Voice-to-Code}. This will be feasible with the
machine learning advances in Voice-to-Text and Text-to-Code.

%\noindent {\bf Intellectual Merit.}

\noindent {\bf Keywords:} Visually-impaired Assistive Programming; Generative
Machine Learning; Voice-to-Code.

\section{Intellectual Merit}

We seek to create a paradigm shift in assisting the visually-impaired
programmers and learners with a Large Language Model
(LLM)-based Voice-to-Code ({\bf LLM-V2C}) framework. We aim to establish {\em
  a scientific foundation, novel methodologies, frameworks, models,
  and algorithmic solutions for Voice-to-Code assistive technology}.
%
To address the issues with the current assistive technology for
visually-impaired programmers and learners (VIPLs), we embark on the
development of a groundbreaking framework that leverages voice
interaction and advanced AI technologies. We aim to focus our effort
in the following~thrusts: {\bf Enabling Voice-Driven Interaction}: Our
primary objective is to empower visually-impaired programmers and
students to interact seamlessly with an IDE through the power of voice
to code AI via LLMs. {\bf Voice-to-Code Conversion}: In our framework,
the spoken word becomes a bridge to the digital realm. We utilize
state-of-the-art voice recognition technology to accurately convert
spoken language into text. Upon converting voice inputs into textual
form, we introduce the power of generative AI. A large language model
(LLM) with generative capabilities takes the converted text and
transforms it into actual code. {\bf Code Playback for Verification}:
An integral aspect is ensuring that the generated code is faithful to
the user's input.
 
\section{Broader Impacts}


(1)
%{\tool} will be {\em transformative}.
Our novel environment equips individuals with visual impairments with
a versatile set of tools, enabling an interactive programming
experience while reducing accessibility obstacles.
%
%It enables \textit{partial program slicing} and \emph{empowers}
%dynamic analysis by identifying additional path conditions (neural
%constraints) for the SAT/SMT solvers which helps in exploring the
%right subset of the symbolic state space.
%
Our validation also involves students and professionals, promoting
teaching and learning of software qualities.
%that have wide impacts in industry
%and academic communities.
(2) Our results will {\em foster research activities} in related
fields such as deep learning and software engineering. This project
will also produce novelties in deep learning, e.g., novel neural
networks for code. (3) The research will enhance the tools for
teaching and research by providing tools and data sets for use by
students and practitioners, and for enhancement by other
researchers. Creating accessible learning resources, and raising
awareness about the challenges faced by visually impaired people
can help create a more diverse and inclusive community.
%We will provide related learning modules.




%Finding and fixing bugs are vital to produce reliable and high-quality
%software. Failing to fix a bug could result in severe consequences.
%In 1996, the Ariane 5 rocket, the European Space Agency's \$1 billion,
%was destroyed less than a minute after launch, due to a bug in the
%on-board guidance computer program. A study commissioned by the US
%Department of Commerce' National Institute of Standards and Technology
%(NIST) concluded that software bugs, or errors, are so prevalent and
%so detrimental that they cost the US economy an estimated \$59 billion
%annually, or about 0.6 percent of the gross domestic product.

%\textbf{- We could learn to fix a bug from similar ones}.

%There is no known, established straightforward methodology for bug
%fixing. Thus, like other problem-solving tasks, developers generally
%base on their own knowledge and experience with the systems, or learn
%from the others to do this task. In other words, they could
%effectively find and fix a bug by consulting the similar bugs and
%fixes to the one they are dealing with. However, currently, such
%learning is still ad-hoc, manually, and un-systematically. For
%example, when facing a bug, people could go to a forum to post a
%question and wait for the help from the others. They could also
%compile the reference manuals, as well as reports, tutorials,
%discussions to find a suitable fix.

%If we have automatic tool support that captures knowledge about bugs
%and fixes and leverages it in fixing the recurring/similar bugs, we
%could reduce the cost for software development.

%\textbf{- Current approaches supporting fixing recurring bugs are still limited}.

%However, current tool support that captures knowledge of known fixes
%and leverage it in fixing/patching in similar bugs are still
%limited. Some automatic tools are limited in recognize and synchronize
%the fixes, etc. For example, there is no any extensive research that
%discover the cause, the nature, and the characteristics of recurring
%bugs and fixes? That is, why are they recurring? How popular they are?
%How they are alike? What features help us recognize their
%recurrence/similarity? 1. How could we identify the API-related code
%peers within and across software projects and assess their popularity?
%. Does the limited support to their co-evolution affect the quality
%of the software system and the effectiveness of the development
%process? 3. How could we synchronize their changes, i.e. given the
%fix/patch of a peer, recommend developers the fixes/patches for its
%other peers?

\end{document} 
