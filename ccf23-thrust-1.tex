\section{Thrust 1. Empirical Study on Current Assistive Technology for VIPLs (on-going)}
\label{sec:thrust1}

In this section, let us describe an preliminary empirical study as
part of an on-going effort of our proposed project. Six participants,
aged 20 to 25, were purposefully selected from a list of recruited
participants. These participants were divided into two distinct
groups. The first group comprised two participants, P1.1 and P1.2, who
had pursued technical training after high-school for disabled
individuals. The second group consisted of four participants, namely
P2.1, P2.2, P2.3, and P2.4, who had been learning programming for less
than two years, either through formal institutions or
self-study. Prior to any research activities, participants were
presented with and consented to a Consent Form approved by the
university's IRB Committee.

Participant Group 1 underwent 60-minute interviews, aimed at gaining
insights into the obstacles hindering their programming learning. In
addition to introductory questions, the primary ones were as
follows:

\begin{enumerate}

\item What is your current occupation?
  
\item What motivated you to pursue a career in your chosen field?

\item Are you familiar with the concept of programming?

\item If yes, when did you first encounter programming?
  
\item Have you considered delving deeper into programming?

\item If yes, what led you to discontinue your programming endeavors?

\item What factors do you believe would help you engage in a learning program and pursue related~careers?

\end{enumerate}
  
Participant Group 2 participated in 30-minute interviews and 30-minute
usability testing sessions. These activities aimed to uncover
challenges faced by these participants when using popular integrated
development environments (IDEs) for novice programmers, such as Google
Colab and Visual Studio Code. During the initial 30 minutes, in
addition to introductory questions, the following questions were posed:

\begin{enumerate}

\item Which IDE(s) do you use regularly?
  
\item In your learning programming, can you recall moments when you encountered difficulties? If so, what made those situations challenging, and what kind of support do you think would have been helpful?

\item Have you ever found your IDE to be user-unfriendly? If yes, please describe the circumstances.

\item Kindly rank the issues you've encountered based on their frequency of occurrence.

\end{enumerate}

In the latter 30 minutes, we conducted usability testing sessions
involving five basic programming assignments. Each participant was
asked to complete these assignments in order of increasing
difficulty. This testing phase aimed to identify usability issues that
participants may not have mentioned in their interviews. Notably, this
approach distinguished the research design from previous studies by
incorporating direct observations as a source of insights. The
following are details of the five programming tasks assigned to
participants in Group 2: (1) Write a program to display "Hello World!"
(2) Create a simple if-else condition.  (3) Debug a provided if-else
condition with an indentation problem.  (4) Locate the definition of
variable x in a given code.  (5) Explain the operations of a provided
Python script.

