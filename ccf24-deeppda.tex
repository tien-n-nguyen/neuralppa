\subsection{Preliminary Work on Neural Program Dependence Analysis}
\label{sec:deeppda}

Let us explain our pre-training solution for statically building
program dependence graphs for any (in)complete code. In the neural
network-based dependency parsing~\cite{chen-manning-2014-fast} in NLP,
by projecting the words into an embedding space, researchers
successfully learn the semantic relationships that model the
dependencies between them. We can follow suit to learn the
control-flow and program dependencies between the statements in
programs as well. Moreover, given that the quality of the statement
embeddings determines the accurate prediction of the program
dependence relations, enhancing them by incorporating context from
both within and across the statements is crucial. Intra-statement
contextualization can help relay the local information within
individual statements globally. Such information can help
distinguishably identify the declaration and reference of the same
variable across statements. In contrast, inter-statement
contextualization helps learn the dependencies between the statements
in the context of the surrounding statements. Finally, we aim to model
the CFG/PDG construction as a pairwise dependence decoding task,
wherein the combination of all the predicted edges in the statement
pairs can be formalized as CFG or PDG.

\begin{figure*}[t]
\begin{center}
    \includegraphics[width=0.9\textwidth]{model-abstract.jpg}
    \caption{Preliminary Design for \tool Model Infrastructure for Dependency Analysis for Partial Code}
    \label{fig:model}
    \vspace{-26pt}
\end{center}
\end{figure*}

%In Figure~\ref{fig:model}, we present a preliminary design of the
%general architecture of \tool model.

%Given that attention is the driver behind the now ubiquitous
%Transformers’~\cite{Vaswani-2017} success in efficiently learning
%representations for different entities in different contexts, we plan
%to make it the foundation of the context learning components in our
%model as well. Each is intended to learn different aspects of
%contextualization.

%\begin{figure*}
%    \centering
%  \subfloat[Intra-Statement Context Learning\label{fig:input_repr_a}]{%
%       \includegraphics[width=0.45\linewidth]{figures/input_repr_tok.jpg}}
%  \hspace{0.25cm}
%  \subfloat[Inter-Statement Context Learning\label{fig:input_repr_b}]{%
%        \includegraphics[width=0.45\linewidth]{figures/input_repr_stmt.jpg}}
%    \hfill
%  \caption{Input representations: (a) tokens in a statement are the sums of token embeddings, and (token) position embeddings; (b) statements in a snippet are the sums of statement embeddings, statement-type embeddings, and (statement) position~embeddings.}
%  \label{fig:input_repr}
%\end{figure*}


%\subsection{\bf Model Architecture}
%\label{sec:ppda_arch}

%As explained in Section~\ref{sec:overview}, better contextualization
%is the main idea behind \tool's design. Given that attention is the
%component in the ubiquitous Transformers'~\cite{Vaswani-2017} success
%in efficiently learning representations for different entities in
%different contexts, we chose to make it the foundation of our
%model. In brief, we realize \tool via a hierarchical, self-attention
%network (SAN)-based model architecture, where each sub-network is
%intended to capture different aspects of contextualization. Its
%details are as follows:

%\vspace{2pt}
%\subsubsection{{\bf Intra-Statement Context Learning (IntraS-CL)}}
%\label{sec:ppda_arch_intra}
%The syntactic and semantic knowledge of the code tokens within a
%single statement must be made available globally to other statements
%in a program to learn the inter-statement program dependencies
%effectively. We enable this via a \textit{1-Layer} (i.e., $N_X$$=$$1$)
%Self Attention Network (1L-SAN). The self-attention layer in an 1L-SAN
%inputs $x_1, x_2, ..., x_n \in \mathbb{R}^d$, performs self-attention
%once by projecting the inputs from all attention heads $\in
%\mathbb{R}^{d_h}$ into the head dimension space $d_h$ via linear
%transformations, and generate outputs $y_1, y_2, ..., y_n \in
%\mathbb{R}^d$ which are linear combinations of the concatenated
%attention head values. We use one attention head
%(i.e., $h$$=$$1$) for the self-attention layer in 1L-SAN, and the size
%of the input representations, i.e., $d$ is set to 512. Our experiments
%revealed only a marginal performance gain by expanding the 1L-SAN to a
%2L-SAN, which also came with high computational
%overhead. Besides, increasing the number of attention heads did not
%help either performance or interpretability. A more detailed analysis
%on hyperparameters and subsequent
%trade-offs is left to future~work.

%\vspace{1pt} \underline{Input Representations}. For a program
%comprising $N$ statements $s_1, s_2, ..., s_N$, \tool takes as input a
%concatenation of $N$ sequences of $M$ tokens each, $\langle
%t_1^{(1)}$, $t_2^{(1)}$.., $t_M^{(1)} \rangle$, ..., $\langle
%t_1^{(N)}$, $t_2^{(N)}$.., $t_M^{(N)} \rangle$. Next, each token
%sequence $\langle t_1^{(i)}$, $t_2^{(i)}$.., $t_M^{(i)} \rangle$ is
%input to the 1L-SAN for intra-statement contextualization. Previous
%works~\cite{radford2019language, liu2019roberta} have demonstrated
%the~advantages of a byte-level Byte-Pair Encoding (BPE)-scheme for
%tokenization. We follow suit to train a byte-level BPE tokenizer for
%converting a given statement into a sequence of tokens. Here, $M$ is
%the maximum number of tokens allowed in a statement.  For statements
%with token sequences having ${<}M$ tokens, a special \textit{[PAD]}
%token is appended. In contrast, token sequences having ${>}M$ tokens
%are truncated to $M$ tokens.

%\vspace{2pt}
%\subsubsection{\bf Inter-Statement Context Learning (InterS-CL)}
%The knowledge of surrounding statements in the context of a given
%statement helps \tool model the dependencies between them better. We
%enable this via a multi-layer bidirectional Transformer encoder based
%on the work by Vaswani {\em et
%  al.}~\cite{Vaswani-2017}. Owing to its common usage, we will omit
%exhaustive background details on Transformers' model architecture and
%will refer the readers to ~\cite{Vaswani-2017}. As shown in
%Fig.~\ref{fig:model}, we denote the number of layers in the
%Transformer encoder by $N_Y$, which we set to 6. We employ 4 attention
%heads, i.e., $h$$=$$4$ to increase parallelization (since
%$d_h$$=$$\frac{d}{h}$, i.e., $d_h$$=$$128$) and learn different
%aspects of the syntactic and semantic structure in the statements,
%while still being interpretable. We also set the feed-forward module
%size to be 4 times that of the size of the input representations $d$,
%i.e., 2048. Overall, Transformer in InterS-CL phase inputs
%local context-aware statement representations $u_i \in \mathbb{R}^d$
%for all statements $s_i$ in a given program, and outputs statement
%representations $v_i \in \mathbb{R}^d$ that are both local and global
%context-aware.

%\vspace{1pt} \underline{Statement Input Representations}. For all the
%statements $s_i$, statement embeddings (i.e., $u_i \in
%\mathbb{R}^d$) are obtained from the 1L-SAN in the IntraS-CL phase.
%$N$ is the maximum number of statements in a method in the dataset. If
%a given program has less than $N$ statements,
%zero vectors ($\in \mathbb{R}^d$) are padded to the inputs.

\vspace{1pt}
\subsubsection{\bf Pairwise Dependence Decoding}
%Pairs of latent feature vectors $\langle v_i, v_j \rangle$ (1$\leq i, j\leq$ N) are taken from the sequence of intra and inter-statement contextualized statement representations corresponding to all statements in a program, to detect the presence of CFG/PDG edges between them. A combination of all such CFG/PDG edges extracted via the arc-factored approach is realized as the CFG/PDG for the given program. We leverage 2-layered multi-layer perceptron networks (one each for detecting CFG and PDG edges, i.e., MLP\textsubscript{CFG} and MLP\textsubscript{PDG} respectively) for the pairwise dependence decoding phase, which are scored as follows:

From the sequence of contextualized statement representations $v_i \in
\mathbb{R}^d$ corresponding to all the statements in a program passed
on by the Transformer encoder in the InterS-CL phase, pairs such
as $\langle v_i, v_j \rangle$ (1$\leq i, j\leq$ N) are taken to detect
the presence of CFG/PDG edges between two statements $s_i$ and
$s_j$. We leverage 2-layered multi-layer perceptron networks (each
for detecting the CFG and PDG edges, i.e., MLP\textsubscript{CFG} and
MLP\textsubscript{PDG}, respectively) in the pairwise dependence
decoding phase, which are scored as follows:
\begin{equation}
\centering
    score\textsubscript{rel}(i, j) = MLP\textsubscript{rel}(v_i \circ v_j \circ (v_i * v_j) \circ |v_i - v_j|)
\end{equation}
where $\circ$, $*$ and $|.|$ correspond to concatenation, element-wise
product, and absolute element-wise difference operations respectively;
and \textit{rel} represents either the control-flow or program
dependence relations. Attaining a $score\textsubscript{rel}(i, j) >
0.5$ represents the detection of the corresponding CFG/PDG edge from
statement $s_i$ to statement $s_j$. The combination of all the CFG/PDG
edges extracted via such an arc-factored approach is realized as the
CFG/PDG for the given program.

\subsubsection{\bf Training Process}

Training \tool requires the knowledge of ground-truth CFG and PDG edges between statements in a program. Thus, it can only be trained on complete programs (at a minimum, which are at a method-level granularity) so as to be able to leverage program analysis tools to extract them. The CFG and PDG edge information can then be utilized to compute the training objective loss (i.e., $\mathcal{L}$) for our model as follows: $\mathcal{L} = \mathcal{L}\textsubscript{CFG} + \mathcal{L}\textsubscript{PDG}$
%\begin{equation}
%    \centering
%    $$\mathcal{L} = \mathcal{L}\textsubscript{CFG} + \mathcal{L}\textsubscript{PDG}$$
%\end{equation}
where $\mathcal{L}\textsubscript{CFG}$ is the loss for CFG edge-decoding, and $\mathcal{L}\textsubscript{PDG}$ is the loss for PDG edge-decoding. Moreover, $\mathcal{L}\textsubscript{CFG}$ and $\mathcal{L}\textsubscript{PDG}$ are computed as the sums of all binary-cross entropy (BCE) losses corresponding to the CFG and PDG edge predictions between different statements in a program. Note that the inter-statement losses corresponding to the edges from/to the zero-padded statements do not contribute to either $\mathcal{L}\textsubscript{CFG}$ or $\mathcal{L}\textsubscript{PDG}$. The model parameters which are learned to minimize $\mathcal{L}$ include learnable embeddings (token, token position, statement type, and statement position), attention, Tr-FFNN (i.e., feed-forward neural network in Transformer encoder), MLP\textsubscript{CFG}, and MLP\textsubscript{PDG}.

%Overall, \tool has about 39M parameters.

%\subsubsection{\bf Inference for Dependency Discovery}
%\label{sec:inference}
%Despite being trained on only complete code, one can leverage \tool to extract the control-flow and program dependence edges for both complete and partial code.% The following, however, are the important points of consideration:
%\textit{Statement Types}: To extract the syntactic information encoded
%in \textit{statement types}, one would need the program's AST. In
%Java, for example, this can be retrieved even if the code is
%incomplete using tools such as PPA~\cite{dagenais-2008}. However, this
%is not possible for all programming languages. In such cases, \tool
%can be trained without statement types, i.e., by computing the input
%representation for statements in a program as just the sums of the
%statement and their position embeddings. In
%Section~\ref{sec:ablation}, we demonstrate the practicality of such an
%alternative.
%\textit{ If programs with ${\leq}N$ statements}: For both complete and
%partial programs, the number of statements in which is less than the
%\textit{maximum statements} allowed in the model is
%straightforward. In such cases, \tool predicts the CFG/PDG edges from
%one statement to another by contextualizing over all the other
%statements in the program.
%If a program has ${>}N$ statements, for (both complete and partial)
%programs having number of statements greater than that allowed in the
%trained \tool model, we have the following strategies in {\tool}: (a)
%train a model with a higher value of $N$, (b) chunk the program into
%$N$-statement code fragments, predict CFG/PDG edges for each of the
%code fragments independently, and finally, combine the CFG/PDG
%predictions for all the fragments. For example, if a trained model
%allows a maximum of 16 statements, to predict for a program with 46
%statements, one can break it down into code fragments with 16, 16, and
%14 statements, respectively.

%A potential downside to strategy (b), however, is that a statement in
%a fragment will be contextualized only over the other statements in
%that fragment, and the control-flow and program dependencies across
%fragments will not be captured. Increasing $N$ could address this
%issue, albeit with more computational overhead.

%\end{itemize}
