\section{Introduction}
\label{sec:intro}

Integrated Development Environments (IDEs) are indispensable tools in the realm of software development, providing developers with a centralized platform to craft, refine, and deploy their code efficiently. These environments offer a rich array of features and utilities designed to streamline the development process, making them indispensable assets for programmers across various domains. Within the framework of an IDE, the programming assistant stands out as a crucial component, offering invaluable support to developers as they navigate the intricacies of coding. This assistant acts as a guiding hand, offering suggestions, automating repetitive tasks, and providing real-time feedback to enhance productivity and code quality.

At the heart of programming assistant tools lie sophisticated program analysis techniques. By delving deep into the structure and semantics of code, these techniques empower the assistant to offer intelligent insights and recommendations to developers. Whether it's code completion, syntax highlighting, or error detection, program analysis forms the bedrock upon which the programming assistant operates, ensuring that developers can write code with confidence and precision.

Among them, {\em the programming assistant tools that rely on
understanding a program's behaviors at runtime} play a crucial role in
assisting developers with debugging, performance optimization,
security analysis, runtime error detection, etc. By providing insights
into how a program behaves while running, these tools empower
developers to write more reliable, efficient, and secure code. First,
{\em debuggers} are essential programming assistant tools used by
developers to identify and resolve issues in their code. They allow
developers to pause the execution of a program, inspect variables and
memory, and step through code line by line. By understanding the
runtime behavior, debuggers help developers diagnose and fix bugs more
efficiently. Second, {\em dynamic analysis tools}, such as dynamic
taint analysis and dynamic symbolic execution, analyze a program's
behavior during execution to detect security vulnerabilities, such as
buffer overflows, injection attacks, and data leaks. These tools track
the flow of data and control within the program and identify potential
security threats in real-time, helping developers write more secure
code. Finally, {\em runtime error detection tools} monitor a program's
execution for runtime errors, such as null pointer dereferences,
division by zero, and out-of-bounds memory accesses. By detecting
these errors as they occur during runtime, these tools help developers
identify and fix potential issues before they manifest into serious
bugs or crashes.

However, the journey of program analysis or dynamic program's
behaviors is not without its challenges, especially when dealing with
code under editing. The incompleteness and inexecutability of code
fragments present formidable obstacles, making it difficult for
analysis tools to provide accurate assessments and
suggestions. Moreover, the lack of contextual information further
complicates matters, as program analysis may struggle to discern the
broader project landscape and dependencies. Another significant
challenge stems from the absence of a comprehensive global context for
program analysis within the current project. Without access to the
complete picture of the codebase and its interconnections, analysis
tools may struggle to provide nuanced and accurate suggestions,
hindering the effectiveness of the programming assistant.

Using {\em static analysis techniques} to analyze runtime program
behaviors has several disadvantages. Firstly, static analysis operates
solely on the source code without executing it, which means that it
tends to overestimate the dynamic behavior of the program as it
runs. This limitation makes it challenging to identify certain
runtime-specific errors or issues.
%such as concurrency issues, timing-dependent bugs, and interactions
%with external systems or resources.
Additionally, static analysis may produce false positives or false
negatives, leading to inaccuracies in the analysis results.

Conducting runtime program analysis for incomplete and inexecutable
code during the editing phase is fraught with challenges and
limitations. The lack of completeness and executability impedes the
ability of runtime analysis tools to accurately predict program
behavior and provide meaningful insights to developers.  When code is
incomplete, lacking crucial elements such as function definitions,
variable declarations, or control flow structures, conducting runtime
analysis becomes inherently difficult. Without a complete
understanding of the code's structure and functionality, runtime
analysis tools struggle to predict how the program would behave when
executed. Specifically, inexecutable code—code that contains missing
dependencies, or import or variable declarations presents another
obstacle to runtime analysis. Since inexecutable code cannot be
executed due to these errors, runtime analysis tools are unable to
observe the program's behavior firsthand. As a result, they cannot
gather runtime data or insights to inform their analysis.
%
Furthermore, even if runtime analysis tools attempt to analyze
incomplete or inexecutable code, their findings may be unreliable or
inaccurate. The absence of essential program elements can lead to
erroneous conclusions or misleading feedback from the analysis
tools. This can potentially misguide developers in their coding
decisions and hinder the overall development process.

It is desirable to have an approach that can strike a balance and
obtain the best of both worlds. To this effect, we propose a novel
paradigm called {\bf predictive program analysis}, which aims to learn
to analyze program behaviors without actual program execution.

%This is enabled via the learning of semantic and execution behaviors
%of programs obtained from ultra-large-scale, open-source software
%repositories.





\subsection{Research Objectives and Anticipated Results}

\begin{figure}[t]
    \centering
%    \includegraphics[width=0.83\textwidth]{graphs/neuralppa}
%    \includegraphics[width=0.92\textwidth]{figures/infra-design-3.png}
    \includegraphics[width=0.92\textwidth]{overview.png}
    \vspace{-10pt}
    \caption{Predictive Program Analysis: Learning to Analyze Program Behaviors for Incomplete Code}
    \label{fig:arch}
\end{figure}


In this proposal, we seek to advance the state-of-the-art in
traditional program analysis by means of {\tool}, a Predictive Program
Analysis framework, with the goal to support program analysis for
incomplete/inexecutable code. We aim to establish {\em a scientific
  foundation, novel methodologies, frameworks, models, and algorithmic
  solutions for predictive program analysis} with the following focus
areas:



(1) {\bf enabling static program analysis on incomplete code}, and

(2) {\bf supporting predictive analysis for dynamic program behaviors}.

%\noindent Figure~\ref{fig:arch} illustrates the framework for {\tool},
%which will allow the construction of efficient program analysis
%techniques for (partial) code, also based on which downstream vulnerability detection and assessment applications can be built.

The key philosophy that drives our work is that the analysis of
partial code can be learned from the analysis of entire programs in
the wealth of information obtained from ultra-large-scale, open-source
software repositories. To accomplish these tasks, we propose the
following thrusts of research in {\tool} (Figure~\ref{fig:arch}):

\noindent \textbf{Thrust 1. Neural Structural Analysis Infrastructure.} ({\em Section~\ref{sec:thrust1}})
One of the key foundations of {\tool} lies in its neural structural analysis component, which is built upon the well-defined structure and semantics of source code. This component serves two primary functions. Firstly, it draws upon the syntactic structures of comprehensive code samples from large-scale repositories in the training dataset. From this data, it constructs an abstract syntax tree (AST) that best encapsulates the syntactic arrangement of the provided partial code, aiming for the highest likelihood or probability. In contrast, existing program-analysis-based partial parsing methods~\cite{ppa08} rely on heuristics derived from the syntactic rules of programming languages, lacking the capacity to rank or score potential candidates. The second task of this component involves labeling code tokens with the corresponding types of syntactic units, encompassing statement types (\code{if}, \code{for}, etc.), variables, fields, methods, classes, and more. Both tasks can be efficiently executed through our dual-learning-based approaches.

%Source code has a well-defined structure and semantics. Thus, the basic infrastructure in {\tool} is the neural structural analysis component, which primarily has two tasks. First, it learns from the syntactic structures of the complete code in the training dataset collected from large-scale code repositories, to derive the abstract syntax tree (AST) that best represents the syntactic structure of the given partial code, i.e., with the highest likelihood/probability. The existing program-analysis-based partial parsing approaches~\cite{ppa08} rely on the heuristics on the syntactic rules of the programming languages. They do not give us any ranking or scores among the potential candidates. The second task of this component is to tag the code tokens with the types of the syntactic units including the statement types (\code{if}, \code{for}, etc.), variables, fields, methods, classes, etc. Both of the tasks can be performed with our learning-based approaches in a dual-learning manner.
  
\vspace{3pt}
\noindent \textbf{Thrust 2. Neural Semantic Analysis Infrastructure.}
({\em Section~\ref{sec:thrust2}}) The basis components for several
analysis techniques on the semantics of the program could tentatively
include the following (more components will be added as the project
evolves):

1) the identification of the APIs of the external libraries in the
external references in the partial code: this is needed because the
partial code contains the undeclared reference and/or
declaration/reference ambiguity without explicit declaration of the
APIs in the external libraries. The knowledge on the external
libraries enables more precise analysis of the code snippets.

2) the inference of the type information for the entities in the
partial code: due to the ambiguity in the declaration, the types of
the variables and statements are not always obviously
identified. Thus, the type inference is a basic service within
{\tool}.

3) the inference of the program dependencies among the statements in
the partial code: several program analysis techniques are based on the
program dependencies, which are not always obtainable due to the
incompleteness of the given code fragment.

4) Program slicing: program slices are important in both code
understanding and program analysis for code snippets. That allows the
analysis on the statements affecting or to be affected by a specified
variable. All traditional program slicing techniques require the code
to be complete.

\vspace{3pt}
\noindent \textbf{Thrust 3. Predictive Execution.}  ({\em
  Section~\ref{sec:thrust3}}) We advocate for an execution paradigm
called predictive execution. In predictive execution, with a specific
input, the execution is not carried out with the computer performing
the instruction in the program. Instead, a trained machine learning
model predicts the execution steps and as a result, the execution
trace corresponding to the input is derived without actual execution.
The predictive execution paradigm has the potential to improve
efficiency and effectiveness in software testing, fault localization,
and software vulnerability detection.

%Symbolic execution is a means of analyzing a program to determine what
%inputs cause each part of a program to execute. Symbolic execution
%performs executing a program abstractly, so that one abstract
%execution covers multiple possible inputs, which are assumed to have
%symbolic values. We aim to explore the novel area in AI named
%neuro-symbolic learning, which seeks to combine traditional
%rules-based AI approaches with modern deep learning techniques.  We
%will leverage traditional program analysis rules to enhance the
%learning of the characteristics on the execution of the partial code
%fragment.

\vspace{3pt}
\noindent \textbf{Thrust 4. Neural Partial Program Analysis
  Applications.}  ({\em Section~\ref{sec:thrust4}}) Our last thrust of
research is aimed to evaluate our partial program analysis
infrastructure in vulnerability detection for code snippets.
%2) fault localization, and 3) code completion.

%\vspace{3pt}
%\noindent \textbf{Thrust ???. Neural Execution Analysis Infrastructure.}
%({\em Section~\ref{}}) All the dynamic analysis techniques require the
%analysis and understanding of the execution. However, for an
%incomplete code, we first need to design a component that can wrap
%around the given code fragment with the minimum code so that the code
%fragment can be executed. When the code is executed, we also need the
%approaches that represent the executed statements and their relations,
%model the execution and stack traces, and model the code coverages
%for an execution.




Toward this theme, in our preliminary work, we developed DeepPDA
(Section~\ref{sec:deeppda}), a neural network-based partial program
dependence analysis approach that learns to derive the program
dependencies for any code fragments (i.e., both complete and
incomplete). In our preliminary empirical evaluation, we intrinsically
evaluated it on Java and C/C++ programs. We trained DeepPDA on
complete code. For testing, we treated each method individually and
chose a consecutive portion within the method to predict the program
dependencies, and compared them against the actual
dependencies. Overall, DeepPDA predicts CFGs/PDGs in Java with
an F-score of 94.29\%, and in C++ with an F-score of 92.46\%. As
another preliminary work (Section~\ref{sec:statype}), we also
developed an approach to derive the data types of the variables in the
code snippets. We treat the problem as statistical machine translation
from source code with partially qualified names to source code with
full names. Our preliminary evaluation on StackOverflow posts shows
that our technique achieves high accuracy with 97.6\% precision and
96.7\% recall in deriving data types in code snippets.

We also test the usefulness of the PDGs predicted by DeepPDA (i.e.,
PDG*) on the downstream task of method-level vulnerability
detection. We discover that the performance of the vulnerability
detection tool utilizing PDG* is only 1.1\% less than that utilizing
the PDGs generated by a program analysis tool.
%Furthermore, we report 
We also report the detection of 14 real-world vulnerable code snippets
from StackOverflow by a learning-based vulnerability detection
tool that employs the PDGs predicted by DeepPDA for these code snippets.

\begin{table*}[t]
	\vspace{-15pt}
\begin{center}
{\footnotesize{
\begin{tabular}{cc}
\begin{tabular}[t]{|p{0.2in}|p{2.95in}|} 
\hline
\multicolumn{2}{|>{\columncolor[gray]{0}}c|}{\textcolor{white}
{\bf Year 1 Project Milestones \& Deliverables}}\\
\hline 
\hline
\multicolumn{2}{|c|}{\bf T1. Neural Structure Analysis Infrastructure}\\
\hline
{\bf 1.1} & Neural Syntactic Type Tagging\\
{\bf 1.2} & Neural Partial AST Building\\
{\bf 1.3} & Evaluation of the components\\
\hline
\hline
\multicolumn{2}{|c|}{\bf T2. Neural Semantic Analysis Infrastructure}\\ 
\hline
{\bf 2.1} & External-Library Identification\\
\hline
%\hline
%\multicolumn{2}{|c|}{\bf Integrate Code Synthesis into Tools}\\
%\hline
%{\bf 1.5} & \goalOneFour.\\
%\hline
\multicolumn{2}{c}{}
\end{tabular}
&
\begin{tabular}[t]{|p{0.2in}|p{2.95in}|} \hline
\multicolumn{2}{|>{\columncolor[gray]{0}}c|}{\textcolor{white}
{\bf Year 2 Project Milestones \& Deliverables}}\\
\hline 
\hline
\multicolumn{2}{|c|}{\bf T2. Neural Semantic Analysis Infrastructure}\\
\hline
{\bf 2.2} & Neural Type Inference\\
{\bf 2.3} & Neural Dependence Analysis\\
%{\bf 2.3} & Integrate Evaluation Framework into Design Environment\\
%{\bf 2.4} & Evaluate CRL Framework with Existing Models\\
%{\bf 2.3} & \goalTwoThree.\\

\hline
\hline
\multicolumn{2}{|c|}{\bf T3. Neural Execution Analysis}\\ 
\hline
%{\bf 3.1} & Design New Code Representations and Learning Models.\\
{\bf 3.1} & Neural Execution-Trace Modeling\\
%{\bf 2.4} & Advance FL and RT-CI Approaches.\\
%{\bf 2.5} & Advance Regression Testing in CI Approaches.\\
%{\bf 2.5} & Advance APR Approaches with Framework.\\
\hline
%\hline
%\multicolumn{2}{|c|}{\bf Community Involvement: Capacity Building}\\
%\hline
%{\bf 2.4} & \goalTwoFour.\\
%{\bf 2.5} & \goalTwoFive.\\
%{\bf 2.6} & \goalTwoSix.\\
%\hline
\multicolumn{2}{c}{}
\end{tabular}
\end{tabular}\\
\vspace*{-.3cm}
\begin{tabular}{c}\hline
\multicolumn{1}{|>{\centering\columncolor[gray]{0}}p{6.44in}|}{\textcolor{white}
{\bf Year 3 Project Milestones \& Deliverables}}\\
\hline
\end{tabular}\\
\vspace*{-.2cm}
\begin{tabular}{cc}
\begin{tabular}[t]{|p{0.2in}|p{2.95in}|}
\hline
\multicolumn{2}{|c|}{\bf T3. Neural Execution Analysis}\\
\hline
{\bf 3.2} & Neural Stack Trace Modeling\\
{\bf 3.3} & Neural Code Coverage Modeling\\

%{\bf 3.3} & Testing on Models in IDE tools.\\
\hline
%\hline
%\multicolumn{2}{|c|}{\bf \goalTwo}\\ 
%\hline
%{\bf 3.3} & \goalThreeThree.\\
%\hline
\multicolumn{2}{c}{}
\end{tabular}
&
\begin{tabular}[t]{|p{0.2in}|p{2.95in}|}
\hline
\multicolumn{2}{|c|}{\bf T4. Neural Partial Program Analysis Applications}\\
\hline
%{\bf 3.1} & Design New Code Representations\\

{\bf 4.1} & Security Vulnerablity Detection with {\tool}\\
{\bf 4.2} & Fault Localization and Completion with {\tool}\\

\hline
\multicolumn{2}{c}{}
\end{tabular}
\end{tabular}
\vspace{-15pt}
}}
\end{center}
\vspace*{-.3in}
%\caption{Tasks and Milestones. (Rep. = Representation)}
\caption{The 3-year schedule of Thrusts, Tasks, and Milestones of this proposal. (Rep. = Representation)}
%the schedule of Thrusts, Tasks, and Milestones of this proposal.
%\vspace{-10pt}
\label{tab:milestones}
\vspace{-10pt}
\end{table*}
%




%\subsection{Significance of This Proposed Project: NSF Merit Criteria}

%\section{Relevance to Secure and Trustworthy Cyberspace}

%This project will develop novel concepts, representations, algorithms,
%models, and tools to support early software vulnerability
%detection. It is transformative and directly help improve software
%quality with novel program analysis-based software security and
%vulnerability detection tools on code snippets.

\section{Intellectual Merits}

%The results of this project will be transformative and directly help
%improve software quality with novel program analysis-based software
%security tools.

\noindent \underline{{\bf Advance the state-of-the-art knowledge and
    understanding}}. Predictive program analysis infrastructure in Thrusts
1--3 will advance the body of knowledge and theoretical foundations
in the area of {\bf machine learning and AI for code}. Thrust 4 will also help advance the practical tools in software engineering.

\noindent \underline{{\bf Scientific foundation, creative/original
    research}}. (1) to enable the analysis on partial code, (2) to empower
the program analysis techniques on both (in)complete code,
and (3) to enable the applications of program analysis on incomplete
code such as runtime error and vulnerability detection for code snippets, debugging tools for code under editing, etc.

\section{Broader Impacts}

\underline{{\bf (1) Transformative and benefits to society}}. Our
results will be transformative and directly benefit to our society.
They will lead to increasing developers' productivity, software
quality \& reliability.  Our validation involves students and
professionals, promoting teaching, training, and learning of both {\bf
  program analysis} and {\bf machine learning} techniques that
have wide impacts in industry and academic communities.

\noindent\underline{{\bf (2) Foster other related research
    activities}}. Our results will foster {\em research activities in
  related fields of {\bf machine learning} and {\bf software security}},
and the applications in software security and reliability.
%We will produce theoretical concepts and techniques that are novel in
%deep learning, e.g., novel neural networks to model and learn for
%code.
%The applications of our neural program analysis in software
%engineering applications will advance software security and
%reliability.

%The collected {\bf large scale bug\&fix corpus} will be useful for
%software quality and reliability research.
%Innovations in CRL could be used to {\bf advance other SE tasks}. We
%will also develop {\bf novel DL-based bug detect-fix} approaches.


\noindent\underline{{\bf (3) Education, dissemination, and broader participation}} (Section~\ref{edu}). The
research will enhance the infrastructure for teaching/research via
tools and data sets for use by students and practitioners, and for
enhancement by researchers. We will provide related learning
modules for educators as well. It will include outreach activities for
undergraduate students, underrepresented groups, minorities, and women
in science.
%contribute novel
%teaching modules to our curriculum.
%Details will be presented in Section~\ref{edu}


\iffalse
\begin{itemize}
	\vspace{-5pt}
\itemsep-0.2em 
  \item {\bf Transformative and benefits to society}. Our 
    results will be transformative and directly benefit to our
    society. 
    They will lead to increasing developers' productivity
    and software quality \& reliability. 
    Our validation involves students and
    professionals, promoting teaching, training, and learning of bug detecting and fixing techniques that have wide impacts 
    in industry and academic communities.

%report

  \item {\bf Foster other research activities}. Our results will
    foster research activities in related fields of deep learning and
    software quality. 
    This project will produce theoretical
    concepts and techniques that are novel even in deep learning, e.g.,
    novel neural networks for modeling and learning code. 
    The collected large scale bug fixing corpus will be useful for software quality and reliability research in general.
    %e.g.,   code transformation. 
%    This project will also advance
%    the state-of-the-art research in large-scale program analysis with
%    deep neural network models.

%The representation for software security vulnerabili will be useful in
%research on software security, malware detection, vulnerability
%reports, and automatic security patching.


  \item {\bf Education, dissemination, and broader participation}.
  The research will enhance the infrastructure for teaching and
  research by providing tools and data sets for use by students and
  practitioners, and for enhancement by other researchers. We will
  provide related learning modules for educators as well. It will
  contribute novel teaching modules to our curriculum. Details will be
  presented in Section~\ref{edu}

%Details are in Section 4.

\end{itemize}
\fi
