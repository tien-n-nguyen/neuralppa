Integrated Development Environments (IDEs) are indispensable tools in the realm of software development, providing developers with a centralized platform to craft, refine, and deploy their code efficiently. These environments offer a rich array of features and utilities designed to streamline the development process, making them indispensable assets for programmers across various domains. Within the framework of an IDE, the programming assistant stands out as a crucial component, offering invaluable support to developers as they navigate the intricacies of coding. This assistant acts as a guiding hand, offering suggestions, automating repetitive tasks, and providing real-time feedback to enhance productivity and code quality.

At the heart of programming assistant tools lie sophisticated program analysis techniques. By delving deep into the structure and semantics of code, these techniques empower the assistant to offer intelligent insights and recommendations to developers. Whether it's code completion, syntax highlighting, or error detection, program analysis forms the bedrock upon which the programming assistant operates, ensuring that developers can write code with confidence and precision.

Among them, {\em the programming assistant tools that rely on
understanding a program's behaviors at runtime} play a crucial role in
assisting developers with debugging, performance optimization,
security analysis, runtime error detection, etc. By providing insights
into how a program behaves while running, these tools empower
developers to write more reliable, efficient, and secure code. First,
{\em debuggers} are essential programming assistant tools used by
developers to identify and resolve issues in their code. They allow
developers to pause the execution of a program, inspect variables and
memory, and step through code line by line. By understanding the
runtime behavior, debuggers help developers diagnose and fix bugs more
efficiently. Second, {\em dynamic analysis tools}, such as dynamic
taint analysis and dynamic symbolic execution, analyze a program's
behavior during execution to detect security vulnerabilities, such as
buffer overflows, injection attacks, and data leaks. These tools track
the flow of data and control within the program and identify potential
security threats in real-time, helping developers write more secure
code. Finally, {\em runtime error detection tools} monitor a program's
execution for runtime errors, such as null pointer dereferences,
division by zero, and out-of-bounds memory accesses. By detecting
these errors as they occur during runtime, these tools help developers
identify and fix potential issues before they manifest into serious
bugs or crashes.

However, the journey of program analysis or dynamic program's
behaviors is not without its challenges, especially when dealing with
code under editing. The incompleteness and inexecutability of code
fragments present formidable obstacles, making it difficult for
analysis tools to provide accurate assessments and
suggestions. Moreover, the lack of contextual information further
complicates matters, as program analysis may struggle to discern the
broader project landscape and dependencies. Another significant
challenge stems from the absence of a comprehensive global context for
program analysis within the current project. Without access to the
complete picture of the codebase and its interconnections, analysis
tools may struggle to provide nuanced and accurate suggestions,
hindering the effectiveness of the programming assistant.

Using {\em static analysis techniques} to analyze runtime program
behaviors has several disadvantages. Firstly, static analysis operates
solely on the source code without executing it, which means that it
tends to overestimate the dynamic behavior of the program as it
runs. This limitation makes it challenging to identify certain
runtime-specific errors or issues.
%such as concurrency issues, timing-dependent bugs, and interactions
%with external systems or resources.
Additionally, static analysis may produce false positives or false
negatives, leading to inaccuracies in the analysis results.

Conducting runtime program analysis for incomplete and inexecutable
code during the editing phase is fraught with challenges and
limitations. The lack of completeness and executability impedes the
ability of runtime analysis tools to accurately predict program
behavior and provide meaningful insights to developers.  When code is
incomplete, lacking crucial elements such as function definitions,
variable declarations, or control flow structures, conducting runtime
analysis becomes inherently difficult. Without a complete
understanding of the code's structure and functionality, runtime
analysis tools struggle to predict how the program would behave when
executed. Specifically, inexecutable code—code that contains missing
dependencies, or import or variable declarations presents another
obstacle to runtime analysis. Since inexecutable code cannot be
executed due to these errors, runtime analysis tools are unable to
observe the program's behavior firsthand. As a result, they cannot
gather runtime data or insights to inform their analysis.
%
Furthermore, even if runtime analysis tools attempt to analyze
incomplete or inexecutable code, their findings may be unreliable or
inaccurate. The absence of essential program elements can lead to
erroneous conclusions or misleading feedback from the analysis
tools. This can potentially misguide developers in their coding
decisions and hinder the overall development process.

It is desirable to have an approach that can strike a balance and
obtain the best of both worlds. To this effect, we propose a novel
paradigm called {\bf predictive program analysis}, which aims to learn
to analyze program behaviors without actual program execution.

%This is enabled via the learning of semantic and execution behaviors
%of programs obtained from ultra-large-scale, open-source software
%repositories.



