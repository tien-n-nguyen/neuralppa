\documentclass[11pt]{article}

\usepackage{graphicx,times}
\usepackage{wrapfig}
\usepackage{amsmath,epsfig}
\usepackage{setspace,array}
\usepackage{cite}
\usepackage{xspace}
\usepackage{enumitem}

\usepackage{sectsty}
\sectionfont{\large}
\subsectionfont{\normalsize}
\subsubsectionfont{\normalsize}

\usepackage[compact]{titlesec}

\renewcommand{\theequation}{\thesection.\arabic{equation}}
\renewcommand{\baselinestretch}{1.0}

\newtheorem{Definition}{Definition}
\newtheorem{Claim}{Claim}
\newtheorem{Lemma}{Lemma}
\newtheorem{Theorem}{Theorem}
\newtheorem{Property}{Property}

\newcommand{\revise} {\bf}

\newcommand{\code}[1]{{\small\texttt{#1}}}
\newcommand{\op}{\tau}
\newcommand{\refac}{\rho}
\newcommand{\edit}{\sigma}
\newcommand{\T}{\theta}
\newcommand{\comp}{;}
\newcommand{\pre}{\prec_P}
\newcommand{\meth}{KSISA}


\newcommand{\MyParagraph}[1]{\textbf{#1}{ }}

\usepackage{vmargin}
\setpapersize{USletter}
\setmarginsrb{1.0in}{1in}{1.0in}{1in}%
           {0pt}{0mm}{0pt}{10mm}
\newcommand{\remove}[1]{}

\usepackage{tweaklist}
\renewcommand{\enumhook}{\setlength{\topsep}{0pt}%
  \setlength{\itemsep}{0pt}}
\renewcommand{\itemhook}{\setlength{\topsep}{0pt}%
  \setlength{\itemsep}{0pt}}
\renewcommand{\descripthook}{\setlength{\topsep}{0pt}%
  \setlength{\itemsep}{0pt}}

\newcommand{\tool}{\textsc{PredPA}\xspace}

\thispagestyle{empty}

\begin{document}

% Collaborative Research: SHF: Small: 



\begin{center}
  {\bf SHF: Small: Predictive Program Analysis: Learning to Analyze Program Behaviors}
\end{center}
\vspace{-.1in}


%\noindent {\bf Overview.}

\section{Overview}

The importance of programming assistant tools that understand a
program's behaviors at runtime cannot be overstated. They aid
developers in debugging, performance optimization, security analysis,
and runtime error detection, etc.
%enabling the creation of reliable, efficient, and secure code.
However, analyzing dynamic program behaviors faces challenges,
particularly with code under editing. Incomplete and inexecutable code
fragments hinder accurate assessment and suggestion provision by
analysis tools. The absence of contextual information and a
comprehensive global context further complicates analysis, hindering
the effectiveness of programming assistants. Static analysis, while
useful, has limitations. Static analysis tends to overestimate program
behavior and may produce false positives or negatives, reducing
accuracy. Conducting runtime analysis for incomplete or inexecutable
code during editing is challenging. Incomplete code lacks essential
elements, making analysis difficult. Inexecutable code, due to errors
like missing dependencies, prevents direct observation of program
behavior. Even if analyzed, results may be unreliable, potentially
misleading developers and impeding development progress.


In this proposal, we seek to advance the state-of-the-art in
traditional program analysis by means of {\tool}, a Predictive Program
Analysis framework, with the goal to support program analysis for
incomplete/inexecutable code. We aim to establish {\em a scientific
  foundation, novel methodologies, frameworks, models, and algorithmic
  solutions for predictive program analysis} with the following focus
areas:


(1) {\bf enabling static program analysis on incomplete code}, and

(2) {\bf supporting predictive analysis for dynamic program behaviors}.

The key philosophy for our work is that the analysis of
partial code can be learned from the analysis of entire programs in
the wealth of information obtained from large-scale, open-source
software repositories.


\noindent {\bf Keywords:} Software Engineering, ML and AI; Program Analysis, Software; Systems;

\section{Intellectual Merit}

Our first thrust advocates for a novel paradigm, called {\bf
  predictive program analysis}, which operates on the principles of
{\em Approximation-Refinement} for Analysis. In the Approximation
phase, a large language model (LLM) acts as a machine learning (ML)
agent to fill in missing information within incomplete code. The
Refinement phase employs a program analysis (PA)-based agent to verify
the completed code output by the LLM. We propose a tandem solution
that combines LLMs and PA agents to leverage the strengths of both
methodologies. Our second thrust of research advocates for an
execution paradigm called {\bf predictive execution}. In predictive
execution, with a specific input, the execution is not carried out
with the computer performing the instruction in the program. Instead,
a trained machine learning model predicts the execution steps and as a
result, the execution trace corresponding to the input is derived
without actual execution. Our last thrust of research is aimed to
demonstrate the usefulness of our solution in different programming
assistant applications: 1) vulnerability detection, 2) dynamic
slicing, 3) runtime error prediction, etc.

\section{Broader Impacts}

%Our predictive program analysis framework is also beneficial in other
%scenarios in addition to programming assistants.

Our predictive program analysis and predictive execution also offer a
solution where the actual execution is impossible, e.g., i) where the complete source code is
not available, or ii) where the analysis or approximation on
program behavior is needed/desired without actual execution. (1)
{\tool} will be transformative and directly benefit to our
  society, leading to increasing software quality.  Our validation
involves students and professionals, promoting teaching and learning
of software qualities that have wide impacts in industry and academic
communities. (2) Our results will {\em foster research activities} in
related fields, e.g., AI for code. (3) It will
enhance the tools for teaching and research by providing tools and
data sets for use by students and practitioners, and for enhancement
by other researchers with related learning modules.

\end{document} 
