\section{Two-Prompt \tool}\label{sec:plan->predict}
In this section, we present our design of the two-phase \textit{Plan} $\xrightarrow[]{}$\textit{Predict}.

\subsection{Plan Formulation}
\subsubsection{Basic Structure}
This section presents the first phase, i.e., \textit{Plan Formulation}, in \textit{Plan} $\xrightarrow{}$\textit{Predict}. In this phase, we design a prompt comprising three primary segments: (1) instructions to the LLM to devise a PA-based plan, (2) exemplar(s) comprising code snippet and corresponding plans, (3) test code snippet. The output from the LLM comprises of the generated plan for the test code snippet.

%The Plan Formation Module is the first segment of {\tool} and is responsible for the prediction of the step by step reasoning process that will later be essential to predict the code coverage of the given test code. Functioning as a singular prompt, the module comprises multiple segments, each holding equal significance and contributing indispensably to the Plan's prediction. The comprehensive breakdown of this module as follows :

%\subsection{\bf Basic Architecture}
%\label{sec:pfm_architecture}

\subsubsection*{Instructions} The first segment consists of the problem statement explained in natural language, which instructs the LLM to predict/build its own plan to be pursued for predicting the code coverage of the given
test code (see example in Section~\ref{sec:pfm_example-2}).

\subsubsection*{Exemplar(s)}
\label{subsec:examplar}
%Also known as The Reasoning Prompt, this segment of {\tool} comprises 3 sections, as depicted in Figure 4. The first section consists of the problem statement presented in natural language, which instructs the LLM to predict the plan to be pursued for the prediction of code coverage in the given test code. {\color{red}DOUBT : include a figure showing the prompt for the plan prediction!!!}
The second segment of the prompt incorporates exemplar(s) to enable the few-shot setting, each comprising a code snippet (distinct from the given test code) and a manually-crafted plan. Each step within that plan provides a succinct elucidation of the reasons behind the (non)-execution of the associated statement(s). The procedure for crafting the plan is same as in Section~\ref{sec:pfm_procedure-1}. The LLM uses this as a guide to understand the execution flow.

%Using this as a guide, the LLM predicts a plan for understanding the execution-flow in the test code snippet.

\subsubsection*{Given Test Code Snippet}
\label{subsec:testcode}
The last segment is the test code for which LLM has to create the code execution plan. 

\subsubsection{Illustrating Example}\label{sec:pfm_example-2}
\begin{figure}[t]
	\centering
	\lstset{
		numbers=left,
		numberstyle= \tiny,
		keywordstyle= \color{blue!70},
		commentstyle= \color{red!50!green!50!blue!50},
		frame=shadowbox,
		rulesepcolor= \color{red!20!green!20!blue!20} ,
		xleftmargin=1.5em,xrightmargin=1em, aboveskip=1em,
		framexleftmargin=1.5em,
                numbersep= 5pt,
		language=Python,
    basicstyle=\tiny\ttfamily,
    numberstyle=\tiny\ttfamily,
    emphstyle=\bfseries,
                moredelim=**[is][\color{red}]{@}{@},
		escapeinside= {(*@}{@*)}
	}
\hspace{8pt}
% \begin{minipage}{.45\textwidth}        
% \begin{lstlisting}[caption = ]
% (*@{\color{red}{PLAN FORMATION MODULE}@*)
% For the given code snippet, give the plan to predict the code coverage. The code coverage indicates whether a statement has been executed (*@\color{black}{or not.}@*) 

% You need to develop a plan (*@\color{black}{for}@*) step by step execution of the code snippet. 
% Below (*@\color{black}{is}@*) an illustration of the process you need to follow to predict the code coverage of the given code snippet. 

% (*@\color{orange}{Example -}@*) 
% (*@\color{orange}{Given Code snippet :}@*)
% number = 15
% if number % 2 == 0:
%     print("{} is an even number.".format(number))
% else:
%     print("{} is an odd number.".format(number))

% (*@\color{orange}{PLAN :}@*)
% Step 1: Variable Initialization: Initialize the number variable with a specific value. In this case, it(*@\color{black}{'}@*)s (*@\color{black}{set}@*) to 15. Statements "number = 15" will be executed. "if number % 2 == 0:" will be executed
% Step 2: Operation : Use the % (modulo) operator to check (*@\color{black}{if}@*) the number (*@\color{black}{is}@*) divisible by 2. If the result (*@\color{black}{is}@*) 0, the number (*@\color{black}{is}@*) even. If (*@\color{black}{not}@*), it(*@\color{black}{'}@*)s odd.
% Step 3: Branching (*@\color{black}{if-else}@*) Block: Enter the if block (*@\color{black}{if}@*) the number (*@\color{black}{is}@*) even. Since 15%2 results (*@\color{black}{in}@*) 1, statement "print("{} (*@\color{black}{is}@*) an even number.".format(number))" will (*@\color{black}{not}@*) be executed but the else block will (*@\color{black}{if}@*) the number (*@\color{black}{is}@*) odd. Statements "else:" (*@\color{black}{and}@*) "print("{} (*@\color{black}{is}@*) an odd number.".format(number))" will be executed
% Step 4: Output: The required print statement will be executed based on the output of the if-else block

% In a similar fashion, develop a plan of step by step execution of the below code snippet - 
% (*@{\color{red}{<< Test Code... >>}@*)

% \end{lstlisting}
% \vspace{-9pt}
% \subcaption{One-shot Prompt used in Plan Formation}
% \label{fig:example-plan-formation}
% \end{minipage}
\begin{minipage}{.45\textwidth}        
\begin{lstlisting}[caption = ]
(*@{\color{red}{PLAN FORMATION}@*)
For the given code snippet, give the plan to predict the code coverage. The code coverage indicates whether a statement has been executed (*@\color{black}{or not.}@*) 

You need to develop a plan (*@\color{black}{for}@*) step by step execution of the code snippet. 
Below (*@\color{black}{is}@*) an illustration of the process you need to follow to predict the code coverage of the given code snippet. 

(*@\color{red}{<< Exemplar code, as on Lines 16-22 in Figure 5 >>}@*) 

(*@\color{red}{<< Exemplar plan, as on Lines 24-28 in Figure 5 >>}@*) 

In a similar fashion, develop a plan of step by step execution of the below code snippet - 
(*@{\color{red}{<< Test Code... >>}@*)

\end{lstlisting}
\vspace{-9pt}
\subcaption{One-shot prompt used for \textit{Plan Formation}}
\label{fig:example-plan-formation}
\end{minipage}
%\hspace{8pt}
% \begin{minipage}{.45\textwidth}        
% \begin{lstlisting}[caption = ]
% (*@{\color{red}{COVERAGE PREDICTION MODULE}@*)
% For the given code snippet (*@\color{black}{and}@*) plan, give the code coverage that follows the plan. The code coverage indicates whether a statement has been executed (*@\color{black}{or not}@*). 
% > (*@\color{black}{if}@*) the line (*@\color{black}{is}@*) executed
% ! (*@\color{black}{if}@*) the line (*@\color{black}{is not}@*) executed
% Example output:
% > line1
% ! line2
% > line3
% ...
% > lineN
% You need to give the code with its coverage (*@\color{black}{for}@*) the given plan.
% Below (*@\color{black}{is}@*) an illustration of the process you need to follow to predict the code coverage of the given code snippet (*@\color{black}{and}@*) its plan. 
% (*@\color{orange}{Given Code snippet :}@*) 
% number = 15
% if number % 2 == 0:
%     print("{} is an even number.".format(number))
% else:
%     print("{} is an odd number.".format(number))
% DISCLAIMER: Lines that are not executed are to be denoted with a SINGLE '!' whereas lines that are executed are to be denoted with a single '>'

% (*@\color{orange}{Given PLAN :}@*) 
% Step 1: Variable Initialization: Initialize the number variable with a specific value. In this case, it(*@\color{black}{'}@*)s (*@\color{black}{set}@*) to 15. Statements "number = 15" will be executed. "if number % 2 == 0:" will be executed
% Step 2: Operation : Use the % (modulo) operator to check (*@\color{black}{if}@*) the number (*@\color{black}{is}@*) divisible by 2. If the result (*@\color{black}{is}@*) 0, the number (*@\color{black}{is}@*) even. If (*@\color{black}{not}@*), it(*@\color{black}{'}@*)s odd.
% Step 3: Branching (*@\color{black}{if-else}@*) Block: Enter the if block (*@\color{black}{if}@*) the number (*@\color{black}{is}@*) even. Since 15%2 results (*@\color{black}{in}@*) 1, statement "print("{} (*@\color{black}{is}@*) an even number.".format(number))" will (*@\color{black}{not}@*) be executed but the else block will (*@\color{black}{if}@*) the number (*@\color{black}{is}@*) odd. Statements "else:" (*@\color{black}{and}@*) "print("{} (*@\color{black}{is}@*) an odd number.".format(number))" will be executed
% Step 4: Output: The required print statement will be executed based on the output of the if-else block

% (*@\color{orange}{So the final code coverage will be :}@*) 
% > number = 15
% > if number % 2 == 0:
% !     print("{} is an even number.".format(number))
% > else:
% >     print("{} is an odd number.".format(number))

% In a similar fashion, Give the code coverage of the below code snippet based on the given plan - 

% (*@{\color{red}{<< Test Code... >>}@*)
% (*@{\color{red}{Plan generated in Plan Formation Phase}@*)

% \end{lstlisting}
% \vspace{-9pt}
% \subcaption{One-shot Prompt used in Coverage Prediction}
% \label{fig:example-plan-cc-prediction}
% \end{minipage}
\begin{minipage}{.45\textwidth}        
\begin{lstlisting}[caption = ]
(*@{\color{red}{COVERAGE PREDICTION}@*)
For the given code snippet (*@\color{black}{and}@*) plan, give the code coverage that follows the plan. The code coverage indicates whether a statement has been executed (*@\color{black}{or not}@*). 
> (*@\color{black}{if}@*) the line (*@\color{black}{is}@*) executed
! (*@\color{black}{if}@*) the line (*@\color{black}{is not}@*) executed
Example output:
> line1
! line2
> line3
...
> lineN
You need to give the code with its coverage (*@\color{black}{for}@*) the given plan.
Below (*@\color{black}{is}@*) an illustration of the process you need to follow to predict the code coverage of the given code snippet (*@\color{black}{and}@*) its plan. 
(*@\color{red}{<< Exemplar code, as on Lines 16-22 in Figure 5 >>}@*) 

DISCLAIMER: Lines that are not executed are to be denoted with a SINGLE '!' whereas lines that are executed are to be denoted with a single '>'

(*@\color{red}{<< Exemplar plan, as on Lines 24-28 in Figure 5 >>}@*) 

(*@\color{red}{<< Exemplar code coverage, as on Lines 30-36 in Figure 5 >>}@*) 

In a similar fashion, give the code coverage of the below code snippet based on the given plan - 

(*@{\color{red}{<< Test code... >>}@*)
(*@{\color{dkgreen}{<< Test plan (generated by the Plan Formulation phase)... >>}@*)

\end{lstlisting}
\vspace{-9pt}
\subcaption{One-shot prompt used for \textit{Code Coverage Prediction}}
\label{fig:example-plan-cc-prediction}
\end{minipage}
\vspace{-9pt}
\caption{Prompts with a single exemplar (i.e., one-shot) for \textit{Plan Formulation} (top) and \textit{Code Coverage Prediction} (bottom) phases in two-prompt {\tool}.}
\label{fig:example}
\end{figure}

An example prompt for the \textit{Plan Formulation} phase is provided with a single exemplar in Figure~\ref{fig:example}(a). 
Note that the code snippet and the corresponding manually-crafted plan in the exemplar are the same as in Section~\ref{sec:one-prompt-ex}. 

Similar to as in one-prompt {\tool}, the prompt in this phase guides the LLM to draw parallels from the execution rationale in the exemplar plan by learning to reason about the exemplar code, and subsequently, generate a similar plan for the test code. Such a break down of the test code into comprehensible steps enables the LLM to reason about the code execution and make predictions regarding which statements will be executed or skipped. Thus, such a systematic approach encapsulated within the \textit{Plan Formulation} phase instills a structured method for later predicting the code coverage. By design, this approach provides a guide to understand the code coverage prediction, improving the LLM's interpretability.
%
% Already included in Section 4.4 now.
%
% First, the module directs the initialization of the
% \code{number} variable to a specific value (15), ensuring the
% execution of the corresponding statement. Subsequently, the modulo
% operator is employed to assess the divisibility of \code{number} by
% 2. Based on the result, the plan dictates the branching of the
% execution into either the \code{if} block or the \code{else} block. As
% the execution progresses through the plan, we guide the model to
% predict whether specific statements within the code snippet will be
% executed or not. For example, in the given scenario, the plan predicts
% that the print statement within the \code{if} block will not be
% executed due to the specific value of \code{number}. However,
% statements within the \code{else} block will be executed.
%
% This systematic approach, encapsulated in the plan formation phase,
% instills a structured method for later predicting code coverage. The
% module's process involves breaking down the code snippet into
% comprehensible steps, enabling the LLM to reason about the execution
% steps and make predictions regarding which statements will be executed
% or skipped. This approach is not only expected to enhance the LLM's
% interpretability but also provides a valuable guide for predicting the
% code coverage in the later phase.



\subsection{Code Coverage Prediction}

\subsubsection{Basic Structure}
This section presents the second phase, i.e., \textit{Code Coverage Prediction}, in \textit{Plan} $\xrightarrow{}$\textit{Predict}. In this phase, we design a prompt comprising four primary segments: (1) instructions to the LLM to compute the code coverage for the code snippet, (2) exemplar(s) comprising code snippet, corresponding manually-crafted exemplary plan and code coverage, (3) test code snippet, (4) generated plan from \textit{Plan Formulation} phase. The output from the LLM comprises the predicted code coverage for the given test code.

\subsubsection*{Instructions} This segment in the second prompt requests the LLM to adhere to the {\em generated plan} to predict the code coverage for the given test code, using the exemplar(s) to guide this process. 

\subsubsection*{Exemplar(s)} In the two-prompt setting, the exemplar(s) are exactly the same as in one-prompt {\tool} (see Section~\ref{sec:one-prompt-structure}). The inclusion of this serves the purpose of guiding the LLM towards achieving a code coverage prediction in the prescribed format.

\subsubsection*{Given Test Code Snippet} This segment is same in both \textit{Plan Formulation} (Section~\ref{subsec:testcode}) and \textit{Code Coverage Prediction} phases.

\subsubsection*{Generated Plan} In the two-prompt setting, the plan generated from the \textit{Plan Formulation} phase is incorporated into the prompt in this phase to guide the code coverage prediction. Note that this differs from the one-prompt {\tool} where one LLM is tasked with generating both the plan and the code coverage sequentially.

\subsubsection{Illustrating Example}\label{cpm_example-2}
An example prompt for the \textit{Code Coverage Prediction} phase is provided with a single exemplar in Figure~\ref{fig:example-plan-cc-prediction}. Note that the exemplar's structure is same as that in one-prompt {\tool}, including the same code snippet and manually-crafted plan in the exemplar as in the \textit{Plan Formulation} phase (Section~\ref{sec:plan->predict}), as well as the corresponding code coverage. Next, along with the test code, we include the test plan generated in the \textit{Plan Formulation} phase. By design, this prompt facilitates the 
prediction of code coverage for the test code from the generated plan, inherently learning such associations between alike components in the exemplar(s).

% The example presented in Figure~\ref{fig:example-plan-cc-prediction} outlines the process of predicting code coverage using the code coverage prediction within the context of a given code snippet and its plan. The objective is to determine the executed or non-executed status of each line, denoted by '>' for executed lines and '!' for non-executed lines. The detailed instruction is illustrated in lines 2--12.

% The example code snippet (lines 14-18 of
% Figure~\ref{fig:example-plan-cc-prediction}(b)) initializes a
% variable, 'number,' to 15, followed by a conditional statement
% checking if the number is even or odd. The provided plan (lines 21-25)
% details the execution steps, such as variable initialization, modulo
% operation, branching in the \code{if-else} block, and the
% corresponding output. Following the outlined steps, the final
% predicted code coverage is presented (lines 26--31), highlighting the
% lines that are expected to be executed or skipped based on the given
% plan.
% %This approach facilitates a systematic assessment of code
% %coverage, aiding in the evaluation and understanding of the test
% %code's execution flow.
% The subsequent request prompts a similar analysis for a different test
% code (lines 33 - 35), encouraging a comprehensive exploration of code
% coverage prediction within the generated plan.

