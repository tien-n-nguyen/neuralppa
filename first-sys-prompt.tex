\begin{figure}[t]
	\centering
	\lstset{
		numbers=left,
		numberstyle= \tiny,
		keywordstyle= \color{blue!70},
		commentstyle= \color{red!50!green!50!blue!50},
		frame=shadowbox,
		rulesepcolor= \color{red!20!green!20!blue!20} ,
		xleftmargin=2.7em,xrightmargin=2.7em, aboveskip=1em,
		framexleftmargin=1.5em,
                numbersep= 5pt,
		language=Python,
    basicstyle=\tiny\ttfamily,
    numberstyle=\tiny\ttfamily,
    emphstyle=\bfseries,
                moredelim=**[is][\color{red}]{@}{@},
		escapeinside= {(*@}{@*)}
	}
\begin{minipage}{.5\textwidth}        
\begin{lstlisting}[caption = ]
(*@{\color{red}{First One-shot Prompt Setting@*)
(*@{\color{red}{System Prompt@*)
Task: Analyze the given partial Java code snippet and compiler errors. Generate the necessary code to complete the snippet by adding to the header only. Ensure the snippet becomes a compilable Java unit without modifying the original body. Provide type information for any new elements introduced in the code.
Input:
Partial Code Snippet: (*@{\color{blue}<code> @*)
Compiler Errors: (*@{\color{blue}<error> @*)
Output:
1. Code Approximation: Provide the enhanced code snippet.
2. Type Information: Detail the type information for each element introduced or used in the snippet.(*@\color{black}@*). 
\end{lstlisting}
\end{minipage}
\begin{minipage}{.5\textwidth}        
\begin{lstlisting}[caption = ]
(*@{\color{red}{First One-Shot Prompt Setting@*)
(*@{\color{red}{User Prompt}@*)
Given a partial Java code snippet, compiler output/errors, and the expected outcome, your task is to generate the necessary code to complete the snippet. The additional code should address the issues indicated by the compiler output and achieve the desired outcome. Focus on enhancing the code header to make the snippet a compilable Java unit, without modifying the original code body.
Partial Code Snippet (Do Not Modify):
```
(*@{\color{blue}<code> @*)
```
Compiler Output/Errors:
```
(*@{\color{blue}<error> @*)
```
Expected Outcome:
Using the information provided by the compiler, the model should fill in the missing information of the partial code snippet, focusing solely on enhancing the code header. The goal is to make the partial code snippet a compilable Java unit without modifying any part of the existing code body. ...
Your first task is to analyze the provided code, identify what are missing and complete the Java snippet. Do not modify the original code. Additionally, you do not need to write a method to solve missing API calls.
After completing the code, your second task is to fill in type informations ... For example ...
Here is an exampular procedure of how you should construct your answer.
Suppose You have given a partial code snippet
// begin of the partial code snippet
    <sample code snippet>
//end of the partial code snippet

Here is what your response is supposed to be:
Code Approximation:
```java
... // LLM1 approximated code here.
```
Type Information:
... // LLM1 proposed type information here.
Disclaimer: Do not include any explanation or comments.
\end{lstlisting}
\end{minipage}
\vspace{-12pt}
\caption{Initial Design of the Prompts for Approximation}
\label{fig:approx-prompt}
\end{figure}
