Let us formally explain how it works. It
formulates the problem by maximizing the mutual information (MI)
between the minimal graph $\mathcal{G}_M$ and the input~$G_M$:
\begin{equation}\label{maineq}
\max_{\mathcal{G}_M} MI(Y,\mathcal{G}_M, \mathcal{X}_M) = H(Y) - H(Y|G=\mathcal{G}_M, F=\mathcal{X}_M)
\end{equation}
$Y$ is the outcome decision by the FA-GCN model. Thus, the entropy term
$H(Y)$ is constant for the trained FA-GCN model. Maximizing the $MI$
value for all $\mathcal{G}_M$ and $\mathcal{X}_M$ is equivalent to minimizing conditional
entropy $H(Y|G=\mathcal{G}_M, X|F=\mathcal{X}_M)$, which by definition of
conditional entropy can be expressed~as
\begin{equation}
  \label{eq2}
-\mathbb{E}_{Y|\mathcal{G}_M}
  [log P_{FA-GCN} (Y|G=\mathcal{G}_M,F=\mathcal{X}_M)
  \end{equation}
The meaning of this conditional entropy formula is a measure of how
much uncertainty remains about the outcome $Y$ when we know
$G=\mathcal{G}_M$ and $F=\mathcal{X}_M$. GNNEXplainer also limits the
size of $\mathcal{G}_M$ by $K_M$, i.e., taking $K_M$ edges that give
the highest mutual information with the prediction outcome $Y$.
%
Direct optimization of the formula~\ref{eq2} is not tractable, thus,
we can treat $\mathcal{G}_M$ as a random graph variable
$\mathcal{G}$. The objective in Equation~\ref{eq2} becomes:
\begin{equation}
  \label{eq3}
  \min_{\mathcal{G}} \mathbb{E}_{\mathcal{G}_M \sim \mathcal{G}} H(Y|G=\mathcal{G}_M,F=\mathcal{X}_M)
\end{equation}
\begin{equation}
  \label{eq4}
  \min_{\mathcal{G}} H(Y| G=\mathbb{E}_{\mathcal{G}}[\mathcal{G}_M], F = \mathbb{E}_{\mathcal{X}}[\mathcal{X}_M])
\end{equation}
From Equation~\ref{eq3}, we obtain Equation~\ref{eq4} with Jensen's
inequality.  The conditional entropy in Equation~\ref{eq4} can be
optimized by replacing $\mathbb{E}_{\mathcal{G}}[\mathcal{G}_M]$ to be
optimized by masking with $EM$ on the input graph $G_M$.
Now, we can reduce the problem to learning the mask $EM$.
Similar treatment is applied to $XM$.
Details on training can be found in~\cite{GNNExplainer}. The resulting
sub-graph $\mathcal{G}_M$ is directly used as an explainable set. The resulting sub-set of feature $\mathcal{X}_M$ is also an explainable set.
