\documentclass[conference]{IEEEtran}
\IEEEoverridecommandlockouts
% The preceding line is only needed to identify funding in the first footnote. If that is unneeded, please comment it out.
\usepackage{cite}
\usepackage{amsmath,amssymb,amsfonts}
\usepackage{algorithmic}
\usepackage{graphicx}
\usepackage{textcomp}
\usepackage{xcolor}
\def\BibTeX{{\rm B\kern-.05em{\sc i\kern-.025em b}\kern-.08em
    T\kern-.1667em\lower.7ex\hbox{E}\kern-.125emX}}

%\usepackage{amsmath,amssymb,amsfonts}
%\usepackage{algorithmic}
%\usepackage{graphicx}
%\usepackage{textcomp}
%\usepackage{xcolor}


%\usepackage{cite}
\usepackage{booktabs}   %% For formal tables:
                        %% http://ctan.org/pkg/booktabs
\usepackage{subcaption} %% For complex figures with subfigures/subcaptions
                        %% http://ctan.org/pkg/subcaption
\usepackage{array}
%\usepackage{amsmath,amsfonts}
%\usepackage{algorithm}
%\usepackage[noend]{algpseudocode}
%\usepackage{algorithmic}
%\usepackage{graphicx}
%\usepackage{textcomp}
\usepackage{float}
\usepackage{listings}
\usepackage{xspace}
\usepackage{multirow}
\usepackage{amsthm}
\usepackage{enumitem}

\newtheorem{definition}{Definition}
\usepackage{balance}
\usepackage{printlen}
\usepackage[skins]{tcolorbox}

%\usepackage{xcolor,pifont}
%\newcommand*\colourcheck[1]{%
%	\expandafter\newcommand\csname #1check\endcsname{\textcolor{#1}{\ding{52}}}%
%}
%\colourcheck{blue}
%\colourcheck{green}
%\colourcheck{red}

\newtcolorbox{myframe}[2][]{%
  enhanced,colback=white,colframe=black,coltitle=black,
  sharp corners,
  toprule=1.0pt,
  rightrule=0.3pt,
  leftrule=0pt,
  bottomrule=0pt,
  fonttitle=\itshape\scshape\large,
  left=0pt,right=5pt,top=5pt,bottom=3pt,
  attach boxed title to top right={yshift=-0.3\baselineskip-0.4pt,xshift=-5mm},
  boxed title style={tile,size=minimal,left=0.2mm,right=0.5mm,
    colback=white,before upper=\strut},
  title=#2,#1
}

%\newcommand{\code}[1]{{\footnotesize\textsf{#1}}}

\newcommand{\tool}{\textsc{DeepPDA}\xspace}

\newtheorem{Definition}{Definition}
\newtheorem{Claim}{Claim}
\newtheorem{Lemma}{Lemma}
\newtheorem{Theorem}{Theorem}

\newcolumntype{L}[1]{>{\raggedright\arraybackslash}p{#1}}
\newtheorem{observation}{Observation}
\newtheorem{property}{Property}
\newcommand{\code}[1]{{\footnotesize\texttt{#1}}}
\usepackage{amsthm}
 \definecolor{dkgreen}{rgb}{0,0.6,0}
\definecolor{gray}{rgb}{0.5,0.5,0.5}
\definecolor{mauve}{rgb}{0.58,0,0.82}
\lstset{frame=tb,
  language=Java,
  aboveskip=3mm,
  belowskip=3mm,
  showstringspaces=false,
  columns=flexible,
  basicstyle={\small\ttfamily},
  numbers=left,
  numberstyle=\tiny\color{gray},
  keywordstyle=\color{blue},
  commentstyle=\color{dkgreen},
  stringstyle=\color{mauve},
  breaklines=true,
  breakatwhitespace=true,
  tabsize=4
}


%\usepackage{tikz}
%\usetikzlibrary{shapes.arrows}
%\newcommand{\FancyUpArrow}{\begin{tikzpicture}[baseline=-0.3em]
%		\node[single arrow,draw,rotate=90,single arrow head extend=0.1em,inner
%		ysep=0.1em,transform shape,line width=0.03em,top color=green,bottom color=green!50!black] (X){};
%\end{tikzpicture}}

%\def\BibTeX{{\rm B\kern-.05em{\sc i\kern-.025em b}\kern-.08em
%    T\kern-.1667em\lower.7ex\hbox{E}\kern-.125emX}}


\begin{document}

%\title{Neural Partial Program Dependence Analysis}
%\title{A Fast and Accurate Neural Partial Program Dependence Analyzer}
\title{(Partial) Program Dependence Learning}
%\title{Enabling Control-Flow and Dependence Analysis in (Partial) Programs with Neural Networks}


%\author{\IEEEauthorblockN{1\textsuperscript{st} Given Name Surname}
%\IEEEauthorblockA{\textit{dept. name of organization (of Aff.)} \\
%\textit{name of organization (of Aff.)}\\
%City, Country \\
%email address or ORCID}
%\and
%\IEEEauthorblockN{2\textsuperscript{nd} Given Name Surname}
%\IEEEauthorblockA{\textit{dept. name of organization (of Aff.)} \\
%\textit{name of organization (of Aff.)}\\
%City, Country \\
%email address or ORCID}
%\and
%\IEEEauthorblockN{3\textsuperscript{rd} Given Name Surname}
%\IEEEauthorblockA{\textit{dept. name of organization (of Aff.)} \\
%\textit{name of organization (of Aff.)}\\
%City, Country \\
%email address or ORCID}
%\and
%\IEEEauthorblockN{4\textsuperscript{th} Given Name Surname}
%\IEEEauthorblockA{\textit{dept. name of organization (of Aff.)} \\
%\textit{name of organization (of Aff.)}\\
%City, Country \\
%email address or ORCID}
%\and
%\IEEEauthorblockN{5\textsuperscript{th} Given Name Surname}
%\IEEEauthorblockA{\textit{dept. name of organization (of Aff.)} \\
%\textit{name of organization (of Aff.)}\\
%City, Country \\
%email address or ORCID}
%\and
%\IEEEauthorblockN{6\textsuperscript{th} Given Name Surname}
%\IEEEauthorblockA{\textit{dept. name of organization (of Aff.)} \\
%\textit{name of organization (of Aff.)}\\
%City, Country \\
%email address or ORCID}
%}

\maketitle

\begin{abstract}
%% Code fragments from developer forums often migrate to applications due
%% to the code reuse practice. Owing~to~the incomplete nature of such
%% code fragments, analyzing them to early determine the presence of
%% potential vulnerabilities is challenging. In this work, we introduce
%% \tool, a neural network-based program dependence analysis tool for
%% both complete and partial programs. Inspired from dependency parsing
%% in natural language processing, {\tool} models this as the
%% statement-pairwise dependence decoding, with the support of both
%% intra-statement context learning and inter-statement context learning.
%% Our empirical evaluation shows that \tool predicts the CFG and PDG
%% edges in complete Java and C/C++ code with the F-scores of {\bf
%%   94.29\%} and {\bf 92.46\%}, respectively. The F-scores for partial
%% Java and C/C++ code range from 94.29--97.17\% and from
%% 92.46\%--96.01\%, respectively. We test the usefulness of the PDGs
%% predicted by \tool (PDG\textsuperscript{*}) on the task of
%% method-level vulnerability detection (VD). We discover that the
%% performance of the VD tool utilizing PDG\textsuperscript{*} is only
%% 1.1\% less than that utilizing the PDGs generated by a program
%% analysis tool. Furthermore, we report that by leveraging the PDGs
%% predicted by DeepPDA, a machine learning-based VD tool can detect 14
%% real-world vulnerable code snippets from StackOverflow.
%% %for 99 vulnerable code snippets from
  %% %StackOverflow, a VD tool can correctly identify \_\_ of them.
  
Code fragments from developer forums often migrate to applications due
to the code reuse practice. Owing to the incomplete nature of such
code fragments, analyzing them to early determine the presence of
potential vulnerabilities is challenging. In this work, we introduce
\tool, a neural network-based tool to build program dependence graph
(PDG) for both complete and partial programs. We design {\tool} to
efficiently incorporate intra-statement and inter-statement contextual
features into statement representations, thereby modeling program
dependencies as a statement-pair dependence decoding task. Our
empirical evaluation shows that \tool predicts the CFG and PDG edges
in complete Java and C/C++ code with combined F-scores of 94.29\% and
92.46\%, respectively. The F-scores for partial Java and C/C++ code
range from 94.29\%--97.17\% and 92.46\%--96.01\%, respectively.

%We also test the usefulness of the PDGs predicted by \tool (PDG*) on
%the task of method-level vulnerability detection (VD). We discover
%that the performance of the VD tool utilizing PDG* is only 1.1\% less
%than that utilizing the PDGs generated by a program analysis
%tool. Furthermore, we report the detection of 14 real-world vulnerable
%code snippets from StackOverflow by a machine learning-based VD tool
%that employs the PDGs predicted by \tool for these code snippets.

\end{abstract}


%\begin{IEEEkeywords}
%component, formatting, style, styling, insert
%\end{IEEEkeywords}

\input{icse23-demo-intro}
\section{Model Architecture}
\label{sec:arch}


%\input{intro1}
%\input{motiv1}
%\input{overview}
%\input{our_approach}
%\input{empirical_eval}
%\input{results-effectiveness}
%\input{miss.tex}
%\input{results_rq2}
%\input{results_rq3}
%\input{discussion}
%\input{related_work}
%\section{Conclusion and Future Work}
We introduce a neural network-based program dependence analysis tool, \tool, which extends the construction of CFG/PDGs to partial code snippets. Owing to the significant speed markup ($\sim$380$\times$), we envision an opportunity for the integration of such a tool in the software development process, which will help facilitate real-time program analysis. Our tool can also benefit SE tasks such as vulnerability detection (VD), code completion, fault localization, etc., that can tolerate low levels of inaccuracies. The current prototype is highly accurate and supports Java and C/C++ at this point -- we plan to conquer other programming languages as well. In the future, we seek to combine our tool with existing PA tools, while also exploring cross-segment dependency-capturing strategies, so as to make it robust across larger codebases. 


%\balance
\bibliographystyle{IEEEtran}

\bibliography{reference,icse21IntVD,FL}



\end{document}
