%
\usepackage{cite}
\usepackage[T1]{fontenc}
\usepackage{times}
\usepackage{url}
\usepackage{amsmath}
\usepackage{fancyhdr}
\usepackage{fancyvrb}
\usepackage{fancybox}
\usepackage{color}
\usepackage{colortbl}
\usepackage[table]{xcolor}
%\usepackage{tex-helpers/mathpartir}
\usepackage{amssymb}
\usepackage{xspace}
\usepackage{comment}
\usepackage{graphicx}
\graphicspath{{figures/}}
\usepackage{epsfig}
\usepackage{wrapfig}
\usepackage{multirow}
\usepackage{subfig}

% Added listing for code listings
\usepackage{listings}
\usepackage{relsize}
\usepackage{setspace}
\usepackage{sidecap}
\usepackage{algorithm}
\usepackage{algorithmic}

\textwidth=16.5cm 
\textheight=22.5cm 
\textwidth=16.5cm 
\textheight=55.5pc
\topmargin=-0.5cm 
\headsep=0cm 
\headheight=0cm 
\oddsidemargin=0cm
\evensidemargin=0cm 
\marginparwidth=0cm
\parskip=0cm
\itemsep=0.1pt
\parindent=0.5cm

%Set table separation 
\setlength{\tabcolsep}{1pt}

% Allow figures to take up the entire page
\renewcommand\floatpagefraction{.99}
\renewcommand\topfraction{.99}
\renewcommand\bottomfraction{.90}
\renewcommand\textfraction{.01}   
\setcounter{totalnumber}{50}
\setcounter{topnumber}{50}
\setcounter{bottomnumber}{50}

% formatting for grammars
%\input{tex-helpers/obey}
%\input{tex-helpers/grammar}
%\renewcommand{\nonterm}[1]{\mbox{\textit{#1}}}
\newcommand{\oneormore}[1]{#1\ensuremath{^+}}

\newcommand{\opt}[1]{\textbf{[}#1\textbf{]}}

\newcommand{\mc}[1]{\mbox{\rm \ensuremath{\text{\code{#1}}}}} 
\newcommand{\loc}{\ensuremath{\mathord{\mathit{loc}}}}
\newcommand{\ec}{\ensuremath{\mathop{\mathbb{E}}}}
\newcommand{\hole}{\ensuremath{\mathord{\mathit{-}}}}
\newcommand{\reducesto}{\hookrightarrow}

\newcommand*{\seq}[1]{\ensuremath{\left\langle {#1} \right\rangle}}
\newcommand{\config}{\seq}
\newcommand{\udot}{\mathbin{\sqcup \kern-0.53em \cdot \,}}

% Aux functions
\newcommand{\auxFunc}[1]{\ensuremath{\mathop{\mathit{#1}}}}
\newcommand{\concat}{\auxFunc{concat}}
\newcommand{\reverse}{\auxFunc{reverse}}

% Type checking macros
% change symbol type of : from mathrel 
%\DeclareMathSymbol{:}{\mathbin}{operators}{"3A} 
\newcommand{\OK}{\mbox{OK}}
\newcommand{\OKin}{\mbox{OK in }}
\newcommand{\isType}{\mbox{\textit{isType}}}
\newcommand{\isThunkType}{\mbox{\textit{isThunkType}}}
\newcommand{\isClass}{\mbox{\textit{isClass}}}
\newcommand{\Types}{\mbox{\textit{Types}}}
\newcommand{\Names}{\mbox{\textit{Names}}}
\newcommand{\TypeEnv}{\mbox{\textit{TypeEnv}}}
\newcommand{\VD}{\mbox{\textit{VD}}}
\newcommand{\TypesInOrder}{\mbox{\textit{typesInOrder}}}
\newcommand{\dom}{\mbox{\textit{dom}}}
\newcommand{\rng}{\mbox{\textit{rng}}}
\newcommand{\POWERSET}[1]{\mbox{\textit{PowerSet}}(#1)}
\newcommand{\delete}{\mbox{\textit{delete}}}
\newcommand{\mklist}{\mbox{\textit{mksupers}}}
\newcommand{\uminus}{\mbox{$\cup\!\!\!\!-$}}
\newcommand{\iminus}{\mbox{$\cap\!\!\!\!-$}}
\newcommand{\rname}[1]{$\TirName{(#1)}$} % for inline inferrule names
\newcommand{\STO}{\ensuremath{<:}}  % ``subtype of''
\newcommand{\consistent}{\ensuremath{\approx}}

% Notations 
\newcommand{\produces}{\rightsquigarrow}
\newcommand{\refinedBy}{\sqsubseteq}

% Macros for the Figure Editor Example
\newcommand{\FElement}{\mbox{\texttt{FElement}~}}
\newcommand{\ChangedFE}{\mbox{\texttt{changedFE}~}}
\newcommand{\FEChange}{\mbox{\texttt{FEChange}~}}


% Macros for defining phase transition analysis
\newcommand{\cfg}{\ensuremath{{\cal CFG}}\xspace}
\newcommand{\globTypeMap}{\ensuremath{T}\xspace}

%Redefinition of some math commands widely used outside
%mathmode
\newcommand{\memOf}{\ensuremath{\in}\xspace}

% Help LaTeX not violate the column margins
\tolerance=50000

% settings for listings
\definecolor{lightgray}{gray}{0.97}
\definecolor{darkgray}{gray}{0.5}
\definecolor{OliveGreen}{cmyk}{0.64,0,0.95,0.40}
\definecolor{DarkGreen}{cmyk}{0.58,0,0.66,0.26}
\definecolor{LightRed}{cmyk}{0,0.682,0.728,0}
\definecolor{purple}{cmyk}{0.41,0.73,0,0}

\lstset{
	language=bash, emph={},
	mathescape=false, escapechar=@,
	backgroundcolor=\color{lightgray},
	commentstyle=\color{darkgray},
	keywordstyle=\color{OliveGreen}\bfseries,
	basicstyle=\relsize{-3}\sffamily,
	numberstyle=\scriptsize\sffamily,
	emphstyle=\color{purple},
	emphstyle={[2]\color{LightRed}},
	commentstyle=\color{DarkGreen},
	stringstyle=\color{OliveGreen},
	numbers=left, stepnumber=1,
	numberblanklines=false,
	numberstyle=\tiny,
	numbersep=-3pt,
	frame=none, framexleftmargin=0pt, framexrightmargin=0pt, 
	%xleftmargin=15pt, xrightmargin=4pt,
	columns=flexible, breaklines=true,
	showspaces=false, showstringspaces=false, showtabs=false, tabsize=2,
	morekeywords={input,exists,foreach,ifall,output,of,weight,stop,visit,before,after},
	emph={int,string,bool,time,array,stack,map,visitor,%
true,false,%
top,sum,mean,maximum,minimum,set,collection,%
Project,CodeRepository,Revision,ChangedFile,ASTRoot,Namespace,Declaration,Type,Method,Variable,Statement,Expression,Modifier,%
ExpressionKind,NEW,LITERAL,EQ,NEQ,%
TypeKind,CLASS,ANONYMOUS,%
ModifierKind,OTHER,%
ChangeKind,DELETED,%
StatementKind,IF,%
RepositoryKind,SVN},
	emph={[2]isfixingrevision,getast,iskind,hasfiletype,isliteral,getsnapshot,has_modifier_public,%
format,def,len,match,lowercase,yearof,haskey,remove,strfind,push,pop},
}

% could use \relsize{-2} instead of \scriptsize below
\newcommand{\FIGCODEFONT}{\relsize{-2.5}\ttfamily}

% cross referencing
%\newcommand{\algref}[1]{Algorithm~\ref{#1}}
%\newcommand{\figref}[1]{Figure~\ref{#1}\xspace}
%\newcommand{\tabref}[1]{Table~\ref{#1}\xspace}
%\newcommand{\fignref}[1]{Figure~\ref{#1}\xspace}
%\newcommand{\secref}[1]{Section~\ref{#1}\xspace}
%\newcommand{\secnref}[1]{Section~\ref{#1}\xspace}

%\newcommand{\etal}{~\textit{et al.}\@\xspace}
\newcommand{\kind}{\textit{kind}\xspace}
\newcommand{\KIND}{\textit{KIND}\xspace}

\newcommand{\lang}{\textit{Boa}\@\xspace}

% Theorems environments...
%{theorems}
\newtheorem{theorem}{Theorem}[section]
\newtheorem{axiom}[theorem]{Axiom}
\newtheorem{corollary}[theorem]{Corollary}
\newtheorem{definition}[theorem]{Definition}
\newtheorem{example}[theorem]{Example}
\newtheorem{fact}[theorem]{Fact}
\newtheorem{lemma}[theorem]{Lemma}
\newtheorem{proposition}[theorem]{Proposition}
\newtheorem{remark}[theorem]{Remark}
\newtheorem{conjecture}[theorem]{Conjecture}
% Some helpful notation
\newcommand{\PROOF}{{\em Proof:\/}~~}
\newcommand{\PROOFSKETCH}{{\em Proof Sketch:\/}~~}
\newcommand{\QED}{\rule{0.4em}{0.65em}}

\definecolor{light-gray}{gray}{0.9}
\definecolor{very-light-gray}{gray}{0.95}

% Change section headings to look nicer
\makeatletter
\renewcommand\thesection{{\large\Alph{section}.}}
\renewcommand\section{\@startsection {section}{1}{\z@}%
                                   {-3.5ex \@plus -1ex \@minus -.2ex}%
                                   {2.3ex \@plus.2ex}%
                                   {\large\scshape\bf}}
\makeatother

\renewcommand\thesubsection{{\normalsize\Alph{section}.}{\normalsize\arabic{subsection}}}

\makeatletter
\renewcommand\subsection{\@startsection {subsection}{1}{\z@}%
                                   {-3.5ex \@plus -1ex \@minus -.2ex}%
                                   {2.3ex \@plus.2ex}%
                                   {\normalsize\scshape\bf}}
\makeatother

\renewcommand\thesubsubsection{{\normalsize\Alph{section}.}{\normalsize\arabic{subsection}.}{\normalsize\arabic{subsubsection}}}

\makeatletter
\renewcommand\subsubsection{\@startsection {subsubsection}{1}{\z@}%
                                   {-3.5ex \@plus -1ex \@minus -.2ex}%
                                   {2.3ex \@plus.2ex}%
                                   {\normalsize\scshape\bf}}
\makeatother

\newcommand\para[1]{\vspace{-1em}\paragraph{#1\ }}

% Different font in captions
\newcommand{\captionfonts}{\footnotesize}

% formatting of initial section quotations
\newcommand{\QUOTATION}[1]{\begin{flushright}\begin{footnotesize}\emph{#1}\end{footnotesize}\end{flushright}}
 
\makeatletter  % Allow the use of @ in command names
\long\def\@makecaption#1#2{%
  \vskip\abovecaptionskip
  \sbox\@tempboxa{{\captionfonts #1: #2}}%
  \ifdim \wd\@tempboxa >\hsize
    {\captionfonts #1: #2\par}
  \else
    \hbox to\hsize{\hfil\box\@tempboxa\hfil}%
  \fi
  \vskip\belowcaptionskip}
\makeatother   % Cancel the effect of \makeatletter

%\usepackage[ps2pdf,bookmarks=true]{hyperref}

\definecolor{light-gray}{gray}{0.9}
\definecolor{dark-gray}{gray}{0.7}

\newcommand\doctitle[1]{\newpage \setcounter{page}{1}\thispagestyle{fancyplain} \headheight=14pt%
                        \fancyhead[C]{\large{\bf #1}}\xspace\vspace{0.5em}}


                                                                        
