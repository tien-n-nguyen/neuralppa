%\begin{figure}[t]
\begin{wrapfigure}{r}{0.65\textwidth}
	\centering
	\lstset{
		numbers=left,
		numberstyle= \tiny,
		keywordstyle= \color{blue!70},
		commentstyle= \color{red!50!green!50!blue!50},
		frame=shadowbox,
		rulesepcolor= \color{red!20!green!20!blue!20} ,
		xleftmargin=1.5em,xrightmargin=1em, aboveskip=1em,
		framexleftmargin=1.5em,
                numbersep= 5pt,
		language=Python,
    basicstyle=\tiny\ttfamily,
    numberstyle=\tiny\ttfamily,
    emphstyle=\bfseries,
                moredelim=**[is][\color{red}]{@}{@},
		escapeinside= {(*@}{@*)}
	}
\begin{minipage}{.45\textwidth}        
\begin{lstlisting}[caption = ]
(*@{\color{red}{ONE-PROMPT SETTING@*)
For the given code snippet, Predict the code coverage. The code coverage indicates whether a statement has been executed (*@\color{black}{or not}@*). 
> (*@\color{black}{if}@*) the line (*@\color{black}{is}@*) executed
! (*@\color{black}{if}@*) the line (*@\color{black}{is not}@*) executed

Example output:
> line1
! line2
> line3
...
> linen

You need to develop a plan (*@\color{black}{for}@*) step by step execution of the code snippet. 
Below (*@\color{black}{is}@*) an illustration of the process you need to follow to predict the code coverage of the given code snippet. 

(*@\color{orange}{Example - Given Code Snippet:}@*) 
number = 15
if number % 2 == 0:
    print("{} is an even number.".format(number))
else:
    print("{} is an odd number.".format(number))

(*@\color{orange}{PLAN :}@*) 
Step 1: Variable Initialization: Initialize the number variable with a specific value. In this case, it(*@\color{black}{'}@*)s (*@\color{black}{set}@*) to 15. Statements "number = 15" will be executed. "if number % 2 == 0:" will be executed
Step 2: Operation : Use the % (modulo) operator to check (*@\color{black}{if}@*) the number (*@\color{black}{is}@*) divisible by 2. If the result (*@\color{black}{is}@*) 0, the number (*@\color{black}{is}@*) even. If (*@\color{black}{not}@*), it(*@\color{black}{'}@*)s odd.
Step 3: Branching (*@\color{black}{if-else}@*) Block: Enter the if block (*@\color{black}{if}@*) the number (*@\color{black}{is}@*) even. Since 15%2 results (*@\color{black}{in}@*) 1, statement "print("{} (*@\color{black}{is}@*) an even number.".format(number))" will (*@\color{black}{not}@*) be executed but the else block will (*@\color{black}{if}@*) the number (*@\color{black}{is}@*) odd. Statements "else:" (*@\color{black}{and}@*) "print("{} (*@\color{black}{is}@*) an odd number.".format(number))" will be executed
Step 4: Output: The required print statement will be executed based on the output of the if-else block

(*@\color{orange}{So the code coverage for the given code snippet will be:}@*) 
> from math import factorial
> number = 15
> if number % 2 == 0:
!     print("{} is an even number.".format(number))
> else:
>     print("{} is an odd number.".format(number))

In a similar fashion, develop a plan of step by step execution of the below code snippet (*@\color{black}{and}@*) predict the code coverage - 
(*@{\color{red}{<< Test Code... >>@*)
\end{lstlisting}
\end{minipage}
\vspace{-18pt}
\caption{Example on one-prompt setting with {\tool}}
\label{fig:one-prompt}
\end{wrapfigure}
