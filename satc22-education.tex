%\section{Education and Dissemination Plan - Curriculum Development Activities}
\section{Dissemination and Educational Plan}
\label{edu}

\paragraph{Engaging Students into Research}

This project will create opportunities for students at UT-Dallas to
participate into cutting-edge research: PI Nguyen has currently supported
four female students (2 PhD students and 2 undergraduates), with a total
of four PhD students, two M.Sc. students, and four undergraduates.

%has worked with 6 undergraduate students (i.e., two are female
%students) and two master students (i.e., independent
%study). Additionally, PI Wang is actively doing research with a group
%of two undergraduate students (Vrushali Koli (female) from NJIT and
%Delmond Wyllis from Rowan Univ. in NJ) and \underline{six high-school
%students} in the NJ Governor's STEM Scholars Program.  PI Wang
%supervise 4 Ph.D students (one female) and 1 master student.


%(2) {\em UTD's Collegium Honors Program}:
%{\em ISU's Freshman Honor program}: 
%PI Nguyen has been engaging two freshman honor students into his
%research program since he joined UTD in 2016. 
%REU supplements will be requested to support this undergraduate research; 
%
%(3) {\em HackersUTD}: This Fall, UT Dallas Computer Science department
%welcomed 2,876 students, including 466 CS/software engineering
%freshmen, with activities and events as a way to welcome and
%familiarize students with the UT Dallas CS. We will engage members of
%this organization; 
%(3) UT Dallas works with North Texas high school
%seniors to host IT empowerment for their camps. To generate interests
%in computing studies from {\em high-school} students in Dallas area,
%we will involve them in design projects that target the use of
%software in teaching {\em K-12} subjects; 
%(2) {\em UTD's Women Who
%Compute}: The PI Nguyen has actively used this program to generate interests
%in SE research from women students. From the past, PI Nguyen has
%mentored 3 women students; 
%(Dong Fei, Kristina Gervais, Taylor Schreck); 
%(3) {\em Involving Under-represented Minorities}: We
%will attract minority students funded by GEM fellowships, involve
%high-school teachers via Alliances for Graduate Education and the
%Professoriate.
%and UT Dallas Programs for Minors: we will work
%with this program to recruit more student in minority.

%This project will also create opportunities for students at UT-Dallas:
PI Nguyen has extensive experience in engaging students in
outreach programs: (1) {\em UTD's Collegium Honors Program}: He
has been engaging several freshman honor students into his research
program since 2005. (2) {\em HackersUTD}: In Fall'19, UT Dallas CS
welcomed 2,876 students, including 466 freshmen, with activities and
events as a way to familiarize students with CS/SE. We will use the
results of this project in this program.
%
(3) UT Dallas works with {\em North Texas high school seniors to host
IT empowerment} for their camps. To generate interests in computing
studies from {\em high-school} students in Dallas area, PI Nguyen has
been involving them in design projects that target the use of software
in teaching {\em K-12} subjects; (4) {\em UTD's Women Who Compute}: he
has actively used this program to generate interests in SE research
from women students. PI Nguyen has mentored several {\em female
undergraduate students} (Weining Gao, Kristina Gervais, Taylor
Schreck, etc.); (4) {\em Involving Under-represented Minorities}: We
will attract minority students funded by GEM fellowships, involve
high-school teachers via Alliances for Graduate Education and the
Professoriate; and (5) {\em UT Dallas Programs for Minors}: we will
work with this program to recruit more student in minority.


\paragraph{Dissemination of Research and Teaching Materials}

Beside publications at professional \emph{conferences, journals} and
public Web sites, we will use the available resources at UT-Dallas
through various forums for dissemination.
%
%PI Wang is a member of DIMACS~\cite{dimacs}, the Center for Discrete
%Mathematics and Theoretical Computer Science as an NSF-funded Science
%and Technology Center (STC) and a New Jersey Commission on Science and
%Technology Advanced Technology Center. DIMACS has over 350 members
%across the U.S. PI Wang will continue to advocate our research through
%DIMACS events, research and academic programs.  PI Wang will use
%the \textit{annual NJIT Research Showcase and President Forum} to show
%our accomplishments to all NJIT people.
PI Nguyen will continue to use the following resources: (1) TExAs
Software Engineering Research (TEASER) Doctoral Symposium: where SE
researchers from the DFW Metroplex (and beyond) meet to discuss and
work on SE topics.  The mission of TEASER is to provide a supportive
space in which PhD students can present and receive feedback on their
research work, while at the same time giving both researchers and
students a venue to get to know one another and network productively.
(2) American Society for Engineering Education: PI Nguyen, with his
prior NSF-funded TUES project, has disseminated educational results
via this channel.
%He will continue to disseminate educational outcomes via this
%educational society.
(3) Leadership through Engineering Academic Diversity (LEAD): PI
Nguyen have served as mentors for this program which aims to enrich
educational experience of {\em minority engineering students}.

\paragraph{Research Community Building.}

PI Nguyen has successfully organized the 1st {\bf International
Workshop on Representation Learning for Software Engineering and
Programming Languages (RL+SE\&PL 2020)}~\cite{rlsepl}, associated with
ESEC/FSE'20. The workshop had 104 paid registers and featured one
keynote speaker (Dr. Miltos Allamanis, well-known CRL scientist at
Microsoft Research) and six technical presentations. RL+SE\&PL'20 has
sparked constructive and inspiring discussions. With its success, we
are building a diverse and active community. We plan to build on the
success of the first edition to continue this series to disseminate
ideas to a wider community, and to build a larger community bridging
SE and ML.


\noindent {\bf Undergraduate and Graduate Software Engineering (SE) Education.} 
%PI Wang teaches fundamental and core SE courses NJIT: Building Web Applications (IS218) and System Design (IS390).
PI Nguyen is one of the key SE faculty members in the BS, MS, and Ph.D.
degree programs in SE at UT Dallas. 
%
He is one of the faculty who has initiated and contributed to
Undergraduate SE Program when he was at Iowa State University.
%He has
%successfully introduced several courses including Software
%Architecture and Design (CprE339) and Software Project Management
%(CprE329). 
PI Nguyen will introduce a deep
learning component and a new course in NLP+SE.
The key teaching philosophy in this course is the combination of theory and practice.
%in which students will be introduced different principles and theories in the application of NLP in SE artifacts. 
%The tentative modules include 1) basic principles, processes, and paradigms in NLP, 2) programming languages versus natural languages, 3) API usages and reuse, 4) Statisical models used in SE applications, 5) machine translation and code migration, 6) language models for source code, 7) applications of machine translation in SE, 8) word embeddings and SE applications, 9) deep learning and SE applications, etc.
%\noindent {\em Graduate SE Education} 
He will teach a newly developed course, 
Selected Topics on AI in SE and PL,
including relevant topics to this proposal, e.g, 
AI in bug detection, fault localization, automated repair. 
%The PI Nguyen works
%with other UTD faculty to develop a series of six to seven SE graduate
%courses that will be offered at least every semester. This year, the
He introduced a new graduate level course on ``AI/ML for
Code''. The PI will introduce a new graduate course on the topic of
NLP+SE.
%Tools developed by this research will be used in class projects. 
The course will focus on several SE advanced methods that
aim to help advance software engineering with AI/ML techniques. 
The tentative topics include 1) program analysis, 2) source code
analysis with AI/ML, 3) cross-language analysis,
4) AI/ML for code, etc.
%4) code and text retrieval in SE applications, 
%3) NLP techniques and
%type inference, 4) NLP and de-ofuscation, 5) bug-fixing and
%machine learning.



