\section{PI's Prior Relevant NSF Supports and Qualifications for this Project}
\label{prior}

%PI Wang has no prior relevant NSF supports.
%His relevant work on improving software maintenance, quality and reliability has been published in
%OOPSLA~\cite{yioopsla19}, ASE~\cite{son-2019-ase}, EMSE~\cite{noei2019towards,wang2016improving}, ICWS~\cite{venkatesh2016client}, ICSOC~\cite{wang2014developers}, and MSR~\cite{wang2013improving,wang2019extracting}.
%two ICSE submissions, and one PLDI submission.
PI Nguyen. CCF-1723215, \$260,709, 07/01/16-30/06/21, ``Collaborative
Research: Exploiting the Naturalness of Software''.
%
\noindent {\bf Intellectual Merit.} The work has the following
thrusts: (1) investigating NLP techniques for SE
applications, and (2) Developing a statistical machine translation
model for language migration.
%Investigating the integration of semantic information
%including data types, semantic roles, etc., into a language model, (2)
%Developing an accurate code-completion tool using statistical semantic
%language model, and (3) Developing a statistical machine translation
%model for language migration.
%
\noindent {\bf Broader Impact.}
%We developed practical software engineering tools, using statistical
%NL techniques for (a) code suggestion, completion, and correction
%tools, (b) assisting tools for programmers, (c) summarization, (d)
%retrieval, and (e) porting tools.
The project has several publications at top-tier SE
conferences, e.g., ICSE, FSE, ASE, TSE.
%including ICSE
%papers~\cite{icse05,icse07,icse09,icse10,icse11}, FSE
%papers~\cite{fse06,fse09,fse11}, ASE
%papers~\cite{ase06,ase08,ase08-2,ase09,ase10,ase11-phpsync,ase11-bugscout,ase11-idiff},
%OOPSLA papers~\cite{oopsla04,oopsla06,oopsla10}, and TSE
%papers~\cite{tse08,tse11}.
%
He has been awarded {\bf 4 ACM SIGSOFT Distinguished Paper Awards, one
  Best Paper Award, and one best ICSE Formal Research Demonstration
  Award} at the top-tier SE conferences including ICSE, FSE, and ASE,
one {\bf IEEE TCSE Distinguished Paper Award}. He has served on
Program Committees and Program Boards of ICSE, FSE, ASE, OOPSLA,
ECOOP. PI Nguyen was a Program Co-Chair of ASE'17. Since 2005, PI
Nguyen published at the top-tier Software Engineering conferences
including 21 ICSE full papers, 12 ESEC/FSE full papers, 13 ASE full
papers, 3 OOPSLA full papers. From Google Scholar, PI
Nguyen's H-index is 49, Citations: 9,431,  Citations in past 5-years: 5,879. From CSRankings.org, he
is ranked at the 3rd place among all Software Engineering researchers
in the US in the past 10 years.

%making software development more accessible, enjoyable and productive;
%and therefore will broadly enhance the value that software
%professionals deliver to business and society.

%Dr. Nguyen is currently the PI of the NSF-funded project, ``
%Collaborative Research: Exploiting the Naturalness of Software'', that
%will end August 2018. The work has three thrusts of research.  (1)
%Investigating the integration of semantic information including data
%types, semantic roles, etc., into a language model, (2) Developing an
%accurate code-completion tool using statistical semantic language
%model, and (3) Developing a statistical machine translation model for
%language migration. So far, the majority of the tasks have been
%%completed except empirical evaluation in a real-word setting is
%needed.



%Several papers on our results were published through ICSE
%2010~\cite{icse10}, ASE 2010~\cite{ase10}, OOPSLA
%2010~\cite{oopsla10}, ICSM 2010~\cite{icsm10}, ICSE
%2011~\cite{icse11,nier11-1,nier11-2}, FSE 2011~\cite{fse11}, ASE
%2011~\cite{ase11-phpsync,ase11-bugscout,ase11-idiff}, and an IEEE TSE
%journal article~\cite{tse11}.

%Our next phase will be involved more with the tasks (2) and (3).

%Dr. Nguyen is currently an PI on a NSF-funded project: \#CCF-1018600,
%``Find and Fix Similar Software Bugs'', 08/15/2010 through
%07/31/2013. In this project, an empirical study will be conducted to
%collect, analyze, and understand the nature and characteristics of
%recurring and similar bugs within one and across multiple
%systems. This project is expected to advance software engineering
%knowledge on the theoretical foundation, concepts, practical
%techniques, and automated tools to (1) capture the characteristics and
%measure the similarity of code units involved in prior known fixed
%bugs, (2) identify the locations of potential buggy units and derive
%the guidelines to fix them by matching them to the relevant peer code
%units of the known bugs, and (3) support the similar bug detection and
%fixing process. Teaching modules and validation efforts in this
%project will involve students and professionals, promoting teaching
%and training software quality assurance.

%Dr. Nguyen is an expert in software building, software configuration
%management (SCM), and software refactoring. His research work on SCM
%has been published at various prestigious software engineering
%journals and conferences including clone-aware configuration
%management and operation-based version control (TSE'11~\cite{tse11},
%FASE'10~\cite{fase10}, ASE'09~\cite{ase09}), software
%refactoring-aware SCM and software merging (TSE'08 \cite{tse08},
%ICSE'07~\cite{icse07}, FSE'06~\cite{fse06}, OOPSLA'06
%demo~\cite{oopsla06}), object-oriented configuration management
%(ICSE'07~\cite{icse07}, WWW'06~\cite{www06}, ICSE'05~\cite{icse05},
%ICSM'05~\cite{icsm05}).

%His other software maintenance works include clone-related bug
%detection (TSE'11 \cite{tse11}), bug localization
%(ASE'11~\cite{ase11-phpsync}), bug localization from bug reports
%(ASE'11~\cite{ase11-bugscout}), recurring bug detection
%(ICSE'10~\cite{icse10}), API misuse detection
%(OOPSLA'10~\cite{oopsla10}, FSE'09~\cite{fse09}), recurring
%vulnerabilities detection (ASE'10~\cite{ase10}), etc.




%Since 2005, his work has been resulting several publications at
%top-tier SE conferences including 5 ICSE
%papers~\cite{icse05,icse07,icse09,icse10,icse11}, 3 FSE
%papers~\cite{fse06,fse09,fse11}, 8 ASE
%papers~\cite{ase06,ase08,ase08-2,ase09,ase10,ase11-phpsync,ase11-bugscout,ase11-idiff},
%2 WWW papers~\cite{www04,www06}, 3 OOPSLA
%papers~\cite{oopsla04,oopsla06,oopsla10}, and 2 FASE
%papers~\cite{fase09,fase10}, 2 TSE papers~\cite{tse08,tse11}.
