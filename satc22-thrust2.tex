\section{Explainable AI-enabled Software Vulnerability Assessment}
\label{sec:thrust2}

\subsection{Motivating Example}
\label{exe:sec}

%CVE-2021-37714: description
%CVSS Scores

%Input-Output

%Code Change: src/main/java/org/jsoup/parser/HtmlTreeBuilder.java

%Crash-point or infinite loop: HtmlTreeBuilderState.java


Let us present an example from an HTML parser,~named {\em
  jsoup}, and our observations.
%for motivation.
Fig.~\ref{CVSS-tab} displays the information on the vulnerability
CVE-2021-37714 that was reported on {\em jsoup}, and published on
08/18/21.~The change that was deemed to contribute to the
vulnerability were committed at version 1.12.1 to the method
\code{process(Token,HtmlTreeBuilder)} of the
\code{Html\-Tree\-Builder\-State} class (lines 10--11, and 12 of
Fig.~\ref{fig:motiv-code}). That change directly uses the value
returned from \code{reset\-Inser\-tion\-Mode()} as the condition to
insert \code{starTag} (line 13). With this~change, certain input HTML
code with a specific start tag could make the program go to line 16
with a recursive call to the method \code{process(...)}. That~call
resulted in an NullPointerException at line 3 as noted in the
log:~{\em ``java.\-lang.\-Null\-Pointer\-Ex\-ception: Cannot invoke
  "org.\-jsoup.\-nodes.\-Element.\-normalName()" because the return
  value of
  "org.\-jsoup.\-parser.\-HtmlTree\-Builder.\-current\-Element()" is
  null.''}. In other cases, the parser can get stuck, i.e., {\em
  ``loop indefinitely until canceled''} as described in the official
description associated with CVE-2021-37714 (Fig.~\ref{CVSS-tab}). Due
to those effects, this vulnerability is considered as {\em a denial of
  service (DoS)}.

\begin{figure}[t]
  \begin{flushleft}
    \footnotesize
\textbf{Vulnerability Details: CVE-2021-37714}\\
\textbf{1. Description}:
{\em jsoup is a Java library for working with HTML. Those using jsoup versions prior to 1.14.2 to parse untrusted HTML or XML may be vulnerable to DOS attacks. If the parser is run on user supplied input, an attacker may supply content that causes the parser to get stuck (loop indefinitely until cancelled), to complete more slowly than usual, or to throw an unexpected exception. This effect may support a denial of service attack. The issue is patched in version 1.14.2. There are a few available workarounds. Users may rate limit input parsing, limit the size of inputs based on system resources, and/or implement thread watchdogs to cap and timeout parse runtimes.
  Publish Date : 2021-08-18 Last Update Date : 2022-02-07}
%\textbf{2. Vulnerability Type(s)}: Denial Of Service
%{\bf 3. CVSS Score:} ...\\
%{\bf 4. Detailed CVSS Grades:}\\
\end{flushleft}
%  \centering
%  \tabcolsep 3pt
%  \scriptsize
%  \begin{tabular}{lll}
%   Vulner. Assess. Type   & Value & Description \\
%      \hline
%    Confidentiality Impact & {\bf None}  & No impact to the confidentiality \\
%    Integrity Impact & {\bf None}  & No impact to the integrity \\
%    Availability Impact & {\bf Partial} & There is reduced performance or\\
%    & & interruptions in availability\\
%    Access Complexity & {\bf Low} & Specialized access conditions or \\
%    & & extenuating circumstances do not exist\\
%    & & Little knowledge is required to exploit\\
%    Authentication & {\bf Not Req} & Authentication is not required \\
%    & & to exploit the vulnerability\\
%    Gained Access & {\bf None}  & No gained access with the vulnerability \\
%    Acccess Vector & {\bf Local} & The vulnerability is in the local parser \\
%    \end{tabular}%
  %  \label{CVSS:tab}%
  \vspace{-16pt}
\caption{Vulnerability Details: CVE-2021-37714}
\label{CVSS-tab}
\end{figure}

%\begin{wrapfigure}{l}{0.5\textwidth}
%	\centering
%	\lstset{
%		numbers=left,
%		numberstyle= \tiny,
%		keywordstyle= \color{blue!70},
%		commentstyle= \color{red!50!green!50!blue!50},
%		frame=shadowbox,
%		rulesepcolor= \color{red!20!green!20!blue!20} ,
%		xleftmargin=1.5em,xrightmargin=0em, aboveskip=1em,
%		framexleftmargin=1.7em,
%                numbersep= 5pt,
%		language=Java,
%    basicstyle=\tiny\ttfamily,
%    numberstyle=\tiny\ttfamily,
%    emphstyle=\bfseries,
%                moredelim=**[is][\color{red}]{@}{@},
%		escapeinside= {(*@}{@*)}
%	}
%	\begin{lstlisting}[]
%// .../jsoup/parser/HtmlTreeBuilderState.java
%boolean process(Token t, HtmlTreeBuilder tb) { ...
%  if (t.isCharacter()&& inSorted( (*@{\color{red}{tb.currentElement().normalName()}@*),InTableFoster)){
%     ...
%     return tb.process(t);
%  }
%  ...
%  } else {
%      tb.popStackToClose(name);
%(*@{\color{orange}{- \quad \quad tb.resetInsertionMode();}@*)
%(*@{\color{orange}{- \quad \quad if (tb.state() == InTable) \{}@*)
%(*@{\color{cyan}{+ \quad \quad if (!tb.resetInsertionMode()) \{}@*)
%         tb.insert(startTag);
%         return true;
%      }
%(*@{\color{red}{\quad \quad \quad return tb.process(t, InHead);}@*)
%      ...
%}
%	\end{lstlisting}
%        \vspace{-15pt}
%        \caption{Code Change at Version 1.12.1 for CVE-2021-37714}
%        \vspace{-6pt}
%        \label{fig:motiv-code}
%\end{wrapfigure}

%     tb.newPendingTableCharacters();
%     tb.markInsertionMode();
%     tb.transition(InTableText);


%vulnerability}.

Fig.~\ref{cvss} shows the Common Vulnerability Scoring System
grades (CVSS) given by security experts for
different \underline{v}ulnerability \underline{a}ssessment
\underline{t}ypes (VATs) for CVE-2021-37714. Due to the above effects,
the availability impact for this~vulnerability is rated as {\em
  Partial} (i.e., for some inputs, there will be reduced performance
and interruptions in available services).

%Recent research~\cite{deepCVA-ase21} has developed a machine learning
%(ML) model that learns from the existing grading from security experts
%to provide the new grading for a committed code change that was deemed
%to be vulnerable. This type of commit-level automated vulnerability
%assessment together with a vulnerability detection (VD) tool are very
%useful in helping developers to early detect and assess the impacts of
%the detected vulnerability as soon as the code is committed.  However,
%the state-of-the-art approach for commit-level vulnerability
%assessment is still limited as explained in the following
%observations.

\begin{figure}
     \centering
     \begin{minipage}{0.45\textwidth}
\centering
\lstset{
		numbers=left,
		numberstyle= \tiny,
		keywordstyle= \color{blue!70},
		commentstyle= \color{red!50!green!50!blue!50},
		frame=shadowbox,
		rulesepcolor= \color{red!20!green!20!blue!20} ,
		xleftmargin=1.5em,xrightmargin=0em, aboveskip=1em,
		framexleftmargin=1.7em,
                numbersep= 5pt,
		language=Java,
    basicstyle=\tiny\ttfamily,
    numberstyle=\tiny\ttfamily,
    emphstyle=\bfseries,
                moredelim=**[is][\color{red}]{@}{@},
		escapeinside= {(*@}{@*)}
	}
	\begin{lstlisting}[]
// .../jsoup/parser/HtmlTreeBuilderState.java
boolean process(Token t, HtmlTreeBuilder tb) { ...
  if (t.isCharacter()&& inSorted( (*@{\color{red}{tb.currentElement().normalName()}@*),InTableFoster)){
     ...
     return tb.process(t);
  }
  ...
  } else {
      tb.popStackToClose(name);
(*@{\color{orange}{- \quad \quad tb.resetInsertionMode();}@*)
(*@{\color{orange}{- \quad \quad if (tb.state() == InTable) \{}@*)
(*@{\color{cyan}{+ \quad \quad if (!tb.resetInsertionMode()) \{}@*)
         tb.insert(startTag);
         return true;
      }
(*@{\color{red}{\quad \quad \quad return tb.process(t, InHead);}@*)
      ...
}
	\end{lstlisting}
         \caption{Code Change v1.12.1 for CVE-2021-37714}
         \label{fig:motiv-code}
     \end{minipage}
     \hfill
     \begin{minipage}{0.5\textwidth}
%          \centering
%\textbf{2. Vulnerability Type(s)}: Denial Of Service

%{\bf 3. CVSS Score:} ...\\

%{\bf 4. Detailed CVSS Grades:}\\
       %  \centering
       \begin{flushleft}
  \tabcolsep 3pt
  \scriptsize
  \begin{tabular}{lll}
   Vulner. Assess. Type   & Value & Description \\
      \hline
    Confidentiality Impact & {\bf None}  & No impact to the confidentiality \\
    Integrity Impact & {\bf None}  & No impact to the integrity \\
    Availability Impact & {\bf Partial} & There is reduced performance or\\
    & & interruptions in availability\\
    Access Complexity & {\bf Low} & Specialized access conditions or \\
    & & extenuating circumstances do not exist\\
    & & Little knowledge is required to exploit\\
    Authentication & {\bf Not Req} & Authentication is not required \\
    & & to exploit the vulnerability\\
    Gained Access & {\bf None}  & No gained access with the vulnerability \\
    Acccess Vector & {\bf Local} & The vulnerability is in the local parser \\
  \end{tabular}%
  \end{flushleft}
    \caption{Detailed CVSS Impact Grades}
         \label{cvss}
     \end{minipage}
\end{figure}

        
