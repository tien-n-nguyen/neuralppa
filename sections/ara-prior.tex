\subsubsection{Dependency Parsing \& Link Prediction}
In this proposal, we seek inspiration for our problem setting from Chen and Manning~\cite{chen-manning-2014-fast}, who first proposed a neural network-based approach to dependency parsing. The major benefits we envision to such a formulation include a significant speedup in dependency discovery and the extendibility of program dependence analysis to partial programs. However, employing a transition-based parsing technique~\cite{chen-manning-2014-fast} requires excessive feature engineering, while also assuming the projectivity of the dependency tree. This limits the applicability of such approaches to source code. In contrast, neural graph-based dependency parsing appoaches'~\cite{kiperwasser-goldberg-2016-simple, DBLP:conf/iclr/DozatM17} outputs can be non-projective. However, unlike with graph-based dependency parsing, our output is not just one of the possible valid trees (or the maximum spanning tree). Rather, program dependence graphs are directed and cyclic structures. Thus, we plan to design the dependence decoding stage by extending the link prediction~\cite{10.5555/3327345.3327423} task to all possible statement pairs, the combination of all of which can be formalized as the predicted CFG/PDG.


\subsubsection{Probabilistic Graphical Models}
In SE domain, the proposed research is loosely related to works that leverage probabilistic models to enhance the program dependence graph (PDG). Probabilistic PDG~\cite{baah-issta08-probabilistic} is an augmentation of the structural dependencies represented by a PDG with estimates of statistical dependencies between node states derived from test cases. Feng et al.~\cite{feng-paste10} propose Error-Flow Graph as a Bayesian Network, constructed from the dynamic dependence graphs of the runs. Bayesian Network-based Program Dependence Graph (BNPDG)~\cite{yu-jss17-bayesian} is capable of inferring the dependencies across non-adjacent nodes. MOAD (Modeling Observation-based Approximate Dependency)~\cite{lee-scam19-moad} reformulates program dependency as the likelihood that one program element is dependent on another, instead of a boolean relationship.  Lee~\cite{lee-icse20} proposes a scalable approximate program dependence analysis by estimating the likelihood of dependence. It uses lexical analysis~\cite{lee-jss20}, partial observations on executions, and the merging of static and observation-based approaches. Those approaches leverage the knowledge from the executions to enhance the PDG for complete code. In contrast, we aim to use neural networks for deriving dependencies for both partial and complete code.
