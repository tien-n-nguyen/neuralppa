Developers often use online question and answering (Q\&A) forums, e.g., StackOverflow (S/O), to learn how to use software libraries and frameworks. Sometimes, the answer to a question comes as a fragment/chunk of code, which later makes it to the production applications, stemming from the copy-and-paste software reuse practice. Unfortunately, if the copied code fragments are vulnerable, i.e., possess defects that can potentially be exploited, it will lead to the applications being prone to attacks. Verdi et al.~\cite{verdi-tse22} reviewed more than 72K C++ code snippets that migrated from 1,325 S/O answers. Of these, they reported a total of 99 vulnerable code snippets of 31 different types that made their way to 2,589 GitHub repositories. Thus, it is crucial to detect early the vulnerabilities in the code snippets from online forums.

Security researchers have proposed several automated approaches for vulnerability detection (VD) in software systems using program analysis (PA)~\cite{FlawFinder,RATS,viega2000its4,Checkmarx,HPFortify,Coverity,BufferOverFlow,SQLInj,Cross-siteScripting,AuthBypassSpoofing}, as well as machine and deep learning (ML \& DL)~\cite{fse21,chakraborty2020deep,zhou2019devign,li2018sysevr,li2018vuldeepecker} techniques. These approaches typically leverage program representations such as the control-flow graph (CFG)~\cite{fse21} and program dependence graph (PDG)~\cite{fse21} to succinctly represent all execution flows and the data-flow information in the program, so as to model/describe the vulnerable features. However, the PA tools employed to derive these program representations warrant the code to exist as complete program units, at the very least, at the method-level granularity. This makes it impossible to utilize such VD tools to find vulnerabilities in code snippets. A possible alternative would be to plug the code snippet into the method, resolve any ambiguities, and test it with a VD tool. However, such a strategy is limited. First, if found vulnerable, the efforts of integrating the code snippet into the existing method would be lost. Second, due to the black-box nature of DL models, we would not know the origin of the vulnerability, i.e., whether it arises due to the flawed code snippet or the existing part of the code.

Besides, analyzing code snippets is not straightforward as they are often incomplete, un-parseable, contain declaration/reference ambiguity, and are interspersed between user comments. Currently, there exist tools such as PPA~\cite{ppa08},~which parse an incomplete code fragment to build the AST and~ex\-tract data types in a best-effort manner, while StatType \cite{icse18} resolves the libraries and recovers only the fully-qualified names for references. However, deriving the program dependencies on incomplete code snippets is not yet possible. Let us call such an analysis, {\em partial program dependence analysis}.

In addition to vulnerability detection, such partial program dependence analysis is also beneficial to the other software engineering (SE) tasks that can tolerate a low level of errors and imprecision in building the dependencies. For example, consider code completion~\cite{codefill-icse22,facebook-icse21}, in which a model provides suggestions to complete partial code. Existing ML/DL-based code completion models are just based on the code sequences or utilize the syntactic structure in ASTs, but none leverage the program dependencies due to the nature of partial code. Next, consider the task of analyzing the code fragments in a bug report to connect it to the relevant source files for bug localization purposes~\cite{euler-fse19,icpc17}. Here too, a need for partial program dependence analysis can be observed.
