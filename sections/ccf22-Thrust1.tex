

%\section{Thrust 1. Ongoing Evaluation of Existing Code Representation Learnings in Bug Fixing Processes}

%\section{\Tone}
%\label{thrust1:sec}

%\subsection{Challenge 1. Ongoing Empirical Study on which statistical NLP models and their characteristics that fit with source code and API code synthesis}

%\subsection{Challenge 1. Mine a large-scale corpus of texts and API usage code}
\vspace{-5pt}
\subsection{Empirical Evaluation on Code Representation Learning for Bug Detect-Fix Approaches}

\label{prelim-study:sec}

Our goal was to show that the bug detect-fix approaches using the
general-purpose CRLs do not perform well and there is no silver
bullet CRL for all the approaches. We collected a large-scale dataset
of bugs and fixes in Java~\cite{yioopsla19} with $\approx$5 million
Java methods and their bug fixes by crawling on-line GitHub
repositories.


%The first task is to build large scale datasets with bug-fixes in
%different program languages: Java, Python, JavaScripts.  We built a
%Java corpus~\cite{yioopsla19} with $\approx$5 million Java methods and
%their bug fixes by crawling on-line resources, e.g., Github,
%collecting source code and commits from the Github repository, and bug
%reports from the issue tracking systems. Specifically, (1) We
%downloaded all bug reports marked as resolved or closed and bugs from
%issue tracking systems. (2) We used the same approach as the ones used
%in prior studies, e.g.,~\cite{mockus-2000, Ray-2014, Ray-2016}, to
%process each commit message and mark it as a bug-fix if the message
%contains a bug id and at least one of the error related keywords:
%{error, bug, fix, issue, mistake, incorrect, fault, defect, flaw and
%  type}. (3) We downloaded source code as a clean version and use
%the bug-fix commits to recover the buggy version from the source code,
%as a code commit records all of the changes.
%Following the above steps, we will continue to collect more bug fixes
%from Java, Python and JavaScripts projects in on-line resources.


%
\iffalse

\begin{table}[h]
	\caption{Statistics on Collected Java Dataset}
	%	\vspace{-15pt}
	{\footnotesize
	\begin{center}
		\renewcommand{\arraystretch}{1}
		\begin{tabular}{p{3cm}<{\centering}p{3cm}<{\centering}p{2cm}<{\centering}p{3cm}<{\centering}p{4cm}<{\centering}}
			%\hline
			\hline
			\textbf{Project Name} & \textbf{ \# of Versions} & \textbf{\# of Files} & \textbf{\# of Methods} & \textbf{\# of Buggy Methods}\\ \toprule
			\hline
			pig & 9 & 8k & 95k & 21k\\\hline
			avro & 7 & 2k & 30k & 1k\\\hline
			lucene-solr & 14 & 93k & 1.032M & 518k\\\hline
			hbase & 9 & 14k & 318k & 258k\\\hline
			flink & 14 & 49k & 419k & 173k\\\hline
			hive & 18 & 53k & 981K & 411k\\\hline
			cloudstack & 11 & 55k & 766k & 307k\\\hline
			camel & 10 & 172k & 1.332M & 135k\\
			\hline
			Total& 92 & 402K & 4.973M & 1.824M\\ \bottomrule
			\hline
		\end{tabular}
		\label{javadata}
	\end{center}
}
		\vspace{-25pt}
\end{table}

\fi


%\subsection{T1-Task 2. Ongoing Evaluation of Code Representation Learning (CRL) for Bug Detection}

%This task is to provide comprehensive and systematic empirical knowledge
%on pros and cons of existing CRL approaches in bug detection.

We first compared different state-of-the-art CRL approaches learning
code vectors from various code representations in the context of bug
detection: \textbf{DeepSim} (DS)~\cite{Zhao-2018} used feature values
to encode control and data flows into a semantic matrix and a
feed-forward neural network for learning for code functional
similarity.  \textbf{Code2Vec} (C2V)~\cite{Alon-2018} built
vocabulary-based embeddings for frequent paths on the ASTs and applied
a fully connected layer (in \cnn) with an attention for general
CRL. \textbf{Code Vectors} (CV)~\cite{Henkel-2018} encoded
abstractions of traces from symbolic execution as words and sentences
using word2vec~\cite{word2vec}.  \textbf{Deep Learning Similarity}
(DLS)~\cite{Tufano-2018} used Recursive Autoencoders on Identifiers
and ASTs, a graph Embedding (HOPE~ \cite{goyal2018graph}) on CFGs and
adopt ensemble learning techniques: Algebraic combiners and Random
Forest, to detect code clones. \textbf{Tree LSTM} (TL)~\cite{Tai-2015}
trained a tree-structured LSTM model with the AST of the source code
of a method.  \textbf{Graph-based Generative Modeling
  (GGM)}~\cite{brockschmidt2018generative} generated code by
interleaving grammar-driven expansion steps with graph augmentation
and neural message passing steps.


\begin{wraptable}{r}{6.8cm}
		\vspace{-20pt}
	%\caption{RQ2. Comparison with the Baselines in the Cross-Project Setting.}
	{\footnotesize	
	\caption{Comparison of CRL on Bug Detection.}
		\vspace{-10pt}
	\begin{center}
	
		%	\vspace{-10pt}
		\renewcommand{\arraystretch}{1}
		\begin{tabular}{p{1.2cm}<{\centering}|p{0.8cm}<{\centering}|p{0.8cm}<{\centering}|p{0.8cm}<{\centering}|p{0.8cm}<{\centering}|p{0.8cm}<{\centering}|p{1cm}<{\centering}}
			%\hline
			\hline
			\textbf{Category} & \textbf{DS} & \textbf{DLS} & \textbf{C2V} & \textbf{TL} & \textbf{CV} & \textbf{GGM}\\
			\hline
			Recall & 0.67 & 0.71 & 0.69 & 0.82 & 0.70&0.73\\
			Precision & 0.19 & 0.24 & 0.17 & 0.09 & 0.15&0.30\\
			F-score & 0.30 & 0.36 & 0.27 & 0.16 & 0.25&0.43\\
			FP Rate & 0.35 & 0.36 & 0.43 & 0.69 & 0.45 &0.28\\
			\hline
		\end{tabular}
		\label{RQ2_cross}

	\end{center}
}
	\vspace{-20pt}
%\end{table}
\end{wraptable}

%As those CRL approaches are not aimed for bug detection, to be fair,

Those approaches focused on the detection models, however, used
off-the-shelf CRL to learn vectors to represent the code.  For each
approach, we used a customized CRL~\cite{yioopsla19} to learn vectors
to represent code snippets (incl. buggy and non-buggy).  We applied
the same basic \cnn on the learned code vectors to build a model to
classify a given code snippet into buggy or non-buggy.
Table~\ref{RQ2_cross} shows that with proper CRL, a model achieved
superior performance (Details are in our OOSLA'19
paper~\cite{yioopsla19}). The generic, off-the-shelf CRLs (word2vec,
code2vec) work on other SE tasks, but they are not ideal for bug
detection. For example, Code2Vec obtained a high false positive rate
of 43\% and low precision of 17\%.  In conclusion, {\em no code
  representations and learning models are silver bullet for every
  task, and customized CRLs are needed to improve accuracy}.


%\subsection{T1-Task 3. Plan-to-Do: Evaluation of Code Representation Learning on FL, RT-CI, and APR}

%This task is to provide comprehensive empirical knowledge of existing
%CRL approaches on Fault Localization (FL), Regression Testing in CI
%(RT-CI), and Automated Program Repair (APR). In FL, one very recent
%deep learning based FL work, DeepFL~\cite{DeepFL}, was proposed to
%improve method-level FL.  DeepFL collects over 200 features from
%different aspects of information of a method, such as code metrics,
%spectrum and mutation based testing, into one vector containing
%feature values, then apply a Multilayer Perceptrons Model (\MLP) on
%the method vectors to locate faulty methods.  However, DeepFL does not
%conduct representation learning on code or test coverage information.
%In RT-CI, the most state-of-the-art
%approach~\cite{bertolino2020learning} utilizes learning-to-rank
%algorithms consuming code and history test metrics. However, it still
%lacks the extraction of deep features on code under test, test cases,
%and their relations.  In APR, the state-of-the-art approaches are
%pattern-based approaches that mine and learn fixing patterns from
%prior bug fixes. There are several deep learning (DL) based APR, e.g.,
%\Ratchet, \CODIT, \Tufano and \CoCo. We compared existing approaches
%on a benchmark dataset, Defects4J~\cite{defects4j}. Our preliminary
%results show that (1) The best DL APR with fault localization can only
%correctly fix 33 out of 395 bugs. (2) Recent pattern-based approaches
%outperform any DL APR approaches significantly, as all existing DL APR
%approaches model code as a sequence of tokens or mine rules from ASTs,
%and directly utilize the neural translation models to generate
%fixes. Obviously, modeling code as a sequence of tokens is not ideal
%solution in APR. Additionally, the existing DL-based APR tools can
%mainly fix one-statement and one-hunk bugs, and rarely fix
%multi-statement and multi-hunk bugs.

%\begin{wraptable}{r}{0.75\textwidth}
%	\vspace{-35pt}
%	\begin{center}
%		\footnotesize{
%			\caption{Comparison with Pattern and DL APR Baselines on Defects4J.}\label{thust1APR}
%			\begin{tabular}{p{1.1cm}<{\centering}|p{1.25cm}<{\centering}|p{1.3cm}<{\centering}|
%					p{1.2cm}<{\centering}|p{1cm}<{\centering}|p{0.7cm}<{\centering}|p{1.2cm}<{\centering}|p{1.2cm}<{\centering}|p{1.2cm}<{\centering}|p{1.3cm}<{\centering}}
%				\hline
%				\textbf{}&\textbf{ELIXIR}&\textbf{FixMiner}&\textbf{AVATAR}&\textbf{SimFix}&\textbf{TBar}&\textbf{Ratchet}&\textbf{CODIT}&\textbf{Tufano}&\textbf{CoCoNut}\\
%				\hline
%				
%				\# Fix&  26  & 25 & 27  & 34 & 43 &2 & 9 & 16 &33\\
%				\hline
%			\end{tabular}
%		}
%	\end{center}
%	\vspace{-22pt}
%\end{wraptable}


%There is a clear need for comprehensive and systemic knowledge on CRL in FL, RT-CI and APR. We replicate the analysis procedure of Task 2 in FL, RT-CI and APR. Specifically, we plan to apply existing code representations and their learning models in CRL from other software engineering tasks to model FL, RT-CI, and APR. Furthermore, we conduct more ablation studies to study the effects of various code representations and their models in FL, RT-CI, and APR.

\subsection{Vector Locality and Vector Offsetting for API Elements}
\label{embedding-study:sec}

\begin{wrapfigure}{l}{0.45 \textwidth}
%\vspace{2mm}
\scriptsize
\begin{lstlisting}[basicstyle=\scriptsize\sffamily, stepnumber=1, numbers=left, numbersep=-8pt, framexleftmargin=-2mm,
     framexrightmargin=-2mm, emph ={HashMap,FileWriter,String,Integer,for,new,put,keySet,append,get,close}]
   HashMap dict = new HashMap();
   dict.put("A", 1);
   FileWriter writer = new FileWriter("Vocabulary.txt");
   for (String vocab: dict.keySet())
      writer.append(vocab + " " + dict.get(vocab)+"\r\n");
   writer.close();
\end{lstlisting}
\vspace{-0.12in}
\caption{An API Usage in Java JDK}
\label{usageexample}
\vspace{-10pt}
\end{wrapfigure}

To gain the knowledge on the vector representations built from the
standard Word2Vec model on source code, we conducted an empirical
study on the embeddings of the API elements. First, we colllected a
dataset of 14,807 Java projects (2.1M classes, 7M methods, 352M LOCs,
123K unique APIs), and a dataset of 7,724 C\# projects (900K classes,
2.3M methods, 292M LOCs, 130K unique APIs). For each method in a
project, we built a tool to parse and {\em collect the sequence of
  only API method calls, field accesses, and classes' names}.
For example, in the code in Figure~\ref{usageexample}, we produce
\code{HashMap}\#var \code{HashMap}.new,  
%
\code{String}\#ret \code{HashMap}\#rec \code{HashMap.put} \code{String}\#arg \code{Integer}\#arg, 
%
\code{FileWriter}\#var \code{FileWriter.new} \code{String}\#arg,
%
\code{for} \code{String}\#var \code{String[]}\#ret \code{HashMap}\#rec \code{HashMap.keySet},
%
\code{String}\#ret \code{HashMap}\#rec \code{HashMap.get} \code{String}\#arg \code{FileWriter}\#rec \code{FileWriter.append} \code{String}\#arg,
%
\code{FileWriter}\#rec \code{FileWriter.close}. Such sequences for all
methods in a dataset is used to train CBOW. We use the context
window's size of 10. After training, the output of the hidden layer
gives the vector for the current API, which is called {\em API
  embedding}.

\subsubsection*{Nearby Vectors Represent APIs with Similar Contexts and Vice Versa?}

%{\bf Nearby Vectors Represent APIs with Similar Contexts and Vice Versa?}
We first randomly selected 1,000 JDK API methods and fields in our
dataset. For each API, we computed the top-5 API method calls and
field accesses that are closest to that~API in the vector space.  We
processed those 1,000 groups of 6 API methods/fields (one main API of
the group and top-5 closest ones) to verify if each of those 5
elements shares similar usage contexts with the main API. For such
verification, we wrote a program to take two APIs $a$ and $b$ and
search through our Java dataset to compute two sets $A$ and $B$ of API
elements that have been frequently occurred with $a$ and $b$,
respectively (80\% threshold), in the methods in the dataset. If $A$
and $B$ overlaps more than 80\%, we consider $a$ and $b$ share similar
surrounding APIs.

Among 5,000 pairs of APIs (1,000 groups and 5 comparisons each), we
found that 4,632 pairs (92.64\% of them)~have similar surrounding APIs
For the other 7.36\%, this is because an API has multiple contexts and
some contexts with infrequently used APIs were not captured with
insufficient~data. In brief, in a vector space for the APIs,
nearby vectors represent the APIs that have similar usage contexts and
vice versa.

Our results also showed that the APIs in the same class/package
perform functions relevant to the class/package's theme, and often
share similar surrounding APIs. They tend to have nearby vectors.

\subsubsection*{Similar Vector Offsets Reflect Similar Relations?}
%{\bf Similar Vector Offsets Reflect Similar Relations?}
\label{vector-offset-section}
%This experiment focuses on studying {\em whether similar usage
%  relations among two APIs can be reflected with similar vector
%  offsets}.
We first mined the frequent pairs of APIs by collecting~all the
pairs of the APIs in the same methods in our Java~dataset. We ranked
the pairs by their occurrence frequencies.
%used an API usage pattern mining tool, GrouMiner~\cite{fse09}, to
%detect the frequent pairs of API elements in our Java code corpus.
We~then manually checked the most frequent pairs
and collected 120 of them, which are placed into 14 groups of
pairs representing 14 different relations. Similarly, we collected a
set of 138 correct pairs of C\texttt{\#} APIs placed into 16
groups.~We used those two sets of pairs in JDK and .NET as our
benchmarks.

%\begin{table}[t]
%	\centering
%	\small
%	\caption{Relations via Vector Offsets in JDK}
%	\label{offsetJava}
%	\begin{tabular}{lll}
%		\hline
%		Java API                               & C\# API                                & Rank       \\ \hline
%		\multicolumn{3}{l}{\textbf{R.1 Check the existence of the current element before retrieval}} \\ \hline
%		\textit{Iterator.hasNext}              & \textit{Iterator.next}                 &            \\
%		ListIterator.hasNext                   & ListIterator.next                      & 8          \\
%		Enumeration.hasMoreElements            & Enumeration.nextElement                & 1          \\
%		StringTokenizer.hasMoreTokens          & StringTokenizer.nextToken              & 3          \\
%		NamingEnumeration.hasMore              & NamingEnumeration.next                 & 1          \\
%		XMLStreamReader.isEndElement           & XMLStreamReader.next                   & 1          \\
%		XMLStreamReader.hasNext                & XMLStreamReader.nextTag                & 1          \\
%		Scanner.hasNext                        & Scanner.next                           & 3          \\
%		HashMap.containsKey                    & HashMap.get                            & 1          \\
%		Stack.isEmpty                          & Stack.pop                              & 8          \\
%		Queue.isEmpty                          & Queue.poll                             & 9          \\ \hline
%		\multicolumn{3}{l}{\textbf{R2. Check size before get element from a collection}}             \\ \hline
%		\textit{List.size}                     & \textit{List.get}                      &            \\
%		ArrayList.size                         & ArrayList.get                          & 1          \\
%		Vector.size                            & Vector.get                             & 2          \\
%		Map.size                               & Map.get                                & 1          \\
%		LinkedList.size                        & LinkedList.get                         & 1          \\
%		Dictionary.size                        & Dictionary.get                         & 5          \\
%		Hashtable.size                         & Hashtable.get                          & 1          \\
%		TreeMap.size                           & TreeMap.get                            & 13         \\ \hline
%		\multicolumn{3}{l}{\textbf{R3. Write to and flush a stream}}                                 \\ \hline
%		\textit{FileOutputStream.write}        & \textit{FileOutputStream.flush}        &            \\
%		DataOutputStream.write                 & DataOutputStream.flush                 & 1          \\
%		OutputStream.write                     & OutputStream.flush                     & 1          \\
%		PrintWriter.write                      & PrintWriter.flush                      & 1          \\
%		PrintStream.write                      & PrintStream.flush                      & 1          \\
%		BufferedOutputStream.write             & FileOutputStream.flush                 & 1          \\
%		BufferedWriter.write                   & BufferedWriter.flush                   & 2          \\
%		ByteArrayOutputStream.write            & FileOutputStream.flush                 & 5          \\
%		CharArrayWriter.write                  & CharArrayWriter.flush                  & 6          \\
%		FileWriter.write                       & FileWriter.flush                       & 2          \\
%		OutputStreamWriter.write               & OutputStreamWriter.flush               & 2          \\
%		PrintStream.write                      & PrintStream.flush                      & 1          \\
%		StringWriter.write                     & StringWriter.flush                     & 1          \\
%		java.io.Writer.write                   & java.io.Writer.flush                   & 4          \\ \hline
%		\multicolumn{3}{l}{\textbf{R4. Avoid adding duplicate element to the collection}}            \\ \hline
%		\textit{List.contains}                 & \textit{List.add}                      &            \\
%		Set.contains                           & Set.add                                & 1          \\
%		Vector.contains                        & Vector.add                             & 1          \\
%		ArrayList.contains                     & ArrayList.add                          & 1          \\
%		Map.containsKey                        & Map.put                                & 3          \\
%		LinkedList.contains                    & LinkedList.add                         & 1          \\
%		Hashtable.containsKey                  & Hashtable.put                          & 3          \\
%		java.util.TreeMap.containsKey          & java.util.TreeMap.put                  & 1          \\ \hline
%	\end{tabular}
%\end{table}


% Table generated by Excel2LaTeX from sheet 'Copy'
\begin{wraptable}{l}{9.8cm}
	\vspace{-10pt}
  \centering
  \small
  \tabcolsep 1pt
  \caption{Example Relations via Vector Offsets in JDK}
%  \vspace{-0.04in}
    \begin{tabular}{lll}
    \toprule
    \multicolumn{2}{l}{{\bf {\em R1. Check the current element before retrieval}}}  & Rank\\
    ListIterator.hasNext & ListIterator.next & 1 \\
%    XMLStreamReader.hasNext & XMLStreamReader.nextTag & 1 \\
    Enumeration.hasMoreElements & Enumeration.nextElement & 1 \\
%    NamingEnumeration.hasMore & NamingEnumeration.next & 1 \\
    StringTokenizer.hasMoreTokens & StringTokenizer.nextToken & 3 \\
    XMLStreamReader.isEndElement & XMLStreamReader.next & 1 \\
%    HashMap.containsKey & HashMap.get & 1\\
%    Stack.isEmpty & Stack.pop & 8\\
    \midrule
%    \multicolumn{2}{l}{{\bf {\em check the size of a list before getting an element}}} & \\
%    java.util.ArrayList.size & java.util.ArrayList.get &  1\\
%    java.util.Vector.size & java.util.Vector.get &  1\\
%    java.util.Map.size & java.util.Map.get &  1\\
%    java.util.LinkedList.size & java.util.LinkedList.get &  1\\
%    \midrule
%    \textit{\textbf{open a stream needs a flush}} &       &  \\
%    java.io.FileOutputStream\#var & java.io.FileOutputStream.flush &  1\\
%    java.io.PrintWriter\#var & java.io.PrintWriter.flush &  1\\
%    java.io.PrintStream\#var & java.io.PrintStream.flush &  1\\
%    java.io.ObjectOutputStream\#var & java.io.ObjectOutputStream.flush &  1\\
%    java.io.OutputStreamWriter\#var & java.io.OutputStreamWriter.flush &  1\\
%    java.io.BufferedWriter\#var & java.io.BufferedWriter.flush &  \\
%    \midrule
%    \textit{\textbf{open a stream for reading}} &       & \\
%    java.io.DataInputStream\#var & java.io.DataInputStream.readShort &  1\\
%    java.io.DataInput\#var & java.io.DataInput.readShort &  1\\
%    java.io.DataInputStream\#var & java.io.DataInputStream.readByte &  1\\
%    java.io.DataInput\#var & java.io.DataInput.readByte &  1\\
%    \midrule
%    \textit{\textbf{open a stream for writing}} &       &  \\
%    DataOutputStream\#var & DataOutputStream.writeShort &  1\\
%    DataOutput\#var & DataOutput.writeShort &  1\\
%    \midrule
    \multicolumn{2}{l}{{\bf {\em R2. Obtain property after creating system/stream}}} & \\
    System\#var & System.getProperty &  1\\
    Properties\#var & Properties.getProperty &  1\\
    XMLStreamReader\#var & XML...Reader.getAttr...Value &  1\\
    \midrule
%    \textit{\textbf{get source of an event}} &       &  \\
%    java.awt.event.ActionEvent\#var & java.awt.event.ActionEvent.getSource &  1\\
%    java.awt.event.MouseEvent\#var & java.awt.event.MouseEvent.getSource &  1\\
%    javax...event.ChangeEvent\#var & javax...event.ChangeEvent.getSource &  1\\
%    java.awt.event.KeyEvent\#var & java.awt.event.KeyEvent.getSource &  1\\
%    java...Property...Event\#var & java...Property...Event.getSource &  1\\
%    \midrule
%    \textit{\textbf{append data to the end}} &       &  \\
%    \textit{\textbf{of strings or writers}} &       &  1\\
%    java.lang.StringBuilder\#var & java.lang.StringBuilder.append &  1\\
%    java.lang.StringBuffer\#var & java.lang.StringBuffer.append &  1\\
%    java.lang.Appendable\#var & java.lang.Appendable.append &  1\\
%    java.io.PrintWriter\#var & java.io.PrintWriter.append &  1\\
%    \midrule
    \multicolumn{2}{l}{{\bf {\em R3. Add an element to various types of collections}}} & \\
    List\#var & List.add &  1\\
    Map\#var & Map.put &  1\\
%    java.util.Vector\#var & java.util.Vector.add &  1\\
    Hashtable\#var & Hashtable.put &  1\\
%    java.util.TreeMap\#var & java.util.TreeMap.put &  1\\
    Dictionary\#var & Dictionary.put & 1 \\
%    \midrule
%    \multicolumn{2}{l}{{\bf {\em R4. Parse a string into different types of numbers}}} & \\
%    Float\#var & Float.parseFloat &  1\\
%    Double\#var & Double.parseDouble &  1\\
%    Integer\#var & Integer.parseInt &  1\\
%    Long\#var & Long.parseLong &  1\\
    \midrule
     \multicolumn{2}{l}{{\bf {\em R4. Avoid adding duplicate element to a collection}}} & \\
     Set.contains & Set.add &  1\\
     Map.containsKey & Map.put &  3\\
     LinkedList.contains & LinkedList.add &  1\\
     Hashtable.containsKey & Hashtable.put & 3\\
%    java.lang.String\#var & java.lang.String.contains &  1\\
%-----------------------------------------------------
%    \midrule
%    \multicolumn{2}{l}{{\bf {\em R6. check the size of a list before getting an element}}} & \\
%    \hline
%    \textit{\textbf{R7 open a stream needs a flush}} &       &  \\
%    \hline
%    \textit{\textbf{R8. open a stream for reading}} &       & \\
%    \hline
%    \textit{\textbf{R9. open a stream for writing}} &       &  \\
%    \hline
%    \textit{\textbf{R10. get source of an event}} &       &  \\
%    \hline
%    \multicolumn{2}{l}{{\bf {\em R11. append data to the end of strings or writers}}} & \\
%     \hline
%    \multicolumn{2}{l}{{\bf {\em R12. check if a collection contains an element to retrieve}}} & \\
%    \hline
%    {\bf \em {R13. get the name of various objects}} & & \\
%    \hline
%    {\bf \em {R14. get the strings for various types}} & & \\
     \bottomrule
    \end{tabular}%
  \label{offsetJava}%
\end{wraptable}%










We processed the pairs as follows. For each group of pairs of APIs
(representing a relation), we randomly picked a seed pair, e.g.,
(\code{List\#var}, \code{List.add}). For each of the other pairs in
the group, e.g., (\code{Map\#var}, \code{Map.put}), we applied the
vector offset from the seed pair to the vector of the first API of the
current pair to compute the resulting vector, e.g., \textit{X =
  V(\code{List.add}) $-$ V(\code{List\#var}) $+$
  V(\code{Map\#var})}. We then searched for the vectors that are
closest to $X$ (e.g., \code{Map.put}) and considered them as the
candidates (ranked by their respective cosine distances). If the
second API of the current pair in the benchmark is in the top-$k$ of
the candidate list, we count it as a hit. Accuracy is the ratio
between the number of hits over the total number of cases.
%,otherwise, it is a~miss.

%We considered the rank of the correct API in the candidate list as
%well.

There are 94.2\% of the correct APIs in those relations~show\-ing up in the
top-5 candidate lists. 74.1\% are actually at the top
one. Table~\ref{offsetJava} shows examples of 5 groups of relations in
our oracle for JDK APIs and the ranks of the correct APIs~in the
candidate lists. As seen, 
%with simple vector computation, 
the model can capture similar relations between APIs and rank highly
the correct APIs, even when the respective names are different. For
example, in the relation {\em ``add an element to various types of
  collections''}, as using \code{List}, one uses \code{List.add},
however, \code{Map.put} is used for \code{Map}.
%when using
%\code{Map}, one must use \code{Map.put}.


We were also able to observe/interpret the same relations in
C\texttt{\#}: 1) {\em ``Check size before removal''}, e.g.,
  \code{Dictionary.Count} -- \code{Dictionary.Remove}; 2) {\em ``Add
    an element to a collection''}, e.g., \code{Hashtable.new} --
  \code{Hashtable.Add}; 3) {\em ``Read a file with different types''},
  e.g., \code{BinaryReader.ReadInt64} -- \code{System.Int64}; 4) {\em
    ``Check the current element before retrieval''}, e.g.,
  \code{IEnumerator.MoveNext} -- \code{IEnumerator.Current}, etc.


%We also build a tool to derive pairs of API elements with the
%same/similar relations. The tool takes as input a pair of API elements
%in the same class, e.g., \code{List\#var} and \code{List.add}, and
%another API in a different class, e.g., \code{Hashtable\#var}. It then
%uses the vector offsetting operation to derive the corresponding API
%\code{Hashtable.put}, without understanding the meaning of the
%relation. One could use this tool to derive that \code{List.add} could
%be used to achieve the similar functionality as \code{Hashtable.put}. 
%This is useful for developers who are new to the APIs.


