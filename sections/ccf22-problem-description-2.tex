%\section{Introduction}
Integrating the utilization of program analysis (PA) tools early in the development process of modern, large-scale software systems is crucial in determining the weaknesses/vulnerabilities in the code. Consider, for instance, a scenario in which a developer wants to use an online question and answering (Q\&A) forum, e.g., StackOverflow (S/O), to learn how to use software libraries or frameworks. Typically, the answer to a posed question comes as a fragment/chunk of code, which later makes it to the production application, stemming from the copy-and-paste software reuse practice. Unfortunately, if the copied code fragment is vulnerable, i.e., possesses defects that one can potentially exploit, it will result in the application being prone to attacks. Verdi {\em et al.}~\cite{verdi-tse22} reviewed more than 72K C++ code snippets that migrated from 1,325 S/O answers, reporting a total of 99 vulnerable code snippets of 31 different types that made their way to 2,589 GitHub repositories.

Security researchers have proposed several automated approaches for vulnerability detection (VD) in software systems using program analysis (PA)~\cite{FlawFinder,RATS,viega2000its4,Checkmarx,HPFortify,Coverity}, as well as machine learning and deep learning (ML \& DL) \cite{fse21,chakraborty2020deep,zhou2019devign,li2018sysevr,li2018vuldeepecker} techniques. These approaches typically leverage program representations such as abstract syntax tree (AST), program dependence graph (PDG)~\cite{fse21,li2018vuldeepecker}, control-flow graph (CFG)~\cite{zhou2019devign}, data-flow graph (DFG)~\cite{zhou2019devign}, code property graph (CPG)~\cite{chakraborty2020deep}, etc., to model the vulnerable features. 
%However, extending such analyses to code snippets is not straightforward as they are often incomplete, unparseable, contain declaration/reference ambiguity, and may be interspersed between user comments. Currently, there exist tools such as PPA~\cite{ppa08}, which parse an incomplete code fragment to build the AST and extract data types in a best-effort manner, while StaType \cite{icse18} resolves the libraries and recovers only the fully-qualified names for references. However, the basic infrastructure for partial program analysis, i.e., for analyzing incomplete code is not yet available. 
%Such an infrastructure must include fundamental supports/services at the structural, semantic, and execution levels, thus enabling the static and dynamic analysis techniques to be built upon. 
%Let us refer to this as \textit{partial program analysis infrastructure}.

While deep learning (DL) and advanced machine learning (ML) with large
language models (LLMs) have been achieving remarkable successes in the
area of ML for code, the application of ML/DL and LLMs in
vulnerability detection for incomplete code snippets. The program
analysis (PA) tools employed to derive these program representations
warrant the code to exist as complete program units, at the very
least, at the method-level granularity. As a result, it is impossible
to utilize such VD tools to find vulnerabilities in code snippets. A
possible alternative would be to plug the code snippet into the
method, resolve any ambiguities, and test it with a VD tool. However,
such a strategy is limited. First, if a vulnerability is found, the
efforts of integrating the code snippet into the existing method would
be lost. Second, due to the black-box nature of DL models, we would
not know the origin of the vulnerability, i.e., whether it arises due
to the flawed code snippet or the existing part of the code.

Besides, analyzing code snippets is not straightforward as they are often incomplete, un-parseable, contain declaration/reference ambiguity, and may be interspersed between user comments. Currently, there only exist tools such as PPA~\cite{ppa08}, which can parse an incomplete code fragment to build its AST and extract the declared data types in a best-effort manner; and StatType \cite{icse18}, which can resolve the libraries and recover the fully-qualified names for API references. However, the basic infrastructure for partial program analysis, i.e., for analyzing incomplete code is not yet available. Such an infrastructure must include fundamental supports/services at the structural and semantic levels, enabling traditional program analysis techniques to be extended to incomplete code. We refer to this as \textit{partial program analysis infrastructure}.


Currently, performing program analysis for a specific code snippet
necessitates the access to the entire program containing the
snippet. The need for partial program analysis arises in situations
where only partial code is accessible, such as in a vulnerable code
snippet in an online forum, a commit log or code diff, or when partial
code is sent to a server due to security-related concerns. Without
partial program analysis, the tasks including {\em early vulnerability
detection or assessment would be very limited and even impossible}.

%In addition to vulnerability detection, such an infrastructure is also
%beneficial to other software engineering (SE) tasks that can tolerate
%a low level of errors and imprecision in building program
%representations. For example, consider code
%completion~\cite{codefill-icse22,facebook-icse21}, in which a model
%provides suggestions to complete partial code. Existing
%state-of-the-art ML/DL-based code completion models are just based on
%the code sequences or utilize the syntactic structure in ASTs, but
%none leverages the program dependencies due to the nature of partial
%code. Next, consider the task of analyzing the code fragments in a bug
%report to connect it to the relevant source files for bug localization
%purposes~\cite{euler-fse19,icpc17}. Here too, a need for partial
%program analysis, specifically partial program dependence analysis,
%can be observed.

%The other facets of program analysis tools used in vulnerability
%detection that need attention are their soundness and
%completeness. For example, static analysis tools are designed to
%detect errors that are valid for all possible executions, which often
%come at the cost of multiple approximations. As a result, such tools
%tend to overestimate, resulting in false positives. Next, consider the
%task of symbolic execution, which is limited by the path explosion
%problem that hinders its applicability to large-scale software
%systems. The effectiveness of such a symbolic execution engine can be
%improved by guiding it to explore the right subset of symbolic states.

