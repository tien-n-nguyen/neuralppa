%\subsection{Problem Description}


Developers often use online question and answering (Q\&A) forums,
e.g., StackOverflow (S/O), to learn how to use software libraries and
frameworks. Sometimes, the answer to a question comes as a
fragment/chunk of code, which later makes it to the production
applications, stemming from the copy-and-paste software reuse
practice. Unfortunately, if the copied code fragments are vulnerable,
i.e., possess defects that can potentially be exploited, it will lead
to the applications being prone to attacks. Verdi {\em et
  al.}~\cite{verdi-tse22} reviewed more than 72K C++ code snippets
that migrated from 1,325 S/O answers. Of these, they reported a total
of 99 vulnerable code snippets of 31 different types that made their
way to 2,589 GitHub repositories. Thus, it is crucial to detect early
the vulnerabilities in the code snippets from online forums.
%Running a vulnerability detection tool on the source code after the
%integration of a S/O code snippet into the current codebase would
%waste developers' effort for such integration.

Security researchers have proposed several automated approaches for
vulnerability detection (VD) using program
analysis~\cite{FlawFinder,RATS,viega2000its4,Checkmarx,HPFortify,Coverity,BufferOverFlow,SQLInj,Cross-siteScripting,AuthBypassSpoofing},
as well as machine learning (ML) including deep learning
(DL)~\cite{fse21,chakraborty2020deep,zhou2019devign,li2018sysevr,li2018vuldeepecker}
techniques. However, these approaches warrant the code to exist as
complete program units, often making use of program representations
such~as abstract syntax tree (AST), Program Dependence Graph
(PDG)~\cite{fse21,li2018vuldeepecker}, Control Flow Graph
(CFG)~\cite{zhou2019devign}, Data Flow Graph
(DFG)~\cite{zhou2019devign}, Code Property Graph
(CPG)~\cite{chakraborty2020deep}, etc. At a minimum, they operate at
the method-level granularity, making it impossible to utilize them for
directly detecting vulnerabilities in code snippets. A possible
alternative would be to plug the code snippet into the method, resolve
any ambiguities, and test it with a VD tool. However, such a strategy
is limited. First, if found vulnerable, the efforts of integrating the
code snippet into the existing method would be lost. Second, due to
the black-box
%(inexplicable?)
nature of DL models, we would not know the origin of the
vulnerability, i.e., whether it arises due to the flawed code snippet
or the existing part of the code.

%other statements in the method.

%Besides, even a commit-level VD model requires the code before/after changes to be syntactically valid to extract those features.

Importantly, analyzing code snippets is not straightforward as they
are often incomplete, un-parseable, contain declaration/reference
ambiguity, and are interspersed between user comments. Currently,
there exist tools such as PPA~\cite{ppa08},~which parse an incomplete
code fragment to build the AST and~ex\-tract data types in a
best-effort manner, while StaType \cite{icse18} resolves the libraries
and recovers only the fully-qualified names for references. However,
the basic infrastructure for partial program analysis on incomplete
code snippets is not yet possible. The infrastructure includes the
fundamental supports/services such as syntactic, structural, and
semantic analysis so that the static and dynamic analysis techniques
could be built upon. Let us call such an infrastructure, {\em partial
program analysis infrastructure}.

In addition to vulnerability detection, such partial program analysis
infrastructure is also beneficial to the other software engineering
(SE) tasks that can tolerate a low level of errors and imprecision in
building the program representations. For example, consider code
completion~\cite{codefill-icse22,facebook-icse21}, in which a model
provides suggestions to complete partial code. Existing ML/DL-based
code completion models are just based on the code sequences or utilize
the syntactic structure in ASTs, but none leverage the program
dependencies due to the nature of partial code. Next, consider the
task of analyzing the code fragments in a bug report to connect it to
the relevant source files for bug localization
purposes~\cite{euler-fse19,icpc17}. Here too, a need for partial
program analysis, especially for partial program dependence analysis,
can be observed.








