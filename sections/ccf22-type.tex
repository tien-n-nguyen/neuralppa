\subsection{Ongoing Work on Type Inference for Partial Code}
\label{sec:statype}

\subsubsection{An Example of a StackOverflow Code Snippet}

\begin{wrapfigure}{l}{0.55\textwidth}
\begin{lstlisting}[basicstyle=\scriptsize\sffamily, stepnumber=1, numbers=left, numbersep=-6pt, framexleftmargin=0mm, framexrightmargin=0mm, language=Java, emph ={Event,sinkEvents,iframe,setEventListener,EventListener,StyleElement,Document,get,createStyleElement,setInnerText,resources,css,getText,getContentDocument,getHead,getBody,appendChild,ParagraphElement,createPElement,addClassName,whatever}]
    Event.sinkEvents(iframe, Event.ONLOAD);
    Event.setEventListener(iframe, new EventListener() {
        public void onBrowserEvent(Event event) {
            StyleElement style = Document.get().createStyleElement();
            style.setInnerText(resources.css().getText());
            iframe.getContentDocument().getHead().appendChild(style);

            ParagraphElement pEl = Document.get().createPElement();
            pEl.setInnerText("Hello World");
            pEl.addClassName(resources.css().getText());
            iframe.getContentDocument().getBody().appendChild(pEl);
        }
    });
\end{lstlisting}
\vspace{-0.1in}
\caption{StackOverflow post \#34595450 on GWT Widgets~\cite{soexample}}
\label{example}
\end{wrapfigure}

StackOverflow is a good resource of high quality code snippets that
show correct API usages. However, due to the discussion context and
the informal nature of the forum, code snippets rarely contain
sufficient declarations and references to fully-qualified names.

%This example illustrates three key challenges in resolving ype
%information.

Let us use a real-world example to illustrate the 
challenges in~resolving 
the fully qualified names (FQNs) 
for an SO code snippet. Figure~\ref{example} shows a
code snippet of an answer for an SO post
%\#34595450 
on Widgets in the GWT library. 

First, the snippets are not often embedded in methods or classes. In
Figure~\ref{example}, the code does not have an enclosing method or
class.

Second, they may refer to variables whose {\em declarations are not
  included}, and~their {\em identifiers are largely unqualified}. For
example, the variable \code{iframe} at lines 1, 2, 6, and 11 do not
have a declaration because the responder assumes that users would add
a declaration of type \code{IFrameElement} for \code{iframe} or add a
method call to the API \code{createIFrameElement} to get an object of
the type \code{IFrameElement}.
%
%\code{iframe} was declared earlier with the type \code{IFrameElement}
%a declaration of the type \code{IFrameElement} for \code{} 
%or an access to such an object by a prior call to
%\code{createIFrameElement}. 
%

Third, {\em the external references in code snippets are also often
  undeclared}. Specifically, code snippets frequently refer to
external types without specifying their FQNs since the
responder assumes that those FQNs could be implicitly understood in
the context of~the post.
%Tien
%
%assumes that users would add the required imported libraries when
%using the snippet in their programs in an IDE.
%
%
For example, the types \code{Event}, \code{EventListener},
\code{StyleElement}, \code{Document}, and \code{ParagraphElement} are
referenced by simple names only. Fourth, {\em name ambiguity occurs in
  the references to external libraries}. For example, the type
\code{EventListener} at line 2 is a common unqualified type name. In
this snippet, it refers to
\code{com.google.gwt.user.client.\-Event\-Listener}. However, it is
ambiguous with \code{java.util.EventListener} in JDK. Similarly, the
simple type name \code{Document} at line 4 is also popular
(\code{com.google.gwt.dom.client.Document},
\code{org.eclipse.jface.text.Document}, \code{org.w3c.dom.Document},
etc).
%Thus, if the code snippet is placed inside a class within an IDE, it
%will not be compiled due to missing \code{import} statements for
%needed packages.
%In fact, the phenomenon of 
In fact, external reference ambiguity is
popular~\cite{dagenais-icse12}.
%has been reported before.
%by Dagenais and Robillard~\cite{dagenais-icse12}: 89\% of method names
%in their study are ambiguous and on average, a method name has 13
%candidate FQNs.
%
%Subramanian {\em et al.}~\cite{liveapi14} reported that 
For example, the simple names \code{getId} and \code{Log} occur 27,434
and 284 times respectively in various Java libraries~\cite{liveapi14}.


%In a later version, it uses a dictionary of all the APIs of the
%libraries of interest. This is tedious to build a dictionary.


To address those ambiguities,
%in resolving the FQNs, for the elements in a code snippet,
existing approaches rely on rules and heuristics mainly from
program analysis. 
%
%ACE~\cite{rigby-icse13} focuses on resolving types for API names
%within texts via text analysis of the enclosing SO posts, thus, is
%inapplicable for this example.
%
RecoDoc~\cite{dagenais-icse12} uses heuristics in the partial program analysis tool,
PPA~\cite{dagenais-oopsla08}, 
%to apply heuristics on syntactic constructs to recover the declared
%types of certain expressions (\eg assignments). Unfortunately, 
however, in many cases, the heuristics help recover only partially
qualified names, due to much missing information. For example, at line
6, Figure~\ref{example}, the types of the receiver \code{iframe} and
method call \code{getContentDocument}, are not known. In contrast,
Baker~\cite{liveapi14} uses a dictionary of all the API names of
interest. Each API name has multiple candidates of FQNs. Baker
iteratively cuts off the candidates by overlapping the candidate lists
of API names in the same scopes in the code snippet. However, scoping
rule or other rules/heuristics are not always effective in incomplete
code snippets. In this example, lines 4--6 in Figure~\ref{example}
have only one scope with the popular API names having many potential
FQNs (\code{Document}, \code{getText}, \code{appendChild}). Moreover,
building and updating a dictionary of the APIs require
much~effort~\cite{dagenais-fse10}.


%{\em program analysis}, {\em text analysis}, or {\em deductive
%  analysis}, and/or the use of {\em an oracle of APIs of
%  interest}. 
%
%Specifically, partial program analysis (PPA)
%technique~\cite{dagenais-oopsla08} performs partial parsing and type
%resolution with several heuristics. PPA disambiguates the names by
%applying a set of heuristics on the syntactic constructs to recover
%the declared types of certain expressions in an incomplete code
%snippet. Unfortunately, in many cases, the heuristics help recover
%only partially qualified names, due to much missing information. For
%example, at line 6, the types of the receiver, method call, and
%arguments are not known.
%
%RecoDoc~\cite{dagenais-icse12} suffers the same issues since it uses
%PPA to link an API and documentation.
%
%ACE~\cite{rigby-icse13} focuses on resolving types for API names
%within texts via text analysis of the enclosing SO posts, thus, is
%inapplicable for this example. 
%Baker~\cite{liveapi14} uses a dictionary of all the API names of
%interest. Each API name has multiple candidates of FQNs.  Baker
%iteratively cuts off the candidates by overlapping the candidate lists
%of API names in the same scopes in the code snippet. However, scope
%rules are not always effective in incomplete code with few scopes or
%in popular names with many potential FQNs, \eg \code{Document},
%\code{getText}, \code{appendChild} on lines 4--6. Moreover, building
%and updating an oracle of the APIs require
%much~effort~\cite{dagenais-fse10}.

%=========================================================================
%To address those ambiguities in resolving the FQNs for the elements in
%a code snippet, existing approaches rely on {\em program analysis},
%{\em text analysis}, or {\em deductive analysis}, and/or the use of
%{\em an oracle of APIs of interest}.
%
%Specifically, partial program analysis (PPA)
%technique~\cite{dagenais-oopsla08} performs partial parsing and type
%resolution with several heuristics. PPA disambiguates the names by
%applying a set of heuristics on the syntactic constructs to recover
%the declared types of certain expressions in an incomplete code
%snippet. Unfortunately, in many cases, the heuristics help recover
%only partially qualified names, due to much missing information. For
%example, at line 1, the types of the receiver, method call, and
%arguments are not known. RecoDoc~\cite{dagenais-icse12} suffers the
%same issues since it uses PPA to link an API and documentation.
%ACE~\cite{rigby-icse13} focuses on resolving types for API names
%within texts via text analysis of the enclosing SO posts, thus, is
%inapplicable for this example. Baker~\cite{liveapi14} uses a
%dictionary of all the API names of interest. Each API name has
%multiple candidates of FQNs.  Baker iteratively cuts off the
%candidates by overlapping the candidate lists of API names in the same
%scopes in the code snippet. However, scope rules are not always
%effective in incomplete code with few scopes or in popular names with
%many potential FQNs, \eg \code{Document}, \code{getText},
%\code{appendChild} on lines 4--6. Moreover, building and updating an
%oracle of the APIs require much~effort~\cite{dagenais-fse10}.
%=======================================================================

%--------------------------------------------------------------
%To address those ambiguities in resolving the FQNs for the program
%elements in a code snippet, existing approaches rely on {\em program
%  analysis}, {\em text analysis}, and/or the use of {\em an oracle of
%  APIs of interest}. Specifically, partial program analysis (PPA)
%technique~\cite{dagenais-oopsla08} performs partial parsing and type
%resolution with several heuristics. PPA disambiguates the names by
%applying a set of heuristics on~the syntactic constructs to recover
%the declared types of certain expressions in an incomplete code
%snippet. Unfortunately, in many cases,~the heuristics help recover
%only partially qualified names, due to much missing information.  For
%example, at line 1, the types of the receiver, method call, and
%arguments are not known. RecoDoc~\cite{dagenais-icse12} suffers the
%same issues since it uses PPA to link an API and documentation. With
%context analysis, ACE~\cite{rigby-icse13} analyzes the API names by
%considering the {\em textual contexts} surrounding them in SO
%posts. However, it might not be always effective since the surrounding
%texts might not be sufficient and code snippets were not analyzed.
%Baker~\cite{liveapi14} uses a dictionary of all the API names in the
%libraries of interest. Each API name has multiple candidates of FQNs.
%Baker iteratively cuts off the candidates by overlapping the candidate
%lists of multiple names in the same scopes in the code
%snippet. However, scope rules are not always effective in incomplete
%code or in popular names with many potential FQNs, \eg
%\code{Document}, \code{getText}, \code{appendChild} on lines 4--6.
%Thus, after its process, the resulting API names from Baker still have
%multiple potential FQNs. Moreover, building and updating an oracle of
%the APIs require much~effort~\cite{dagenais-fse10}.
%-----------------------------------------------------------------

%is also time-consuming.

%
% many API elements have
% more than one potential fully qualified type names

%problem with oracle
%overlapping

%\vspace{-0.14in}
\subsubsection{Key Ideas in {\tool}}

%type references, method calls, and field references.

%resolve the simple names

As part of {\tool}, we will develop a technique to identify the FQNs for
the simple names of {\em variables, API classes, method calls, and
  field accesses} in a code snippet in online forums.
%
In our solution, we leverage two kinds of context in deriving FQNs for
API elements. 

The first one is the {\em type context} of API elements in a
large code corpus. We call it the principle of {\em regularity of API
  elements}.
%FINAL
%Hindle {\em et al.}~\cite{natural} has shown that software exhibits
%its naturalness: source code has a higher degree of regularity than
%English texts. We expect the same principle of naturalness of software
%to occur for the API elements in the API code usages. 
That is, in the API usages, the API elements including API classes,
method calls, and field accesses of software libraries do not occur
randomly. They appear regularly with certain other API elements due to
the intent of the libraries' designers to provide certain
functionality in API usages. For example, to manipulate style
information for a frame in GWT, at lines 4--6 in Figure~\ref{example},
a \code{StyleElement} in GWT could be created via a call to
\code{createStyleElement} of a \code{Document} object. Then, the style
is assigned with a name via \code{StyleElement.setInnerText} and
associated with an \code{IFrameElement} object. Those API elements of
those types repeatedly occur together because such tasks are provided
by the GWT library and intended via such API usages. Thus, if {\tool}
observes the API usages involving those API elements in a large corpus
of programs using GWT, it can use the context of the given code
snippet to derive the most likely FQNs for the APIs in the
snippet. For example, the context of the API names
\code{StyleElement}, \code{Document}, and \code{createStyleElement}
could give {\tool} a hint on the FQNs of the undeclared variable
\code{iframe}, the method calls \code{getContentDocument},
\code{appendChild}, and vice versa.

To learn the context, we rely on the basis of regularity with a large
corpus: the API elements regularly going together in API
usages have higher impact in deciding the FQNs than the less
regular~ones.

%ones that
%are project-specific and less likely to appear frequently~together.

%To learn the context, we rely on the basis of regularity with a large
%corpus containing the APIs of interest. That is, the API elements that
%regularly go together in API usages will have higher impact in
%deciding the FQNs than the ones that are project-specific and less
%likely to appear frequently~together. 

%
%For example, with a large corpus of programs using GWT, a model could
%learn that \code{com.google.gwt.user.client.Event.setEventListener}
%often goes with \code{com.google.gwt.user.client.Event.EventListener},
%and \code{com.google.\-gwt.user.client.Event.sinkEvents}.

The second context is called {\em resolution context} on how the types
of API names surrounding an API name have been resolved. The
undeclared/ambiguous references are expected to be disambiguated with
{\em the most likely FQNs with regard to the FQNs of the surrounding
  names}.
%
For example, at lines 3--4, when resolving \code{Event},
\code{StyleElement}, and \code{Document}, we could use an effective
search strategy to maintain the sequences of FQNs with the highest
probabilities. For \code{Event}, we have multiple candidates. However,
if we chose \code{com.google.gwt.user.client.Event} for \code{Event},
then based on their regularity, we could maintain the option
\code{com.google.gwt.dom.client.Document} for \code{Document}, with a
higher score than \code{org.w3c.dom.Document}. Finally, the sequence
with the highest score will be reported as the~result.

%Then, we use it to decide the type for the next API.

%with a large corpus of programs using GWT, a model could
%learn that \code{com.google.gwt.user.client.Event.setEventListener}
%often goes with \code{com.google.gwt.user.client.Event.EventListener},
%and \code{com.google.\-gwt.user.client.Event.sinkEvents}.



% ---- old ---
%In this work, to realize that philosophy, we use a data-driven
%approach with the two following key ideas:

%1) In the context of API usages, certain API elements need to be used
%together because they have to follow the API specifications. For
%example, with a large corpus of programs using GWT, a model could
%learn that \code{com.google.gwt.user.client.Event.setEventListener}
%often goes with \code{com.google.gwt.user.client.Event.EventListener},
%and \code{com.google.\-gwt.user.client.Event.sinkEvents}.
%To learn the contexts, we rely on the {\bf basis of regularity} with a
%large corpus containing the APIs of interest. That is, the API
%elements that regularly go together in API usages will have higher
%impact in deciding the FQNs than the ones that are project-specific
%and less likely to appear frequently~together.

%2) To narrow the candidate FQN list for an API name, we could use the
%{\bf context of its surrounding API names}. The undeclared/ambiguous
%references are expected to be disambiguated with the most likely FQNs
%with regard to the FQNs for surrounding names.
%------------------------------------------------------------

%
%\code{com.google.gwt.user.client.Event.sinkEvents} receives two
%arguments of the types \code{com.google.gwt.dom.client.Element} and
%\code{int}, and it often goes with
%\code{com.google.gwt.user.client.Event.setEventListener}, etc.
%
%For example, at line 1 of Figure~\ref{example}, even though we do not
%have the declaration of \code{iframe}, however, with the large corpus
%of open-source projects, the model could learn that the method
%\code{com.google.gwt.user.client.Event.sinkEvents} receives two
%arguments of the type \code{com.google.gwt.dom.client.Element} and
%\code{int}, and it often goes with
%\code{com.google.gwt.user.client.Event.setEventListener}.


%Tien
%\input{rules.tex}
%\input{smt-motiv}



%2) {\em Second}, because the translation takes into account {\bf the
%  context consisting of surrounding API elements/names}, the
%undeclared/ambiguous references will be disambiguated with the most
%likely FQNs with regard to the FQNs of their surrounding names.

%3) ambiguity: Third, not all the changes in the current context are useful in
%recommendation because they can be project-specific and considered
%as noise in the change patterns. Recent work [41] confirms
%that not all code changes are repetitive. To address this, we rely on
%the basis of consensus: given a large number of changes in many
%projects, the project-specific changes are less likely to appear frequently
%than the changes belonging to a higher-level intent pattern.
%-==> undeclared types
