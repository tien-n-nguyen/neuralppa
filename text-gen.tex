\subsection{Generation of Textual Explanations}
\label{sec:tex}

We will use the state-of-the-art code summarization
approaches~\cite{shi-etal-2021-cast,chai2022pyramid} that take the
statements in the PDG sub-graph (explainable subgraphs as in
Figure~\ref{fig:pdg}), and produces the texts to explain the code. The
key challenges in comparison to the code summarization is that the
explanation needs to focus on the data and control flows that pertain
to the vulnerability. To overcome that, we aim to take advantage of
the textual description of the vulnerability provided in the public
vulnerability database such as CVSS~\cite{first-website},
CVE~\cite{cve}, etc for training. For example, we expect to provide
the textual explanation for the above code as {\em ``Race condition
  with double fetching that can cause out-of-bounds array access by
  changing a certain size value''}.

%Race condition in the ec_device_ioctl_xcmd function in drivers/platform/chrome/cros_ec_dev.c in the Linux kernel before 4.7 allows local users to cause a denial of service (out-of-bounds array access) by changing a certain size value, aka a "double fetch" vulnerability.
